\documentclass[12pt]{article}
\usepackage[margin=1.0in]{geometry}
\usepackage{amsmath, amsthm}
\usepackage{times}
\usepackage{newtxmath}
\usepackage{cleveref}
\usepackage{enumitem}




\newtheorem{theorem}{Theorem}
\newtheorem{corollary}{Corollary}

\theoremstyle{definition}
\newtheorem{remark}{Remark}
\newtheorem{definition}{Definition}
\newtheorem{conjecture}{Conjecture}
\newtheorem{assumption}{Assumption}
\newtheorem{setting}{Setting}
\newtheorem{proposition}{Proposition}
\newtheorem{situation}{Situation}


\pagestyle{plain}



\begin{document}

\title{$p$-adic Local Langlands Correspondence}
\author{Xin Tong}
\date{}




\maketitle



\begin{abstract}
\noindent We discuss symmetrical monoidal $\infty$-categoricalizations in relevant $p$-adic functional analysis and $p$-adic analytic geometry. Our motivation has three sources relevant in $p$-adic local Langlands correspondence: one is the corresponding foundation from Bambozzi-Ben-Bassat-Kremnizer on derived functional analysis and Clausen-Scholze on derived topologicalization, and the second one is representations of derived $E_1$-rings which is relevant in integral deformed version of the $p$-adic local Langlands correspondence with eventually Banach coefficients, and the third one is the corresponding comparison of the solid quasicoherent sheaves over two kinds of generalized prismatization stackifications over small arc stacks and small $v$-stacks: one from the de Rham Robba stackification, and the other one from the de Rham prismatization stackification. Small arc stacks imperfectize the prismatization stackification, which will then perfectize the prismatization stackification when we regard them as small $v$-stacks. After Scholze's philosophy, one can in fact imperfectize those significant $v$-stacks in Fargues-Scholze to still have the geometrization at least by using Berkovich motives. Both small arc-stacks and small $v$-stacks can be studied using local totally disconnectedness which is the key observation for condensed mathematics, theory of diamonds and perfectoid rings. Following Scholze, Richarz-Scholbach and Ayoub we then study the $p$-adic local Langlands correspondence by using $p$-adic motivic cohomology theories. We study in some uniform way many significant $p$-adic motivic cohomology theories in families, after the general framework and formalism in the recent work of Ayoub. We extend in some sense Ayoub's formalism after Scholze's recent theory of Berkovich motives, Scholze's theory of small $v$-stacks and Clausen-Scholze's analytic stacks. We then make detailed discussion on those significant motivic cohomology theories, around perspectives for instance on: K\"unneth theorem, 6-functor formalism and so on.   
\end{abstract}


\newpage
\tableofcontents


\newpage
\section{Introduction}



\indent Emerton-Gee-Hellmann's \textit{$p$-adic categoricalization}, Fargues-Scholze's \textit{geometrization} as in \cite{1FS}, \cite{EGH1}, and Scholze's \textit{motivicalization} as in \cite{1S5}, \cite{1S6} by using symmetrical monoidal $\infty$-categoricalization, motivate us to consider the generalization we will present in this paper on Langlands program after \cite{L1}, \cite{D}, \cite{C}. We will consider essential \textit{de Rham prismatization} as in our previous work \cite{1T}. In both settings (one $p=\ell$ and considering $p$-adic Banach space representations, the other one $p$ is allowed to be away from $\ell$) even if we do not consider Banach representations, we consider however extensively $p$-adic coefficients. Therefore we call them all \textit{$p$-adic Local Langlands program}. But our ultimate goal is to parametrize all the corresponding possible condensed representations of all the possible \textit{motivic Galois groups} as in \cite{2A} by using motivic Hopf algebras over schemes, formal schemes, rigid analytic spaces, adic spaces, Berkovich spaces, small $v$-stacks, small arc stacks, derived small arc stacks, condensed analytic stacks and so on: 
\begin{itemize}
\item[A] $p$-adic constructible condensed sheaves with the attached motivic Galois groups $\mathrm{Spec}\mathrm{Hopf}_{p-\text{adic},\blacksquare}$ with the motivic 6-functor formalism ($F_1$,$F_2$,($F_3$,$F_4$,*-adjoint pairs),($F_5$,$F_6$,!-adjoint pairs));
\item[B] $\ell$-adic constructible condensed sheaves with the attached motivic Galois groups $\mathrm{Spec}\mathrm{Hopf}_{\ell-\text{adic},\blacksquare}$ with the motivic 6-functor formalism ($F_1$,$F_2$,($F_3$,$F_4$,*-adjoint pairs),($F_5$,$F_6$,!-adjoint pairs));
\item[C] $p$-adic $+$-de Rham lattice deformed quasi-coherent sheaves with the attached motivic Galois groups $\mathrm{Spec}\mathrm{Hopf}_{\mathrm{dR},+,\text{prismatization}}$, $\mathrm{Spec}\mathrm{Hopf}_{\mathrm{dR},\mathrm{Robba},+,\text{stackification}}$ with the motivic 6-functor formalism ($F_1$,$F_2$,($F_3$,$F_4$,*-adjoint pairs),($F_5$,$F_6$,!-adjoint pairs));
\subitem[C1] Through de Rham prismatization;
\subsubitem[C11] Over small arc-stacks, under arc topology;
\subsubitem[C12] Over small $v$-stacks, under $v$-topology;
\subitem[C2] Through de Rham Robba stackification from pre-Fargues-Fontaine curves;
\item[D] Solid quasicoherent sheaves over Robba rings without Frobenius and the attached motivic Galois groups $\mathrm{Spec}\mathrm{Hopf}_{\mathrm{solid}, \mathrm{quasicoherent},\mathrm{Robba},\blacksquare}$ with the motivic 6-functor formalism ($F_1$,$F_2$,($F_3$,$F_4$,*-adjoint pairs),($F_5$,$F_6$,!-adjoint pairs));
\subitem[D1] Over small arc-stacks, under arc topology;
\subitem[D2] Over small $v$-stacks, under $v$-topology;
\item[E] ... and more motivic cohomology theories satisfying the formalism in \cite{2A} with the motivic 6-functor formalism ($F_1$,$F_2$,($F_3$,$F_4$,*-adjoint pairs),($F_5$,$F_6$,!-adjoint pairs))...
\end{itemize}
Also we believe that ultimately there will be some very well-defined generalizations of Colmez's Montr\'eal foncteur in \cite{C} by using the methods we are considering. We use the techniques of prismatization and its various de Rham stackified versions. Symmetrical monoidal $\infty$-categories in $p$-adic analytic geometry and $p$-adic functional analysis are very significant, though from a higher categorical perspective especially when we have the stability. Many problems emerge in $p$-adic analytic geometry and $p$-adic functional analysis if one sticks to the usual derived categories, such as the lack of enough projective objects and many tricky points of view in the modular representation theory even over $\mathbb{Z}_p/p^n$, let alone taking the inverse limit over the index $n\in \mathbb{Z}$. In this paper we discuss some specific aspects in the $p$-adic Hodge theory and the related $p$-adic functional analysis by using some well-defined $\infty$-categories which carry the corresponding symmetrical monoidal $\infty$-categorical structures. In \cite{T} the author in the derived category studied some generalization of the results in \cite{KPX} by using $p$-adic functional analysis method dated back to \cite{K1}. The resulting derived categorical consideration was of course following \cite{KPX}. However there should be many $\infty$-categorical considerations now which can be better illustrate the issues following \cite{BBBK}, \cite{BBKK}, \cite{CS1}, \cite{CS2}, \cite{CS3} by using certain symmetrical monoidal $\infty$-categories where more robust properties can hold. With the notations in \cref{definition1} and \cref{definition2} we have rings of analytic functions over those rigid affinoids as in \cite{T} after \cite{KPX}, \cite{CKZ}, \cite{PZ}, \cite{1Z}. Then over these rings we have the following well-defined categories, many of which are stable symmetrical monoidal $\infty$-categories.


\begin{definition}
Over $*$=
\begin{align}
L^X_{[a_I,b_I],K_I}(\pi_{K_I}),L^X_{b_I,K_I}(\pi_{K_I}):=L^X_{(0,b_I],K_I}(\pi_{K_I}),L^X_{[0,b_I],K_I}(\pi_{K_I}),L^X_{K_I}(\pi_{K_I}):=\bigcup_{b_I>0}\bigcap_{a_I>0}L^X_{[a_I,b_I],K_I}(\pi_{K_I}),
\end{align} 
we have the notion of $\Gamma$-Frobenius modules and bundles. Then we consider the derived $\infty$-categories 
\begin{align}
\underline{\mathrm{IndBanachModules}}^\sharp_*, \underline{\mathrm{Ind_{\mathrm{mono}}BanachModules}}^\sharp_*.
\end{align}
which are stable. And we have the condensed version:
\begin{align}
\blacksquare D_{*}, \blacksquare D_{*,\mathrm{bounded}},\blacksquare D_{*,\mathrm{perfect}}.
\end{align}
Therefore we have the corresponding Banach perfect complexes and condensed perfect complexes of the corresponding objects in our setting, namely we consider the perfect complexes of the $\Gamma$-Frobenius-Hodge modules in our setting, which again form certain stable $\infty$-categories with symmetrical monoidal structures. In the Banach setting we use the notations:
\begin{align}
&\underline{\mathrm{IndBanachModules}}^\sharp_{*,\mathrm{perfect},\Gamma,F}, \underline{\mathrm{Ind_{\mathrm{mono}}BanachModules}}^\sharp_{*,\mathrm{perfect},\Gamma,F},\\
&\underline{\mathrm{IndBanachModules}}^\sharp_{*,\mathrm{perfect},\mathrm{bounded},\Gamma,F}, \underline{\mathrm{Ind_{\mathrm{mono}}BanachModules}}^\sharp_{*,\mathrm{perfect},\mathrm{bounded},\Gamma,F},\\
&\underline{\mathrm{IndBanachModules}}^\sharp_{*,\mathrm{perfect},-,\Gamma,F}, \underline{\mathrm{Ind_{\mathrm{mono}}BanachModules}}^\sharp_{*,\mathrm{perfect},-,\Gamma,F},
\end{align}
to denote these complexes. And in the corresponding condensed setting we use:
\begin{align}
\blacksquare D_{*,\Gamma,F}, \blacksquare D_{*,\mathrm{bounded},\Gamma,F},\blacksquare D_{*,\mathrm{perfect},\mathrm{bounded},\Gamma,F},\blacksquare D_{*,\mathrm{perfect},\Gamma,F},\blacksquare D_{*,\mathrm{perfect},-,\Gamma,F}.
\end{align}
\end{definition}

\indent Then those derived $\infty$-sheaves in the categories above can have those certain derived cohomologies by using the $F$ and $\Gamma$-structure, by taking those iterated Yoneda groups, which are the corresponding derived $F,\Gamma$-cohomologies as in the following definition:

\begin{definition}
If $I$ is singleton, then we have the notion of $(F,\Gamma)$-complex of any $\Gamma$-Frobenius-Hodge module $F$: $C_{F,\Gamma}(F)$, where we have also the $C_{F,*}(F)$ and $C_{*,\Gamma}(F)$ complexes as well. Using them by induction we have the corresponding notion of $(F,\Gamma)$-complex of any $\Gamma$-Frobenius-Hodge module $F$: $C_{F,\Gamma}(F)$ when $I$ is not just a singleton. In our setting for each $i\in I$ we also have the corresponding $W_i=\varphi_i^{-1}$-operator. In such a way one can form the complex $C_{W}(F)$ directly.
\end{definition}

Following ideas in \cite{T}, \cite{KPX} we consider the symmetrical monoidal $\infty$-categories above instead to study the targeted categories where the cohomology groups are living, due to the fact that they are stable and usually having Grothendieck homotopy triangulated categories, many problems after \cite{T} and \cite{KPX} are solved, even over quite hard quasi-Stein spaces. Of course we are taking about condensed quasi-Stein spaces not the usual ones.

\begin{theorem}
$C_{F,\Gamma}(F)$ is in bounded $(\infty,1)$-derived category of complexes over $X$, restricting to perfect complexes:
\begin{align}
D_{\mathrm{perfect},\mathrm{bounded}}(\mathrm{Mod}_X).
\end{align}
$C_{F,\Gamma}(F)$ is in 
\begin{align}
\underline{\mathrm{IndBanachModules}}^\sharp_{X,\mathrm{perfect},\mathrm{bounded}}, \underline{\mathrm{Ind_{\mathrm{mono}}BanachModules}}^\sharp_{X, \mathrm{perfect},\mathrm{bounded}}.
\end{align} 
$C_{W}(F)$ is in 
\begin{align}
\underline{\mathrm{IndBanachModules}}^\sharp_{L^X_{\infty_I,K_I}(\Gamma_{K_I}),\mathrm{perfect},\mathrm{-}}, \underline{\mathrm{Ind_{\mathrm{mono}}BanachModules}}^\sharp_{L^X_{\infty_I,K_I}(\Gamma_{K_I}), \mathrm{perfect},\mathrm{-}},
\end{align}
where $X$ is just $\mathbb{Q}_p$.
\end{theorem}

\begin{theorem}
$C_{F,\Gamma}(F)$ is in bounded $(\infty,1)$-derived category of complexes over $X$, restricting to perfect complexes:
\begin{align}
D_{\mathrm{perfect},\mathrm{bounded}}(\mathrm{Mod}_X).
\end{align}
$C_{F,\Gamma}(F)$ is in 
\begin{align}
\blacksquare\underline{D}_{X,\mathrm{perfect},\mathrm{bounded}}.
\end{align} 
$C_{W}(F)$ is in 
\begin{align}
\blacksquare\underline{D}_{L^X_{\infty_I,K_I}(\Gamma_{K_I}),\mathrm{perfect},-},
\end{align}
where $X$ is just $\mathbb{Q}_p$.
\end{theorem}

\begin{corollary}
$C_{W}(.)$ induces a derived functor from the stable $\infty$-category 
\begin{align}
\underline{\mathrm{IndBanachModules}}^\sharp_{*,\mathrm{perfect},\mathrm{bounded},\Gamma,F}, \underline{\mathrm{Ind_{\mathrm{mono}}BanachModules}}^\sharp_{*,\mathrm{perfect},\mathrm{bounded},\Gamma,F},
\end{align}
and in the corresponding condensed setting:
\begin{align}
\blacksquare D_{*,\mathrm{perfect},\mathrm{bounded},\Gamma,F},
\end{align}
to the stable $\infty$-category 
\begin{align}
\blacksquare\underline{D}_{L^X_{\infty_I,K_I}(\Gamma_{K_I}),\mathrm{perfect},-},
\end{align}
where $X$ is just $\mathbb{Q}_p$. Here $*=L_{K_I}^X(\pi_{K_I})$. This also induces the morphism on the $K$-group spectra of $\mathbb{E}_\infty$-rings after applying \cite{BGT} to the corresponding stable $(\infty,1)$-categories, after \cite{G2}, \cite{A2}, \cite{BGT}.
\end{corollary}



\indent Here $X$ is some rigid affinoid algebra, which is actually making the context closely related to \cite{Z1}, \cite{ST}. However we find the integral model is also significant by taking the integral model of $X$, namely $X^+$ which is some first of all some $p$-adic $\mathbb{Z}_p$-algebra, then taking the reduction we have some $\mathbb{Z}_p/p^n$-algebra and finally we have some $\mathbb{F}_p$-algebra. Then the construction in such relative setting is closely related to \cite{Sc1}, \cite{So1}, \cite{SS1}, \cite{SS2}, \cite{HM} as well, where we consider some other significant related symmetrical monoidal $\infty$-categores as well as in \cite{Sc1}, \cite{So1}, \cite{SS1}, \cite{SS2}, \cite{HM} with general coefficients in $X^+$. In general over $G$ a reductive $p$-adic Lie group we have the smooth representations in the coefficient in $X^+$, such as the group $\Gamma_{K_I}$ in the above rigid analytic geometric consideration. The corresponding derived $\infty$-categories in coefficients $X^+$ are  Grothendieck symmetrical monoidal $\infty$-categories as in \cite{SS1}, \cite{SS2}, \cite{HM}. In \cite{Sc1} Schneider defined $E_1$-version of the usual Hecke algebra with respect to some compact open. The point is that the derived categories over $E_1$ Hecke algebras are actually related to the representations of the group $G$ in some direct manner. As in \cite{So1} we consider one step further, i.e. we try to find minimal $E_1$-model for the homology of the $E_1$-Hecke algebra. However we need to use derived \textit{$E_1$-rings} in \cite{Sa} to do this, which looks to go into some different direction from \cite{So1} due to the fact that we have $E_1$-rings over a general commutative ring $X^+$. After \cite{Sc1}, \cite{So1}, \cite{SS1}, \cite{SS2}, \cite{HM} we have the following:


\begin{conjecture}\mbox{\textbf{(After Sorensen, \cite[Theorem 1.1]{So1})}}
Let $G$ be compact as in \cite{So1}\footnote{Since \cite{So1} mentioned that one can generalize this to more general setting by using general dg Hecke algebras, this theorem can also be generalized to more general setting.}. Let $X^+$ be a $p$-adic $\mathbb{Z}_p$-formal algebra. $X^+$ can be written as a formal projective limit over $\mathrm{Z}_p/p^n$-algebras. Let $H$ be the dg Hecke algebra over $X^+$ defined as in \cite{Sc1}. Promoting $H$ to a derived $E_1$-ring $\mathbb{H}$ as in \cite{Sa}, we have the derived $\infty$-category of the minimal derived $E_1$-ring as in \cite[Theorem 1.2]{Sa}
\begin{align}
\underset{{\mathrm{totalized},\mathbb{H},\mathbb{H}}}{\mathrm{homomorphism}}
\end{align}
admits a functor from and a functor into the derived $\infty$-category
\begin{align}
D\mathrm{R}_{\mathrm{lisse},X^+,G}.
\end{align}
The latter is the derived $\infty$-category of all the $G$-$X^+$-smooth modules. We conjecture all the functors here are equivalences of symmetrical monoidal $\infty$-categories.
\end{conjecture}

We then consider the deformation functors from these well-defined functors. Then we work in the context of \cite{1S5}, \cite{1S6} where we consider the corresponding two different types of stackifications in the sense of prismatization, after in one situation \cite{1To1}, \cite{1To2}, \cite{1To3}, \cite{1To4}, \cite{1S4}, \cite{1ALBRCS}, \cite{1BS}, \cite{1D}, \cite{1BL}, and after in other situation \cite{1KL}, \cite{1KL1}, \cite{1S1}, \cite{1S2}, \cite{1S3}, \cite{1F1}, \cite{1F2}, \cite{1T1}, \cite{1CSA}, \cite{1CSB}, \cite{1CS}, \cite{1BS1}, \cite{1BL1}, \cite{1BL2}. Again the derived $\infty$-categories we consider in this topic are stable symmetrical monoidal ones. 

\begin{remark}
The $p$-adic aspects of the local Langlands program we presented here (though not very sure if this is the ultimate correct approach to generalize Colmez's work) are deeply inspired by the work of Emerton-Gee-Hellmann \cite{EGH1} where a categoricalization is conjectured. Emerton-Gee-Hellmann kept one side of the correspondence almost the same as in \cite{1FS}, i.e. the smooth representations with coefficient in at least $\mathbb{Z}_p$-coefficients, then one can consider locally analytic representations and ultimately consider Banach representations. However the other side of the fully-faithful functor conjecture in \cite{EGH1} used stacks related to \cite{KPX}, i.e. the arithmetic stacks of $(\varphi,\Gamma)$-modules over Robba rings. We consider the motivic Tannakina categorical consideration then from de Rham prismatization. 
\end{remark}

\begin{theorem}
Assume we are in our general setting by adding the element $b^{1/2}$. The de Rham-Robba stackification and the de Rham-prismatization stackification in our generalized setting by adding $b^{1/2}$ are equivalent, in both $p$-adic and $z$-adic settings, i.e. in the $p$-adic setting we consider the small $v$-stacks over $\mathrm{Spd}\mathbb{Q}_p$, and in the $z$-adic setting we consider the small $v$-stacks over $\mathrm{Spd}\mathbb{F}_p((t))$, in the $v$-topology. This applies immediately to rigid analytic varieties.
\end{theorem}


\begin{definition}
Assume we are in our general setting by adding the element $b^{1/2}$. Let $S$ be a small $v$-stack, which can be either over $\mathrm{Spd}\mathbb{Q}_p$ or $\mathrm{Spd}\mathbb{F}_p((u))$. We use the notation:
\begin{align}
\mathrm{deRhamRobba}_S
\end{align}
to denote the corresponding de Rham-Robba stackification from the FF stacks, in our generalized setting. And we use the notation
\begin{align}
\mathrm{deRhamPrismatization}_S
\end{align}
to denote the corresponding de Rham Prismatization stackification, in our generalized setting. And for $?= \mathrm{deRhamRobba}_S, \mathrm{deRhamPrismatization}_S$ we use the notation:
\begin{align}
\mathrm{SolidQuasiCoh}_?
\end{align}
to denote the corresponding condensed $\infty$-categories of the corresponding solid quasicoherent sheaves over $?$. Then we have a functor:
\begin{align}
\mathrm{SolidQuasiCoh}_{\mathrm{deRhamPrismatization}_S}\rightarrow \mathrm{SolidQuasiCoh}_{\mathrm{deRhamRobba}_S}
\end{align}
by taking the induced functor from identification of the de Rham functors on the perfectoids.
\end{definition}


\begin{theorem}
Assume we are in our general setting by adding the element $b^{1/2}$. The functor defined above:
\begin{align}
\mathrm{SolidQuasiCoh}_{\mathrm{deRhamPrismatization}_S}\rightarrow \mathrm{SolidQuasiCoh}_{\mathrm{deRhamRobba}_S}
\end{align}
is well-defined, and an equivalence of symmetrical monoidal $\infty$-categories which are stable.

\end{theorem}


\begin{definition}
Assume we are in our general setting by adding the element $b^{1/2}$. Let $S$ be a small arc-stack in \cite{1S5} over $\mathbb{Q}_p$ or $\mathbb{F}((u))$. We use the notation:
\begin{align}
\mathrm{deRhamRobba}_S
\end{align}
to denote the corresponding de Rham-Robba stackification from the FF stacks, in our generalized setting. And we use the notation
\begin{align}
\mathrm{deRhamPrismatization}_S
\end{align}
to denote the corresponding de Rham Prismatization stackification, in our generalized setting. And for $?= \mathrm{deRhamRobba}_S, \mathrm{deRhamPrismatization}_S$ we use the notation:
\begin{align}
\mathrm{SolidQuasiCoh}_?
\end{align}
to denote the corresponding condensed $\infty$-categories of the corresponding solid quasicoherent sheaves over $?$. When we consider the de-Rham Robba stackification we consider the corresponding $v$-stack associated to $S$, which is denoted by $\mathrm{Stack}_v(S)$ after \cite{1S5}. Then we have a functor:
\begin{align}
\mathrm{SolidQuasiCoh}_{\mathrm{deRhamPrismatization}_S}\rightarrow \mathrm{SolidQuasiCoh}_{\mathrm{deRhamRobba}_{\mathrm{Stack}_v(S)}}
\end{align}
by taking the induced functor from identification of the de Rham functors on the perfectoids, i.e. we set the Banach ring local chart to be perfectoid to reach the objects in the second $\infty$-category. 
\end{definition}


\begin{theorem}
Assume we are in our general setting by adding the element $b^{1/2}$. The functor defined above:
\begin{align}
\mathrm{SolidQuasiCoh}_{\mathrm{deRhamPrismatization}_S}\longrightarrow \mathrm{SolidQuasiCoh}_{\mathrm{deRhamRobba}_{\mathrm{Stack}_v(S)}}
\end{align}
is well-defined, as a symmetrical monoidal $\infty$-tensor functor.
\end{theorem}



\begin{theorem}
Assume we are in our general setting by adding the element $b^{1/2}$. The functor defined above:
\begin{align}
\mathrm{SolidQuasiCoh}_{\mathrm{deRhamPrismatization}_S}\longrightarrow \mathrm{SolidQuasiCoh}_{\mathrm{deRhamRobba}_{\mathrm{Stack}_v(S)}}
\end{align}
is fully faithful functor of symmetrical monoidal $\infty$-categories which are stable. 
\end{theorem}


\indent We then apply these consideration of de Rham lattice deformations of the $\overline{\mathbb{Q}}_p$-local systems to the local Langlands program after \cite{L1}, \cite{1FS}, \cite{1VL}, \cite{1GL}, \cite{D}, \cite{1LL}, \cite{D2}. Then we promote the discussion to the motivic level after \cite{1S5}, \cite{1S6}, \cite{2LH}, \cite{2A}, \cite{1RS} and construct very general $t$-adic local Langlands correspondence in family after \cite{2LH} by using the motivic Galois groups fibered over $\mathbb{N}^\wedge$. Then as more thorough discussion on motivic cohomology theories, we study in some uniform way many significant $p$-adic motivic theories in families after \cite{3G}, \cite{3A}, \cite{31A}, \cite{3V}, after the general framework and formalism of Ayoub \cite{3A}. We extend in some sense Ayoub's formalism after Scholze's theory of Berkovich motives \cite{3S}, \cite{3S2}. The $p$-adic motivic cohomology theories we are considering are: prismatizations (filtration, syntomification) and stackifications induced from them (de Rham, de Rham-Hodge-Tate, Laurent) in families after \cite{3BS}, \cite{3BL}, \cite{3D}. 


\begin{remark}
We choose to work in families after \cite{3LH} due to some reason to extend certain mixed-characteristic constructions to function field, and \textit{vice versa} extend certain funtion field constructions to mixed-characteristic situations.
\end{remark}


\begin{remark}
We work in some combined and vast generalized manner after \cite{3A} and \cite{3S} in the following way. First \cite{3A} considered up to rigid analytic varieties, and found correspondence from this to the classical motivic theory for schemes. Second \cite{3S} obviously generalized \cite{3A}, but we try to uniformize certain consideration after Scholze in the fashion of \cite{3A} to use certain general Weil cohomology theories over Banach rings, in order to apply to the considerations we are considering. 
\end{remark}


\begin{setting}
We consider the setting as in \cite{3LH}. Let $f$ be some finite field, then we consider a bunch of local fields in mixed-characteristics away from $\mathbb{N}_\infty - \mathbb{N}= \{\infty\}$:
\begin{align}
L1,L2,...
\end{align}
such that we have a profinite version of local rings, where $L_n,n\in \mathbb{N}$ are all local fields over $f$ with characteristic $0$. We assume the growth on the ramification index. Then $L_\infty$ is a function field over $f$. We call the product $L$ throughout the union of $\mathbb{N}$ with the infinity and we call the ring of integer $A$. $z$ is a general uniformizer which will then turn to be $z_n$ for each $n\in \mathbb{N}_\infty$.
\end{setting}

Motives \textit{in families} in our current consideration will be the most generalized sense after \cite{3S}, \cite{3A}, i.e. the generalized motives with coefficients. We start from a category of stacks over some fixed stack $X$ in families over $A$. We consider analytic condensed stacks in families (i.e. with certain structure morphism to $\mathbb{N}_\infty$) in analytic Grothendieck toplogy, we consider small arc stacks in families in arc Grothendieck topology, we consider small $v$-stacks in families in $v$ Grothendieck topology, which form the corresponding big Grothendieck sites over $X$. Over these Grothendieck sites we consider construction of motives in the sense of \cite{3S}, \cite{3A} namely those $\infty$-presheaves of modules over $A$-algebras. Usually we have 5 steps in constructing $\infty$-category (symmetrical monoidal) of motivic sheaves with coefficients (in families as well over $A$) following \cite{3G}, \cite{3V}, \cite{3A}, \cite{3S}:
\begin{itemize}
\item[1] Fix an $\infty$-category of $(\infty,1)$-stacks over $X$ with a fixed Grothendieck topology $T$, which forms a Grothendieck site, $\mathrm{Site}_{G,T,X}$;
\item[2] Start from the $(\infty,1)$-sheaf of ring $\mathrm{Sh}_A$ over $A$ over $\mathrm{Site}_{G,T,X}$, and consider the $\infty$-category of all the $\infty$-sheaves over $\mathrm{Sh}_A$, $\mathrm{Cat}_1$;
\item[3] Take the corresponding localization of $\mathrm{Cat}_1$ to get $\mathrm{Cat}_2$ which is usually the category of effective motives with coefficient in $\mathrm{Sh}_A$;
\item[4] Take the inverse of the Tate object, then make sure it is added to the category, which is the final category of motives $\mathrm{Cat}_3$;
\item[5] Take the corresponding homotopy suspension of $\mathrm{Sh}_A$ for the Tate object, then take the corresponding \v{C}ech conerve of this resulting suspension $\mathrm{Cechconerve}(\mathrm{Suspen}_*\mathrm{Sh}_A)$, which gives rise to the Hopf algebra sheaf over $X$ after we take the loop space functor:
\begin{align}
\mathrm{Loop}^\infty\mathrm{{Cechconerve}}(\mathrm{Suspen}_*\mathrm{Sh}_A).
\end{align} 
Take $\mathrm{Spec}$ we have the motivic Galois sheaf.
\end{itemize}
For instance:
\begin{itemize}
\item[M1] Voevodsky's A1 motives in \cite{3V}, one can use arc topology actually for the schemes over some base scheme $X$;
\item[M2] Ayoub's B1 motives with $\mathbb{Q}$-coefficient \cite{31A}, one considers smooth rigid spaces with \'etale topology for instance;
\item[M3] Scholze's Berkovich motives in \cite{3S} with $\mathbb{Z}$-coefficient, one considers all the small arc stacks and arc topology;
\item[M4] Generalized motives with general coefficients as in \cite{3A}, one considers smooth rigid analytic spaces in \'etale topology, then takes localization for $\infty$-category of sheaves in such \'etale site in any coefficient in well-defined Weil cohomology theories.
\end{itemize}
Following all these and the philosophy from Grothendieck \cite{3G}, we consider de Rham prismatization motives in families, de Rham filtration prismatization motives in families, and de Rham syntomization prismatization motives in families.


\begin{situation}\mbox{\textbf{(Motivic cohomology theories in families I)}}\label{situation1}
We consider all the small arc-stacks/small $v$-stacks over $A$ fibered over $\mathbb{N}_\infty$. This means we fix some small arc-stack or small $v$-stack $X$ and consider the corresponding $\infty$-categories of the arc site or $v$-site over $X$ using all the small arc stacks or small $v$-stacks:
\begin{align}
\mathrm{Cate}_\mathrm{arc}/X,\mathrm{Cate}_v/X.
\end{align}
A $z$-adic motivic cohomology theory over any small arc-stack/small $v$-stack $X$ consists of the datum as in the following after \cite{3A}:
\begin{itemize}
\item[A1] An $\infty$-sheaf of anima $\mathrm{Sh}_A$, of $A$-algebra over the arc stack $X$ in the arc site of small arc stacks over $X$ or over $v$-stack $X$ in the $v$-site of small $v$-stacks over $X$, in familes over $\mathbb{N}_\infty$;
\item[A2] A version of K\"unneth theorem in the derived sense holds for $\mathrm{Sh}_A$, in familes over $\mathbb{N}_\infty$;
\subitem $\blacksquare$ Usually this will mean taking the derived tensor products in the derived $\infty$-categories is such theorem exists; 
\item[A3] A derived $\infty$-category of Weil sheaves $D\mathrm{Weil}(X)$ over again the arc site or $v$-site of $X$ satisfying the arc-descent or $v$-descent, in families over $\mathbb{N}_\infty$;
\item[A4] A motivic six-functor formalism $f_A,f_B,f_C,f_D,f_E,f_F$, in familes over $\mathbb{N}_\infty$;
\item[A5] A formalism of Hopf algebraic motivic fundamental groups as in \cite[4.7]{3A}, in families over $\mathbb{N}_\infty$.
\subitem $\blacksquare$ This is more general than just \textit{motivic Galois groups} of a point for $L$ or $A$ in families over $\mathbb{N}_\infty$. To be more precise for each $\mathrm{Sh}_A$, one takes the corresponding $\infty$-level \textbf{suspension} (with respect to the Tate element in \cite{3A}, \cite{3S}) of this ring to get the sheaf $\widetilde{\mathrm{Sh}}_A$, then takes the \v{C}ech conerve $\mathrm{CechConerve}(\widetilde{\mathrm{Sh}}_A)$ as in \cite[4.9]{3A}, then takes the \textbf{$\infty$-loop space} with respect to the Tate object as in \cite{3A}, \cite{3S}: $\mathrm{Loop}^\infty\mathrm{Cechconerve}(\widetilde{\mathrm{Sh}}_A)$, this is then the desired Hopf \textit{sheaf} of algebra over the arc site or $v$-site of $X$. One then takes the spectrum of this sheaf to get the motivic fundamental group \textit{sheaf} $\mathrm{Spec}\mathrm{Loop}^\infty\mathrm{Cechconerve}(\widetilde{\mathrm{Sh}}_A)$. One can also take the global section over $X$ to reach the algebra and group theoretic objects. This process is compatible with the definition of the Berkovich motives in families: i.e. one considers all the small arc stacks or $v$-stacks over $X$ in arc topology or $v$-topology:
\begin{align}
\mathrm{Cate}(\mathrm{Stack}_{\mathrm{*},X}),
\end{align}
where $*$ is arc or $v$, then takes the localization using the loop space functor at $\infty$-level from the Tate object (over pointed space for the projective space of dimension 1) to define the effective motives in families:
\begin{align}
\mathrm{Effe}\mathrm{Cate}(\mathrm{Stack}_{\mathrm{*},X}),
\end{align}
then one inverts the Tate object to finish the definition of the desired symmetrical monoidal $\infty$-category of arc or $v$-motives in families:
\begin{align}
\mathrm{Cate}(\mathrm{Stack}_{\mathrm{*},X})_{\mathrm{final},\times}.
\end{align}
\end{itemize}
(!) We allow $\mathrm{Sh}_A$ to be a derived $\mathrm{E}_1$-object by considering bigraded resolution as in \cite{3SI}. We allow the six-functor formalism to be completely abstract in the sense of derived $\infty$-category (i.e. without really touching the spaces) by requiring the projection formula, smooth base change, and proper base change and so on in pure $\infty$-categorical sense as in \cite{3CS1}. Moreover there is a \textit{derived} version of this situation for small $(\infty,1)$-arc stacks in families by replacing arc stacks in families by $(\infty,1)$-arc stacks in families (which are fiber categories in families over the arc site of simplicial Banach rings in families). 
\end{situation}

\begin{situation}\mbox{\textbf{(Motivic cohomology theories in families II)}}\label{situation2}
We consider all the analytic stacks as in \cite{3CS1} over $A$ fibered over $\mathbb{N}_\infty$. A $z$-adic motivic cohomology theory over any analytic stack $X$ consists of the datum as in the following after \cite{3A}:
\begin{itemize}
\item[A1] An $\infty$-sheaf of anima $\mathrm{Sh}_A$, of $A$-algebra over a fixed analytic stack $X$ over $A$ (again one consider all the analytic stacks over $X$ in analytic topology to form the analytic site) in analytic topology, in familes over $\mathbb{N}_\infty$;
\item[A2] A version of K\"unneth theorem in the derived sense holds for $\mathrm{Sh}_A$, in familes over $\mathbb{N}_\infty$;
\item[A3] A derived $\infty$-category of Weil sheaves $D\mathrm{Weil}(X)$ over the analytic site of $X$ satisfying the descent under the !-functor formalism as in \cite{3CS1}, in families over $\mathbb{N}_\infty$; 
\item[A4] A motivic six-functor formalism $f_A,f_B,f_C,f_D,f_E,f_F$, in familes over $\mathbb{N}_\infty$;
\item[A5] A formalism of Hopf algebraic motivic fundamental groups as in \cite[4.7]{3A}, in families over $\mathbb{N}_\infty$.
\subitem $\blacksquare$ This is more general than just \textit{motivic Galois groups} of a point for $L$ or $A$ in families over $\mathbb{N}_\infty$. To be more precise for each $\mathrm{Sh}_A$, one takes the corresponding $\infty$-level suspension (with respect to the Tate element in \cite{3A}, \cite{3S}) of this ring to get the sheaf $\widetilde{\mathrm{Sh}}_A$, then takes the \v{C}ech conerve $\mathrm{Cechconerve}(\widetilde{\mathrm{Sh}}_A)$ as in \cite[4.9]{3A}, then takes the $\infty$-loop space with respect to the Tate object as in \cite{3A}, \cite{3S}: $\mathrm{Loop}^\infty\mathrm{Cechconerve}(\widetilde{\mathrm{Sh}}_A)$, this is then the desired Hopf \textit{sheaf} of algebra over the arc site or $v$-site of $X$. One then takes the spectrum of this sheaf to get the motivic fundamental group \textit{sheaf} $\mathrm{Spec}\mathrm{Loop}^\infty\mathrm{Cechconerve}(\widetilde{\mathrm{Sh}}_A)$. One can also take the global section over $X$ to reach the algebra and group theoretic objects. This process is compatible with the definition of the analytic  motives in families: i.e. one considers all the analytic condensed stacks over $X$ in analytic topology, then takes the localization using the loop space functor at $\infty$-level from the Tate object (over pointed space for the projective space of dimension 1) to define the effective motives in families, then one inverts the Tate object to finish the definition.
\end{itemize}
$(!)$ We allow $\mathrm{Sh}_A$ to be a derived $\mathrm{E}_1$-object by considering bigraded resolution, as in \cite{3SI}. We allow the six-functor formalism to be completely abstract in the sense of derived $\infty$-category (i.e. without really touching the spaces) by requiring the projection formula, smooth base change, and proper base change and so on in pure $\infty$-categorical sense as in \cite{3CS1}.

\end{situation}

\begin{remark}
We consider \textit{schemes} (so then $z$-adic $A$-\textit{formal schemes} as well by taking the projective limit over schemes) over $A$ with fibration over $\mathbb{N}_\infty$ in these theories as well, as in \cite{3S} by taking the discrete norms. But we only regard schemes as small arc-stacks in arc topology. This meams we will regard derived formal stacks as derived arc stacks, and we will regard derived algebraic stacks as derived arc stacks as well.
\end{remark}

\begin{remark}
Our consideration is of course a more generalized version of the corresponding consideration in \cite{3A} by using the foundation from \cite{3S}. \cite{3A} considered the motives for rigid analytic varieties.
\end{remark}

\begin{remark}\mbox{(\textbf{Arithmetic $D$-modules in families from $\infty$})}
\cite{3A1} can be integrated into a motivic theory in the sense above by considering the corresponding algebraic stack setting. For this choose some identification from $\infty$ and some finite integer, and translate the theory of arithmetic $D$-module from $A_\infty/z_\infty$ to $A_i/z_i$ for some $i \in \mathbb{N}$, then take the corresponding inverse limit we can reach some analytic version of the arithmetic $D$-module theory in families over $\mathbb{N}$, where one can enlarge the consideration in \cite{3A} to the corresponding schemes over $A$ and then to small arc-stacks over $L$ for instance. One can see that the theory satisfies all the 5 conditions in the above considerations. We have no ideas about this construction on how it may be linked to other $p$-adic motivic cohomology theories, however over $A(\mathbb{N})$ the theory of $F$-isocrystals is not that far away from this. In such a way the existence of 6-functor formalism holds largely due to a tight correspondence between the $\infty\in \mathbb{N}$ and any finite number on the ring level, which leads to the corresponding 6-functor formalism of arithmetic $D$-modules in the category of rigid analytic spaces, again of course in families. However this process on the other hand provides the corresponding 6-functor formalism for arithmetic $D$-modules over rigid analytic spaces in some nontrivial way, in families.
\end{remark}


\begin{remark}
Here are the motivic cohomology theories in families relevant in the current consideration on motivic cohomology theories:
\begin{itemize}
\item[1] Robba sheaves, and solid quasicoherent sheaves over them, the consideration is for $v$-stacks or arc-stacks over $A$ in families over $\mathbb{N}_\infty$;
\item[2] Prismatization over $A$ in families $\mathbb{N}_\infty$;
\subitem[2I] Prismatization over $A$ in families $\mathbb{N}_\infty$;
\subitem[2II] Filtration prismatization over $A$ in families $\mathbb{N}_\infty$;
\subitem[2III] Syntomization prismatization over $A$ in families $\mathbb{N}_\infty$;
\item[3] de Rham prismatization over $A$ in families $\mathbb{N}_\infty$;
\subitem[3I] de Rham prismatization over $A$ in families $\mathbb{N}_\infty$;
\subitem[3II] de Rham filtration prismatization over $A$ in families $\mathbb{N}_\infty$;
\subitem[3III] de Rham syntomization prismatization over $A$ in families $\mathbb{N}_\infty$;
\item[4] de Rham-Hodge-Tate prismatization over $A$ in families $\mathbb{N}_\infty$;
\subitem[4I] de Rham-Hodge-Tate prismatization over $A$ in families $\mathbb{N}_\infty$;
\subitem[4II] de Rham-Hodge-Tate filtration prismatization over $A$ in families $\mathbb{N}_\infty$;
\subitem[4III] de Rham-Hodge-Tate syntomization prismatization over $A$ in families $\mathbb{N}_\infty$;
\item[5] Laurent prismatization over $A$ in families $\mathbb{N}_\infty$;
\subitem[5I] Laurent prismatization over $A$ in families $\mathbb{N}_\infty$;
\subitem[5II] Laurent filtration prismatization over $A$ in families $\mathbb{N}_\infty$;
\subitem[5III] Laurent syntomization prismatization over $A$ in families $\mathbb{N}_\infty$;
\item[6] Analytic prismatization over $A$ in families $\mathbb{N}_\infty$;
\subitem[6I] Analytic prismatization over $A$ in families $\mathbb{N}_\infty$;
\subitem[6II] Analytic filtration prismatization over $A$ in families $\mathbb{N}_\infty$;
\subitem[6III] Analytic syntomization prismatization over $A$ in families $\mathbb{N}_\infty$;
\item[7] Analytic de Rham prismatization over $A$ in families $\mathbb{N}_\infty$;
\subitem[7I] Analytic de Rham prismatization over $A$ in families $\mathbb{N}_\infty$;
\subitem[7II] Analytic de Rham filtration prismatization over $A$ in families $\mathbb{N}_\infty$;
\subitem[7III] Analytic de Rham syntomization prismatization over $A$ in families $\mathbb{N}_\infty$;
\item[8] Analytic de Rham-Hodge-Tate prismatization over $A$ in families $\mathbb{N}_\infty$;
\subitem[8I] Analytic de Rham-Hodge-Tate prismatization over $A$ in families $\mathbb{N}_\infty$;
\subitem[8II] Analytic de Rham-Hodge-Tate filtration prismatization over $A$ in families $\mathbb{N}_\infty$;
\subitem[8III] Analytic de Rham-Hodge-Tate syntomization prismatization over $A$ in families $\mathbb{N}_\infty$;
\item[9] Analytic Laurent prismatization over $A$ in families $\mathbb{N}_\infty$;
\subitem[9I] Analytic Laurent prismatization over $A$ in families $\mathbb{N}_\infty$;
\subitem[9II] Analytic Laurent filtration prismatization over $A$ in families $\mathbb{N}_\infty$;
\subitem[9III] Analytic Laurent syntomization prismatization over $A$ in families $\mathbb{N}_\infty$;
\item[10] $B_{+,\mathrm{dR},\mathbb{N}_\infty,*}$-cohomology theory in families over $\mathbb{N}_\infty$;
\subitem[$\blacksquare$] This can be derived from the de Rham prismatizations in families over $\mathbb{N}_\infty$ after \cite{3Ta}, \cite{3F}, \cite{3S3};
\item[11] $B_{+,\mathrm{dR},\mathrm{Nygaard},\mathbb{N}_\infty,*}$-cohomology theory in families over $\mathbb{N}_\infty$;
\subitem[$\blacksquare$] This can be derived from the de Rham filtration prismatizations in families over $\mathbb{N}_\infty$ after \cite{3Ta}, \cite{3F}, \cite{3S3};
\item[12] $B_{+,\mathrm{dR},\mathrm{syntomization},\mathbb{N}_\infty,*}$-cohomology theory in families over $\mathbb{N}_\infty$;
\subitem[$\blacksquare$] This can be derived from the de Rham syntomization  prismatizations in families over $\mathbb{N}_\infty$ after \cite{3Ta}, \cite{3F}, \cite{3S3};
\item[13] and more motivic cohomology theories.
\end{itemize}
\end{remark}












\newpage
\section{Stable $\infty$-Categories in $\mathbb{F}_1$-Analytic Geometry}

\subsection{Grothendieck Categoricalizations}

\indent \cite{BBBK}, \cite{CS1}, \cite{CS2}, \cite{CS3} enlarge the categories of the rings in order to study the analytic geometry over $\mathbb{F}_1$. For instance in \cite{BBBK} we have the following construction. Several constructions on symmetrical monoidal categories are constructed with certain default tensor product. For instance in \cite{BBBK}, over $\mathbb{F}_1$ it just considers all the Banach sets which forms the corresponding category $\underline{\mathrm{BanachSets}}_*$. This category is not the ideal one since it is not easy to construct some stable $\infty$-categories directly from it to do desired analytic geometry, i.e. to form well-defined ringed-spaces and ringed-stacks. The point of view is to take the corresponding inductive limits over the category to form the $\underline{\mathrm{IndBanachSets}}_*$. Another closely related construction is to look at the corresponding monomorphic morphisms when one forms the inductive limits: $\underline{\mathrm{Ind_{\mathrm{mono}}BanachSets}}_*$. The latter two are actually symmetrical monoidal tensor categories when can then form the corresponding stable $\infty$-categories from them, we use the notations as in the following to denote the corresponding stable $\infty$-categories:
\begin{align}
\underline{\mathrm{IndBanachSets}}^\sharp_*, \underline{\mathrm{Ind_{\mathrm{mono}}BanachSets}}^\sharp_*.
\end{align}

\begin{remark}
One can enlarge the categories by considering the corresponding seminormed and normed sets over $*$:
\begin{align}
&\underline{\mathrm{IndNSets}}^\sharp_*, \underline{\mathrm{Ind_{\mathrm{mono}}NSets}}^\sharp_*,\\
&\underline{\mathrm{IndSNSets}}^\sharp_*, \underline{\mathrm{Ind_{\mathrm{mono}}SNSets}}^\sharp_*.
\end{align}
\end{remark}


\subsection{Modules}

For the $A_\infty$-rings or $E_1$-rings or $E_\infty$-rings in the previous discussion the modules over them are actually quite complicated as in \cite{BBK}. On the other hand they are very significant in the quasicoherent sheaf theoretic consideration in \cite{BBK}. We replace the base by some Banach ring $*$ and we consider not just sets but instead the modules over $*$ carrying the topologizations as in the above. For instance in \cite{BBBK}, over $*$ it just considers all the Banach modules which forms the corresponding category $\underline{\mathrm{BanachModules}}_*$. This category is not the ideal one since it is not easy to construct some stable $\infty$-categories directly from it to do desired analytic geometry, i.e. to form well-defined ringed-spaces and ringed-stacks. The point of view is to take the corresponding inductive limits over the category to form the $\underline{\mathrm{IndBanachModules}}_*$. Another closely related construction is to look at the corresponding monomorphic morphisms (i.e., born\'e) when one forms the inductive limits: $\underline{\mathrm{Ind_{\mathrm{mono}}BanachModules}}_*$. The latter two are actually symmetrical monoidal tensor categories when can then form the corresponding stable $\infty$-categories from them, we use the notations as in the following to denote the corresponding stable $\infty$-categories:
\begin{align}
\underline{\mathrm{IndBanachModules}}^\sharp_*, \underline{\mathrm{Ind_{\mathrm{mono}}BanachModules}}^\sharp_*.
\end{align}

\begin{remark}
One can enlarge the categories by considering the corresponding seminormed and normed sets over $*$:
\begin{align}
&\underline{\mathrm{IndNModules}}^\sharp_*, \underline{\mathrm{Ind_{\mathrm{mono}}NModules}}^\sharp_*,\\
&\underline{\mathrm{IndSNModules}}^\sharp_*, \underline{\mathrm{Ind_{\mathrm{mono}}SNModules}}^\sharp_*.
\end{align}
\end{remark}



\newpage
\section{Symmetrical Monoidal $\infty$-Categories of $p$-adic Hodge Modules}

\subsection{Generalized Hodge modules}

For instance in \cite{BBKK} and \cite{CS3}, we have the well-defined structure sheaf $RX(.)$ over $X$ any $p$-adic analytic space. The structure sheaf provides the definition of quasicoherent sheaves. In \cite{T} we used the $p$-adic functional analytic method from \cite{K1} and \cite{KL} to have studied some spaces emerges after \cite{CKZ} and \cite{PZ}. Though spaces we looked at are those regular higher dimensional rigid analytic spaces, i.e. just multi-discs, the essential consideration and application in mind are quite new. Approximately we have the ring of analytic funcions over a multi-disc with parameter $[a_I,b_I]$ with $I$ some finite set. This ring is the higher dimensional version of the rings studied in \cite{KPX}. We use the notation $L_{[a_I,b_I]}$ to denote these rings. Recall the multi-rigid intervals take the form of:
\begin{align}
p^{-b_i{p/(p-1)}}<||x||_{\mathbb{Q}_p}<p^{-a_i{p/(p-1)}}, \forall i \in I.
\end{align}
Then taking inverse limit on $a_I$ we have the version of rings with just multi-radii $L_{b_I}$. These rings will give the full version of the ring we are considering if we take union on $b_I$ we then use the notation $L$ to denote this. So we have:
\begin{definition} \label{definition1}
We have defined as above the key rings in consideration:
\begin{align}
L_{[a_I,b_I]},L_{b_I}:=L_{(0,b_I]},L_{[0,b_I]},L:=\bigcup_{b_I>0}\bigcap_{a_I>0}L_{[a_I,b_I]}.
\end{align}
Here $b_I$ can be $+\infty$ towards all directions.
\end{definition}


We then enlarge the coeffcients to $I$ different finite extensions of $\mathbb{Q}_p$: $K_1$,...,$K_{|I|}$. Where we have the corresponding  
\begin{align}
L_{[a_I,b_I],K_I},L_{b_I,K_I}:=L_{(0,b_I],K_I},L_{[0,b_I],K_I},L_{K_I}:=\bigcup_{b_I>0}\bigcap_{a_I>0}L_{[a_I,b_I],K_I}.
\end{align}

One also has the version of these rings where we have the corresponding variable substitutions by using the uniformizer $\pi_{K_I}$ for different directions, as well as the group $\Gamma_{K_I}$ for different directions (by using the generator for each component in this product $\Gamma_{K_1}\times...\times\Gamma_{K_{|I|}}$):
\begin{align}
L_{[a_I,b_I],K_I}(\pi_{K_I}),L_{b_I,K_I}(\pi_{K_I}):=L_{(0,b_I],K_I}(\pi_{K_I}),L_{[0,b_I],K_I}(\pi_{K_I}),L_{K_I}(\pi_{K_I}):=\bigcup_{b_I>0}\bigcap_{a_I>0}L_{[a_I,b_I],K_I}(\pi_{K_I}),
\end{align}
\begin{align}
L_{[a_I,b_I],K_I}(\Gamma_{K_I}),L_{b_I,K_I}(\Gamma_{K_I}):=L_{(0,b_I],K_I}(\Gamma_{K_I}),L_{[0,b_I],K_I}(\Gamma_{K_I}),L_{K_I}(\Gamma_{K_I}):=\bigcup_{b_I>0}\bigcap_{a_I>0}L_{[a_I,b_I],K_I}(\Gamma_{K_I}).
\end{align}


\begin{definition} \label{definition2}
Recall from \cite{T} we have the multi Frobenius actions and the multi $\Gamma_{K_I}$ actions over these rings. The corresponding actions through different directions are mutually commutative and actions along different directions will be identity map. Therefore we have the corresponding Frobenius-Hodge modules over these rings (in further relativization by tensoring with some rigid affinoid $X/\mathbb{Q}_p$):
\begin{align}
L^X_{[a_I,b_I],K_I}(\pi_{K_I}),L^X_{b_I,K_I}(\pi_{K_I}):=L^X_{(0,b_I],K_I}(\pi_{K_I}),L^X_{[0,b_I],K_I}(\pi_{K_I}),L^X_{K_I}(\pi_{K_I}):=\bigcup_{b_I>0}\bigcap_{a_I>0}L^X_{[a_I,b_I],K_I}(\pi_{K_I}).
\end{align}
A Frobenius-Hodge module over $L^X_{K_I}(\pi_{K_I})$ is defined to be a Frobenius-Hodge module over some $L^X_{b_I,K_I}(\pi_{K_I})$ through the base change when we take the union over $b_I$, while the latter is defined in the following way. It is defined to be a finite projective module over $L^X_{b_I,K_I}(\pi_{K_I})$ carrying semilinear Frobenius action along each direction $i\in I$ such as:
\begin{align}
\varphi^*F \overset{\sim}{\rightarrow} F
\end{align}
holds true after base change to $L^X_{b_I/b,K_I}(\pi_{K_I})$. We have the notion of a Frobenius-Hodge module over $L^X_{[a_I,b_I],K_I}(\pi_{K_I})$, which is defined to be a finite projective module over  $L^X_{[a_I,b_I],K_I}(\pi_{K_I})$ such as:
\begin{align}
\varphi^*F \overset{\sim}{\rightarrow} F
\end{align}
holds true after base change to $L^X_{[a_I,b_I/p],K_I}(\pi_{K_I})$. Here we assume over each direction $i\in I$ we have that $[a_i,b_i]\cap[a_i/p,b_i/p]=[a_i,b_i/p]\neq \emptyset$. This also leads to the notion of Frobenius-Hodge bundles over all such multi-intervals with the requirement: over each direction $i\in I$ we have that $[a_i,b_i]\cap[a_i/p,b_i/p]=[a_i,b_i/p]\neq \emptyset$. It is defined to be a collection of Frobenius-Hodge modules over all $\{[a_I,b_I]|\forall i\in I, [a_i,b_i]\cap[a_i/p,b_i/p]=[a_i,b_i/p]\neq \emptyset\}$ in a compatible way such as over each $[a_I,b_I/p]$ the sections are assumed to be isomorphic. As in \cite{T} we have the $\Gamma_{K_I}$-actions over these rings which form the objects which are called $\Gamma$-Frobenius modules and bundles.
\end{definition}

\begin{definition}
Over $*$=
\begin{align}
L^X_{[a_I,b_I],K_I}(\pi_{K_I}),L^X_{b_I,K_I}(\pi_{K_I}):=L^X_{(0,b_I],K_I}(\pi_{K_I}),L^X_{[0,b_I],K_I}(\pi_{K_I}),L^X_{K_I}(\pi_{K_I}):=\bigcup_{b_I>0}\bigcap_{a_I>0}L^X_{[a_I,b_I],K_I}(\pi_{K_I}),
\end{align} 
we have the notion of $\Gamma$-Frobenius modules and bundles as above. Then we consider the derived $\infty$-categories 
\begin{align}
\underline{\mathrm{IndBanachModules}}^\sharp_*, \underline{\mathrm{Ind_{\mathrm{mono}}BanachModules}}^\sharp_*.
\end{align}
which are stable. And we have the condensed version:
\begin{align}
\blacksquare D_{*}, \blacksquare D_{*,\mathrm{bounded}},\blacksquare D_{*,\mathrm{perfect}}.
\end{align}
Therefore we have the corresponding Banach perfect complexes and condensed perfect complexes of the corresponding objects in our setting, namely we consider the perfect complexes of the $\Gamma$-Frobenius-Hodge modules in our setting, which again form certain stable $\infty$-categories with symmetrical monoidal structures. In the Banach setting we use the notations:
\begin{align}
&\underline{\mathrm{IndBanachModules}}^\sharp_{*,\mathrm{perfect},\Gamma,F}, \underline{\mathrm{Ind_{\mathrm{mono}}BanachModules}}^\sharp_{*,\mathrm{perfect},\Gamma,F},\\
&\underline{\mathrm{IndBanachModules}}^\sharp_{*,\mathrm{perfect},\mathrm{bounded},\Gamma,F}, \underline{\mathrm{Ind_{\mathrm{mono}}BanachModules}}^\sharp_{*,\mathrm{perfect},\mathrm{bounded},\Gamma,F},\\
&\underline{\mathrm{IndBanachModules}}^\sharp_{*,\mathrm{perfect},-,\Gamma,F}, \underline{\mathrm{Ind_{\mathrm{mono}}BanachModules}}^\sharp_{*,\mathrm{perfect},-,\Gamma,F},
\end{align}
to denote these complexes. And in the corresponding condensed setting we use:
\begin{align}
\blacksquare D_{*,\Gamma,F}, \blacksquare D_{*,\mathrm{bounded},\Gamma,F},\blacksquare D_{*,\mathrm{perfect},\mathrm{bounded},\Gamma,F},\blacksquare D_{*,\mathrm{perfect},\Gamma,F},\blacksquare D_{*,\mathrm{perfect},-,\Gamma,F}.
\end{align}
\end{definition}


\subsection{Derived $\infty$-Categories}

\begin{definition}
If $I$ is singleton, then we have the notion of $(F,\Gamma)$-complex of any $\Gamma$-Frobenius-Hodge module $F$: $C_{F,\Gamma}(F)$, where we have also the $C_{F,*}(F)$ and $C_{*,\Gamma}(F)$ complexes as well. Using them by induction we have the corresponding notion of $(F,\Gamma)$-complex of any $\Gamma$-Frobenius-Hodge module $F$: $C_{F,\Gamma}(F)$ when $I$ is not just a singleton. In our setting for each $i\in I$ we also have the corresponding $W_i=\varphi_i^{-1}$-operator. In such a way one can form the complex $C_{W}(F)$ directly.
\end{definition}


\begin{theorem}
$C_{F,\Gamma}(F)$ is in bounded $(\infty,1)$-derived category of complexes over $X$, restricting to perfect complexes:
\begin{align}
D_{\mathrm{perfect},\mathrm{bounded}}(\mathrm{Mod}_X).
\end{align}
$C_{F,\Gamma}(F)$ is in 
\begin{align}
\underline{\mathrm{IndBanachModules}}^\sharp_{X,\mathrm{perfect},\mathrm{bounded}}, \underline{\mathrm{Ind_{\mathrm{mono}}BanachModules}}^\sharp_{X, \mathrm{perfect},\mathrm{bounded}}.
\end{align} 
$C_{W}(F)$ is in 
\begin{align}
\underline{\mathrm{IndBanachModules}}^\sharp_{L^X_{\infty_I,K_I}(\Gamma_{K_I}),\mathrm{perfect},\mathrm{-}}, \underline{\mathrm{Ind_{\mathrm{mono}}BanachModules}}^\sharp_{L^X_{\infty_I,K_I}(\Gamma_{K_I}), \mathrm{perfect},\mathrm{-}},
\end{align}
where $X$ is just $\mathbb{Q}_p$.
\end{theorem}

\begin{proof}
The statements for $C_{F,\Gamma}(F)$ are \cite{T}. For $C_{W}(F)$, over $L^X_{\infty_I,K_I}(\Gamma_{K_I})$ the same argument in \cite{T} works here where one just takes the corresponding projective resolutions in our stable $\infty$-categories in the current setting.
\end{proof}



\begin{remark}
Here one can in fact work with the usual derived categories after Grothendieck categoricalization, which also produces stable $\infty$-categories. To be more precise we consider 
\begin{align}
\mathrm{Ind}\mathrm{Mod}_{L^X_{\infty_I,K_I}(\Gamma_{K_I})}
\end{align}
which is then Grothendieck. Then we have the following stable derived $\infty$-category of all the complexes formed from the objects in the above category:
\begin{align}
D_{\mathrm{bounded}}(\mathrm{Ind}\mathrm{Mod}_{L^X_{\infty_I,K_I}(\Gamma_{K_I})}), D_{\mathrm{bounded}}(\mathrm{Ind}\mathrm{Mod}_{L^X_{\infty_I,K_I}(\Gamma_{K_I})}).
\end{align}
Then one can further restrict to those complexes with cohomology groups in $\mathrm{Mod}_{L^X_{\infty_I,K_I}(\Gamma_{K_I})}$ to achive the desired $\infty$-categories.
\end{remark}

We have the following parallel result by using the foundation in \cite{CS1}, \cite{CS2}, \cite{CS3}:


\begin{theorem}
$C_{F,\Gamma}(F)$ is in bounded $(\infty,1)$-derived category of complexes over $X$, restricting to perfect complexes:
\begin{align}
D_{\mathrm{perfect},\mathrm{bounded}}(\mathrm{Mod}_X).
\end{align}
$C_{F,\Gamma}(F)$ is in 
\begin{align}
\blacksquare\underline{D}_{X,\mathrm{perfect},\mathrm{bounded}}.
\end{align} 
$C_{W}(F)$ is in 
\begin{align}
\blacksquare\underline{D}_{L^X_{\infty_I,K_I}(\Gamma_{K_I}),\mathrm{perfect},-},
\end{align}
where $X$ is just $\mathbb{Q}_p$.
\end{theorem}

\begin{proof}
The statements for $C_{F,\Gamma}(F)$ are \cite{T}. For $C_{W}(F)$, over $L^X_{\infty_I,K_I}(\Gamma_{K_I})$ again see \cite{T} the argument remains unchange as long as one works in the categories of condensed complexes in the current setting.
\end{proof}

\begin{corollary}
$C_{W}(.)$ induces a derived functor from the stable $\infty$-category 
\begin{align}
\underline{\mathrm{IndBanachModules}}^\sharp_{*,\mathrm{perfect},\mathrm{bounded},\Gamma,F}, \underline{\mathrm{Ind_{\mathrm{mono}}BanachModules}}^\sharp_{*,\mathrm{perfect},\mathrm{bounded},\Gamma,F},
\end{align}
and in the corresponding condensed setting:
\begin{align}
\blacksquare D_{*,\mathrm{perfect},\mathrm{bounded},\Gamma,F},
\end{align}
to the stable $\infty$-category 
\begin{align}
\blacksquare\underline{D}_{L^X_{\infty_I,K_I}(\Gamma_{K_I}),\mathrm{perfect},-},
\end{align}
where $X$ is just $\mathbb{Q}_p$. Here $*=L_{K_I}^X(\pi_{K_I})$. This also induces the morphism on the $K$-group spectra of $\mathbb{E}_\infty$-rings after applying \cite{BGT} to the corresponding stable $(\infty,1)$-categories, after \cite{G2}, \cite{A2}, \cite{BGT}.
\end{corollary}

\begin{proof}
Now we consider the hyper cohomological functor formed from $C_{W}(.)$ and the complexes $F^\ell$ in the categories, which produces the corresponding spectral sequence $E_k^{.,\ell}=E[C_{W}(.)|F^\ell]$ which realizes the derived functor as desired.
\end{proof}

\begin{conjecture}
$C_{W}(.)$ induces a derived functor from the stable $\infty$-category 
\begin{align}
\underline{\mathrm{IndBanachModules}}^\sharp_{*,\mathrm{perfect},\mathrm{bounded},\Gamma,F}, \underline{\mathrm{Ind_{\mathrm{mono}}BanachModules}}^\sharp_{*,\mathrm{perfect},\mathrm{bounded},\Gamma,F},
\end{align}
and in the corresponding condensed setting:
\begin{align}
\blacksquare D_{*,\mathrm{perfect},\mathrm{bounded},\Gamma,F},
\end{align}
to the stable $\infty$-category 
\begin{align}
\blacksquare\underline{D}_{L^X_{\infty_I,K_I}(\Gamma_{K_I}),\mathrm{perfect},\mathrm{bounded}},
\end{align}
where $X$ is just $\mathbb{Q}_p$. Here $*=L_{K_I}^X(\pi_{K_I})$. This also induces the morphism on the $K$-group spectra of $\mathbb{E}_\infty$-rings after applying \cite{BGT} to the corresponding stable $(\infty,1)$-categories, after \cite{G2}, \cite{A2}, \cite{BGT}.
\end{conjecture}



\newpage
\section{Symmetrical Monoidal $\infty$-Categories in $p$-adic Functional Analysis}

\subsection{Representation of $p$-adic Lie groups}


\indent The algebra $L^X_{\infty_I,K_I}(\Gamma_{K_I})$ is some abelian version of the more general algebras as in \cite{ST} and \cite{Z1}. Their integral version will be the algebra taking the following form:
\begin{align}
X^+[[G]]
\end{align}
where $G$ is some Lie group over $\mathbb{Z}_p$ and $X^+$ is some local integral model of $X$, i.e. commutative $p$-adic $\mathbb{Z}_p$-algebras which can be commutative $\mathbb{F}_p$-algebra. We start from $X^+$ to be commutative $\mathbb{F}_p$-algebra, then this includes those $\mathbb{Z}_p/p^n$-algebras. Then by taking the inverse limits along $n$ we have the $p$-adic $\mathbb{Z}_p$-situation and then by taking $p^{-1}$ we can discuss the $\mathbb{Q}_p$-algebra coefficients. It, the ring $X^+[[G]]$, obviously can be highly noncommutative. In the field coefficient situation the category:
\begin{align}
R_{\mathrm{lisse},X^+,G}
\end{align}
is studied in \cite{SS1}, \cite{SS2} where monoidal structure has been established. More general setting is also established in \cite{HM}, where the category is promoted to certain complete category in the condensed mathematics. This is defined to be smooth representation of $G$ over the coefficient in $X^+$-modules. What is related is the corresponding work in \cite{So1} where Schneider's $E_1$-algebra\footnote{Also called $A_\infty$-algebras.} style Hecke algebras defined in \cite{Sc1} are studied in some well-established context. Again \cite{HM} generalized the definition of dg Hecke algebras to the relative setting with general coefficients. One also has the following algebra:
\begin{align}
DE^{X^+[[G]]}_{X^+,X^+}
\end{align}
which is the derived extension of $X^+$ with itself as the representation in some trivial manner. This algebra is actually $E_1$-algebra as in \cite{So1} when $X^+$ is a finite field. Since now the $E_1$-algebraic consideration is over $X^+$ we expect this to be more complicated. What is happening is that this algebra taking this form might be complicatedly hard to be related to the de Hecke algebra (defined in the obvious generalized way after \cite{Sc1}, in \cite{HM}) when the base is not a field, instead the base is a commutative ring as in \cite{So1}. Though similar, the relationship might be possibly taking a more generalized form as we discussed in the following.

\subsection{Derived Categories over $X^+$}

\noindent After \cite{SS1}, \cite{SS2}, \cite{HM} we now consider the derived $\infty$-category of:
\begin{align}
R_{\mathrm{lisse},X^+,G}.
\end{align}
As in \cite{SS1}, \cite{SS2}, \cite{HM} we have the generalized monoidal structure by using the tensor product over $X^+$ (note here $G$ can be noncommutative but we assume that $X^+$ to be commutative). Therefore we then have the derived $\infty$-category associated to this category which we denote it as:
\begin{align}
DR_{\mathrm{lisse},X^+,G}.
\end{align}



\begin{theorem}\mbox{\textbf{(Schneider-Sorensen \cite{SS1} \cite{SS2}, Heyer-Mann \cite{HM})}}
The $\infty$-category 
\begin{align}
DR_{\mathrm{lisse},X^+,G}.
\end{align}
is symmetrical monoidal with respect to the tensor products $\times_{G}$ defined in the same fashion as in \cite{SS1} and \cite{SS2} over tensor products over $X^+$ of two complexes. 
\end{theorem}


\indent When we have that $X^+$ is defined over $\mathbb{Z}_p$ such that:
\begin{align}
X^+ = \varprojlim_{n} \overline{X}^+_{\mathbb{Z}_p/p^n},
\end{align}
we can then put:
\begin{align}
\varprojlim_n DR_{\mathrm{lisse},\overline{X}^+_{\mathbb{Z}_p/p^n},G}.
\end{align}


\subsection{Derived Categories over $E_1$-ring $DE^{X^+[[G]]}_{X^+,X^+}$ over $X^+$}

\noindent Here we consider the derived $\infty$-category of the ring:
\begin{align}
DE^{X^+[[G]]}_{X^+,X^+}.
\end{align}
As in \cite{So1} this is actually $E_1$-ring over $X^+$, which can also be regarded as come derived $E_1$-algebra in the sense of \cite{Sa}. Here we follow \cite{So1} to assume the compactness of the group $G$ with further assumptions as in \cite{So1}. $X^+$ is again a $\mathbb{F}_p$-ring which is commutative. This means that we are assuming that the dg Hecke algebras in \cite{Sc1} and more generally in \cite{HM} are just identical to the ones from $X^+$ directly since we are just taking the induction from $G$ to $G$ itself. When we have that $X^+$ is defined over $\mathbb{Z}_p$ such that:
\begin{align}
X^+ = \varprojlim_{n} \overline{X}^+_{\mathbb{Z}_p/p^n},
\end{align}
we can then put:
\begin{align}
\varprojlim_n DE^{\overline{X}^+_{\mathbb{Z}_p/p^n}[[G]]}_{\overline{X}^+_{\mathbb{Z}_p/p^n},\overline{X}^+_{\mathbb{Z}_p/p^n}}.
\end{align}
We are going to consider $X^+$ to be now taking such limit form which is $p$-adic $\mathbb{Z}_p$-algebra. We then use the notation $H$ to denote the corresponding dg Hecke algera which is defined in \cite{Sc1} and \cite{HM}. First we take the resolution of $X^+$ as the trivial representation, then we just take the derived homomorphism algebra from $X^+$ now with itself. So far the discussion is based on the situation where the generalization is directly through the corresponding generalization in \cite{So1}, i.e. the ring:
\begin{align}
DE^{X^+[[G]]}_{X^+,X^+}.
\end{align}
is just a relative version of the ring in \cite{So1}. \cite{So1} then makes the contact with the homology $E_1$-ring for the dg Hecke algebra from \cite{Sc1} in some transparent way. Then the homology $E_1$-ring can be directly related to the dg Hecke algebra in \cite{Sc1} and \cite{HM}. However in our general setting this is not possible since at least \cite{Sa} points out the issue where we need to take the derived homology instead to reach such similar relationship. Therefore we now change the point of view from the ring:
\begin{align}
DE^{X^+[[G]]}_{X^+,X^+}.
\end{align}
to the corresponding model in \cite{Sa}:
\begin{align}
\underset{{\mathrm{totalized},X^+,X^+}}{\mathrm{homomorphism}}
\end{align}
which is for instance the derived homology in \cite{Sa}. Here the point of view is that in order to reach certain homology in some derived sense which preserves the corresponding derived $E_1$-categories one has to work with the so-called derived $E_1$-algebras in \cite{Sa}. In such a way \cite{So1} can be generalized to the relative setting. First we recall that $H$ is defined by using projective resolution of $X^+$ (in our case this is the same as the induction from $G$ to $G$):
\begin{align}
\mathrm{proj}_\mathrm{resolution}(X^+)
\end{align}
which is then defined as the derived homomorphism of this resolution with itself. From \cite{Sc1}, \cite{HM} we have functors linking the two $\infty$-categories:
\begin{align}
DR_{\mathrm{lisse},X^+,G},
\end{align}
and the derived $\infty$-category of all the $H$-$E_1$-modules:
\begin{align}
D_H.
\end{align}
We use the following notations to denote the functors:
\begin{align}
F_{DR_{\mathrm{lisse},X^+,G}}, F_{D_H}.
\end{align}



\begin{conjecture}\mbox{\textbf{(After Sorensen, \cite[Theorem 1.1]{So1})}}
Promoting $H$ to a derived $E_1$-ring $\mathbb{H}$, we have the derived $\infty$-category of the minimal derived $E_1$-ring
\begin{align}
\underset{{\mathrm{totalized},\mathbb{H},\mathbb{H}}}{\mathrm{homomorphism}}
\end{align}
admits a functor from or a functor into the derived $\infty$-category
\begin{align}
DR_{\mathrm{lisse},X^+,G},
\end{align}
through the functors:
\begin{align}
F_{DR_{\mathrm{lisse},X^+,G}}, F_{D_H}.
\end{align}
We conjecture all functors here are equivalences of symmetrical monoidal $\infty$-categories.
\end{conjecture}

\begin{remark}
We combine \cite[Theorem 1.2]{Sa} with the direct generalization of \cite{Sc1} from \cite{HM}. See construction of \cite[Theorem 9]{Sc1}. Then we do have the functors involved. However we don't know whether all the statements of this conjecture are correct or not.
\end{remark}










\begin{remark}
The goal here eventually will be consider certain Banach $\mathbb{Z}_p$-algebra and consider the corresponding Banach representations. One has to upgrade all the corresponding rings here to certain condensed setting such as in \cite{HM}, i.e. first the representation spaces have to be carrying the Banach topology, then the dg Hecke algebra will then be certain Banach dg algebra, then the corresponding derived $E_1$-ring in \cite{Sa} will also be taken to be the analytification version. For instance one works over the solid Banach $\mathbb{Z}_p$-modules. These consideration will be essentially certain $p$-adic local Langlands correspondence consideration after \cite{L1}, \cite{C}. 
\end{remark}



\begin{remark}
\indent Let us make some discussion on the ring
\begin{align}
\underset{{\mathrm{totalized},\mathrm{Ban},\mathbb{H},\mathbb{H}}}{\mathrm{homomorphism}}.
\end{align}
Recall that this ring is the totalization of the homomorphism group for the bigraded suspension of $\mathbb{H}$. In the Banach setting we need to take the homomorphisms in the solid Banach $\mathbb{Z}_p$-modules, which indicates that we use the notation $\mathrm{Ban}$.
\end{remark}

\indent Then as in \cite[Theorem 1.1]{So1} one can takes one step further to write the explicit formula for the derived homology for:
\begin{align}
\underset{{\mathrm{totalized},\mathbb{H},\mathbb{H}}}{\mathrm{homomorphism}}.
\end{align}
The resulting derived $E_1$-ring should then be following the definition from \cite{Sa}:
\begin{align}
\underset{{\mathrm{totalized},H,H}}{\mathrm{extension}}.
\end{align}



\subsection{Deformation of Modules over Derived $E_1$-Rings}


\indent Then one can use this to study the corresponding deformation functors in both Hecke algebraic setting and the corresponding smooth representation setting. Then this means the deformation happens in the following $\infty$-category of $\mathbb{Z}_p$ $p$-adic formal algebras with residue $\mathbb{F}_p$ by maximal ideals for $\pi_0$:
\begin{align}
\mathrm{LargeCoeff}_{\mathbb{Z}_p,\mathbb{F}_p,\mathrm{local}}
\end{align}
They have generic fibres as rigid analytic spaces, but we start from the corresponding integral picture. Recall the standard the deformation functors in the following sense. Deformation functor for the group $G$ is a functor fibered over:
\begin{align}
\mathrm{LargeCoeff}_{\mathbb{Z}_p,\mathbb{F}_p,\mathrm{local}}
\end{align}
by taking any ring $\blacksquare$ to a corresponding $\infty$-groupoid of all the representations over $G$ over $\blacksquare$. Deformation functor for the $E_1$-ring $H$ is a functor fibered over:
\begin{align}
\mathrm{LargeCoeff}_{\mathbb{Z}_p,\mathbb{F}_p,\mathrm{local}}
\end{align}
by taking any ring $\blacksquare$ to a corresponding $\infty$-groupoid of all the $E_1$-modules over $H$ with coefficient in $\blacksquare$. Deformation functor for the derived $E_1$-ring $\mathbb{H}$ is a functor fibered over:
\begin{align}
\mathrm{LargeCoeff}_{\mathbb{Z}_p,\mathbb{F}_p,\mathrm{local}}
\end{align}
by taking any ring $\blacksquare$ to a corresponding $\infty$-groupoid of all the $E_1$-modules over $\mathbb{H}$ with coefficient in $\blacksquare$. Deformation functor for the derived $E_1$-ring $\underset{{\mathrm{totalized},\mathbb{H},\mathbb{H}}}{\mathrm{homomorphism}}$ is a functor fibered over:
\begin{align}
\mathrm{LargeCoeff}_{\mathbb{Z}_p,\mathbb{F}_p,\mathrm{local}}
\end{align}
by taking any ring $\blacksquare$ to a corresponding $\infty$-groupoid of all the $E_1$-modules over 
\begin{align}
\underset{{\mathrm{totalized},\mathbb{H},\mathbb{H}}}{\mathrm{homomorphism}}
\end{align}
with coefficient in $\blacksquare$. We use the following notations to denote these $\infty$-deformation functors:
\begin{align}
\mathrm{Deform}_{G}, \mathrm{Deform}_H, \mathrm{Deform}_\mathbb{H}, \mathrm{Deform}_{\underset{{\mathrm{totalized},\mathbb{H},\mathbb{H}}}{\mathrm{homomorphism}}}.
\end{align}
Deformation functor for the group $G$ with respect to some representation $r/\mathbb{F}_p$ is a functor fibered over:
\begin{align}
\mathrm{LargeCoeff}_{\mathbb{Z}_p,\mathbb{F}_p,\mathrm{local}}
\end{align}
by taking any ring $\blacksquare$ to a corresponding $\infty$-groupoid of all the representations over $G$ over $\blacksquare$ such that the representations have quotients isomorphic to $r$ over $\blacksquare/m$. Deformation functor for the $E_1$-ring $H$ with respect to some module $r/H_{\mathrm{F}_p}$ is a functor fibered over:
\begin{align}
\mathrm{LargeCoeff}_{\mathbb{Z}_p,\mathbb{F}_p,\mathrm{local}}
\end{align}
by taking any ring $\blacksquare$ to a corresponding $\infty$-groupoid of all the $E_1$-modules over $H$ with coefficient in $\blacksquare$ such that the modules are isomorphic to $r$ after we take the quotient with respect to maximal ideals. Deformation functor for the derived $E_1$-ring $\mathbb{H}$ is similar lifting certain modules over $\mathbb{H}_{\mathbb{F}_p}$. We use the following notations to denote these $\infty$-deformation functors:
\begin{align}
\textit{Deform}_{G,r}, \textit{Deform}_{H,r}, \textit{Deform}_{\mathbb{H},r}, \textit{Deform}_{\underset{{\mathrm{totalized},\mathbb{H},\mathbb{H}}}{\mathrm{homomorphism}},r}.
\end{align}


\begin{conjecture}\mbox{\textbf{(After Sorensen, \cite[Theorem 1.1]{So1})}}
We have well-defined morphisms of deformation functors:
\begin{align}
\mathrm{Deform}_{G}\overset{\sim}{\rightarrow}\mathrm{Deform}_H\overset{\sim}{\rightarrow} \mathrm{Deform}_\mathbb{H}\overset{\sim}{\rightarrow} \mathrm{Deform}_{\underset{{\mathrm{totalized},\mathbb{H},\mathbb{H}}}{\mathrm{homomorphism}}}.
\end{align}
All functors are equivalent in this conjecture.
\end{conjecture}

\begin{conjecture}\mbox{\textbf{(After Sorensen, \cite[Theorem 1.1]{So1})}}
We have well-defined morphisms of deformation functors:
\begin{align}
\textit{Deform}_{G,r}\overset{\sim}{\rightarrow} \textit{Deform}_{H,r}\overset{\sim}{\rightarrow} \textit{Deform}_{\mathbb{H},r}\overset{\sim}{\rightarrow} \textit{Deform}_{\underset{{\mathrm{totalized},\mathbb{H},\mathbb{H}}}{\mathrm{homomorphism}},r}.
\end{align}
All functors are equivalent in this conjecture.
\end{conjecture}

\begin{conjecture}\mbox{\textbf{(After Sorensen, \cite[Theorem 1.1]{So1})}}
We have well-defined morphisms of derived deformation functors:
\begin{align}
\mathrm{Deform}_{G}\overset{\sim}{\rightarrow}\mathrm{Deform}_H\overset{\sim}{\rightarrow} \mathrm{Deform}_\mathbb{H}\overset{\sim}{\rightarrow} \mathrm{Deform}_{\underset{{\mathrm{totalized},\mathbb{H},\mathbb{H}}}{\mathrm{homomorphism}}},
\end{align}
over the animation:
\begin{align}
\underline{\mathrm{LargeCoeff}}_{\mathbb{Z}_p,\mathbb{F}_p,\mathrm{local}}.
\end{align}
All functors are equivalent in this conjecture.
\end{conjecture}

\begin{conjecture}\mbox{\textbf{(After Sorensen, \cite[Theorem 1.1]{So1})}}
We have well-defined morphisms of derived deformation functors:
\begin{align}
\textit{Deform}_{G,r}\overset{\sim}{\rightarrow} \textit{Deform}_{H,r}\overset{\sim}{\rightarrow} \textit{Deform}_{\mathbb{H},r}\overset{\sim}{\rightarrow} \textit{Deform}_{\underset{{\mathrm{totalized},\mathbb{H},\mathbb{H}}}{\mathrm{homomorphism}},r},
\end{align}
over the animation:
\begin{align}
\underline{\mathrm{LargeCoeff}}_{\mathbb{Z}_p,\mathbb{F}_p,\mathrm{local}}.
\end{align}
All functors are equivalent in this conjecture.
\end{conjecture}

\newpage
\section{Witt-Prisms as Generalized Prisms}\label{section5}


\subsection{Witt-Prisms for functional field}
\indent We start from any $v$-stack in the work of Scholze in \cite{1S1}. We use a notation $K$ to denote such stack. We assume that $K$ is over some local field, namely we take the $\mathrm{Spd}$ of some local field $L$. This local field needs to be specified in different characteristics. In positive characteristic situation, we assume this takes the form being finite over some $\mathbb{F}_p((u))$ for some chosen uniformizer $u$. In the $p$-adic setting, we assume it to be finite over $\mathbb{Q}_p$. 
\begin{assumption}
In this \cref{section5}, $L$ is now assumed to be in the function field situation, namely over $\mathbb{F}_p$.
\end{assumption}
Now for such $K$ we have two different versions of the prismatizations, after \cite{1To1}, \cite{1To2}, \cite{1To3}, \cite{1To4}, \cite{1S4}, \cite{1ALBRCS}, \cite{1BS}, \cite{1D}, \cite{1BL}. We in the algebraic setting unveil the definition from \cite{1BS}, \cite{1D}, \cite{1BL} in the following for the convenience of the readers. We consider the following function fields:
\begin{align}
\mathbb{F}_p[[u]][u^{-1}].
\end{align}
Then we consider the moduli stacks of the Witt-line bundles in the following sense. The underlying sites will be chosen to be the categories of all 
\begin{align}
\mathbb{F}_p[[u]]
\end{align}
algebras where the element $u$ presents nilpotency. Then by using the valuations from the corresponding Witt vectors, we can consider the corresponding infinite product:
\begin{align}
\mathrm{Wi(\square)}=\prod_i \square_i = \square \times \square \times \square \times \square \times \square\times \square \times... 
\end{align} 
The Witt vector infinite product presentations present the Witt vectors as the corresponding infinite products. Then over any base ring $\square$ we consider the moduli of all the line bundles mapping to these infinite products, such as all such line bundles are generated principally locally through elements satisfying the following requirements: the first coordinates are nilpotent and the second coordinates are unital after we map the these elements to the corresponding infinite products. Furthermore when we are constructing the motives for some $u$-adic formal scheme we then require the corresponding spectrum of the quotients of $\mathrm{Wi}(\square)$ through the line bundles to be mapped to the formal scheme in the construction. For any such generalized prism pair (a line bundle and a Witt vector deformation under this line bundle) we can then put the structure sheaf
 \begin{align}
 \mathcal{T}
 \end{align}
 of the prestacks as just the ring $\square$ before the deformation. Then over $\mathcal{T}$ we have further the corresponding de Rham structure sheaf 
 \begin{align}
 \mathcal{T}[1/u]_{\mathrm{Li}}
  \end{align}
 for any line bundle $\mathrm{Li}$. The corresponding such pair can be called as the corresponding \textit{Witt-prisms}, in a very generlized fashion.
 
 
 
 
 
 We use the notation:
\begin{align}
\mathrm{CondPrismatization}_{K,v}
\end{align}
to denote the condensation of the prismatization of $K$ over the $v$-topology. To be more precise for each perfectoid $K_i$ living over $K$ we take the corresponding algebraic prismatization:
\begin{align}
\mathrm{AlgPrismatization}_{K_i}
\end{align}
Then we take the analytification from \cite{1CS}:
\begin{align}
\mathrm{AlgPrismatization}_{K_i,\square}
\end{align}
Then what we do is to change $K_i$ in the Grothendieck topology to rich the whole prismatization overall over $K$ as sheaves of categories:
\begin{align}
\mathrm{CondPrismatization}_{K,v}.
\end{align}
In \cite{1To1} we also considered the corresponding de Rham stacks and the cristalline ones, where we allow the untils to be parametrized through the prismatization:
\begin{align}
&\mathrm{ConddeRhamPrismatization}_{K,v},\\
&\mathrm{CondCristallinePrismatization}_{K,v}.
\end{align}


\begin{definition}
We have the corresponding generalized version after \cite{1BS1}, \cite{1F2}, \cite{1BL1}, \cite{1BL2}:
\begin{align}
&\mathrm{ConddeRhamPrismatization}_{K,v,2}\\
&\mathrm{CondCristallinePrismatization}_{K,v,2}\\
&\mathrm{CondPrismatization}_{K,v,2}.
\end{align}
Here $b$ is fixed to be the element in both situations as in \cite{1BL1}, \cite{1BL2}, \cite{1BS1}, in the $u$-adic setting we have the analog element as well where $u$-adic cyclotomic character is also defined on this element. Then we take the corresponding analytification:
 \begin{align}
&\mathrm{ConddeRhamPrismatization}_{K,v,2,\square}\\
&\mathrm{CondCristallinePrismatization}_{K,v,2,\square}\\
&\mathrm{CondPrismatization}_{K,v,2,\square}.
\end{align}
\end{definition}

Then we can consider the corresponding solid quasicoherent sheaves over the stacks. 

\begin{remark}
Here the solid analytification can be explicified in the following sense. Over some $K_i$ locally we have the stacks can be written as \textit{projective limits} of the corresponding formal spectrum of the Witt vector rings, de Rham period sheaves and the cristalline period sheaves. Then the corresponding condensation and solidified analytification will automatically transform the formal topology to the topology encoding the Banach norms from the underlying perfectoid rings after \cite{1KL} and \cite{1KL1}.
\end{remark}

\begin{definition}
Now over any $K_i$ as above, we have the corresponding local version of the $\infty$-category of solid quasicoherent sheaves over the stacks above:
 \begin{align}
&\mathrm{SolidModules}_{\mathrm{ConddeRhamPrismatization}_{K_i,v,2,\square}}\\
&\mathrm{SolidModules}_{\mathrm{CondCristallinePrismatization}_{K_i,v,2,\square}}\\
&\mathrm{SolidModules}_{\mathrm{CondPrismatization}_{K_i,v,2,\square}}.
\end{align}
Here before taking the corresponding condensation we have the corresponding only formal topology versions:
 \begin{align}
&\mathrm{Modules}_{\mathrm{ConddeRhamPrismatization}_{K_i,v,2}}\\
&\mathrm{Modules}_{\mathrm{CondCristallinePrismatization}_{K_i,v,2}}\\
&\mathrm{Modules}_{\mathrm{CondPrismatization}_{K_i,v,2}}.
\end{align}
\end{definition}

We remind the readers the internal structure of these definitions as in the following. First recall that we can have the chance to write the above as $\varprojlim$ of certain derived $\infty$-categories of prisms. In the $u$-adic circumstance we have the parallel definition of \textit{generalized prisms}, which are defined to be any form of pairs as in the following in the definition of Cartier divisor for the $u$-prismatization:
\begin{align}
I,WVL(A)
\end{align}
where $A$ is nilpotent for the element $u$, and we require that the Witt vector ring $WVL(A)$ is complete in the derived sense with respect to the element $u$ and $I$. Then we have:
\begin{align}
&\mathrm{Modules}_{\mathrm{ConddeRhamPrismatization}_{K_i,v,2}}=\varprojlim_{{I,WVL(A)}} \mathrm{Modules}_{\varprojlim_{I}WVL(A)[1/x][b^{1/2}]}\\
&\mathrm{Modules}_{\mathrm{CondCristallinePrismatization}_{K_i,v,2}}=\varprojlim_{{I,WVL(A)}} \mathrm{Modules}_{\ WVL(A)[1/x][b^{1/2}]}\\
&\mathrm{Modules}_{\mathrm{CondPrismatization}_{K_i,v,2}}=\varprojlim_{{I,WVL(A)}} \mathrm{Modules}_{WVL(A)[b^{1/2}]}.
\end{align}


\begin{theorem}
In the $u$-adic setting we have:
\begin{align}
&\mathrm{Modules}_{\mathrm{ConddeRhamPrismatization}_{K_i,v,2}}=\varprojlim_{{I,WVL(A)}} \mathrm{Modules}_{\varprojlim_{I}WVL(A)[1/x][b^{1/2}]}\\
&\mathrm{Modules}_{\mathrm{CondCristallinePrismatization}_{K_i,v,2}}=\varprojlim_{{I,WVL(A)}} \mathrm{Modules}_{\ WVL(A)[1/x][b^{1/2}]}\\
&\mathrm{Modules}_{\mathrm{CondPrismatization}_{K_i,v,2}}=\varprojlim_{{I,WVL(A)}} \mathrm{Modules}_{WVL(A)[b^{1/2}]}
\end{align}
can be further written as:
\begin{align}
&\mathrm{Modules}_{\mathrm{ConddeRhamPrismatization}_{K_i,v,2}}=\varprojlim_{{I\rightarrow WVL(K_i)}} \mathrm{Modules}_{\varprojlim_{I}WVL(K_i)[1/x][b^{1/2}]}\\
&\mathrm{Modules}_{\mathrm{CondCristallinePrismatization}_{K_i,v,2}}=\mathrm{Modules}_{ WVL(K_i)[1/x][b^{1/2}]}\\
&\mathrm{Modules}_{\mathrm{CondPrismatization}_{K_i,v,2}}=\mathrm{Modules}_{WVL(K_i)[b^{1/2}]}.
\end{align}
\end{theorem}

\begin{proof}
In the $u$-adic setting and in the perfectoid setting the corresponding Witt vectors ring can be written as the algebraic tensor product of the $p$-adic version with the ring $\mathcal{O}^L$. Then the last equality can be derived from the $p$-adic situation. Then the first two equations can be then derived.
\end{proof}

\begin{remark}
In the $p$-adic setting the corresponding construction are based on prisms not the Witt-prisms in \cite{1BL}, \cite{1D}. We then have:
\begin{align}
&\mathrm{Modules}_{\mathrm{ConddeRhamPrismatization}_{K_i,v,2}}=\varprojlim_{{I\rightarrow B}} \mathrm{Modules}_{\varprojlim_{I}B[1/x][b^{1/2}]}\\
&\mathrm{Modules}_{\mathrm{CondCristallinePrismatization}_{K_i,v,2}}=\mathrm{Modules}_{ WVL(K_i)[1/x][b^{1/2}]}\\
&\mathrm{Modules}_{\mathrm{CondPrismatization}_{K_i,v,2}}=\mathrm{Modules}_{WVL(K_i)[b^{1/2}]}.
\end{align}
\end{remark}

\indent In the $u$-adic setting the \textit{generalized prisms} are basically being regarded as more general since in the $p$-adic setting pairs like $I,WVL(A)$ are actually generalized the corresponding notion of the prisms, while the latter can be used to construct such pairs. Therefore in the $u$-adic setting one can consider such definitions after \cite{1BS}, \cite{1BL}, \cite{1D}. 

\begin{theorem}
In the $u$-adic setting we have:
\begin{align}
&\mathrm{Modules}_{\mathrm{ConddeRhamPrismatization}_{K_i,v,2}}=\varprojlim_{{I,WVL(A)}} \mathrm{Modules}_{\varprojlim_{I}WVL(A)[1/x][b^{1/2}]}\\
&\mathrm{Modules}_{\mathrm{CondCristallinePrismatization}_{K_i,v,2}}=\varprojlim_{{I,WVL(A)}} \mathrm{Modules}_{\ WVL(A)[1/x][b^{1/2}]}\\
&\mathrm{Modules}_{\mathrm{CondPrismatization}_{K_i,v,2}}=\varprojlim_{{I,WVL(A)}} \mathrm{Modules}_{WVL(A)[b^{1/2}]}
\end{align}
can be further written as:
\begin{align}
&\mathrm{Modules}_{\mathrm{ConddeRhamPrismatization}_{K_i,v,2}}=\varprojlim_{{I\rightarrow WVL(K_i)}} \mathrm{Modules}_{\varprojlim_{I}WVL(K_i)[1/x][b^{1/2}]}\\
&\mathrm{Modules}_{\mathrm{CondCristallinePrismatization}_{K_i,v,2}}=\mathrm{Modules}_{ WVL(K_i)[1/x][b^{1/2}]}\\
&\mathrm{Modules}_{\mathrm{CondPrismatization}_{K_i,v,2}}=\mathrm{Modules}_{WVL(K_i)[b^{1/2}]}.
\end{align}
Then obviously we can recover certain perfectoid picture by requiring the projection to component where $I$ is the one generated by $b$.
\end{theorem}


\newpage
\section{Robba Stacks}

Here we consider another stackification after \cite{1KL}, \cite{1KL1}, \cite{1S1}, \cite{1S2}, \cite{1S3}, \cite{1F1}, \cite{1F2}, \cite{1T1}. We consider foundation from \cite{1CSA}, \cite{1CSB}, \cite{1CS}. Also we consider generalization following  \cite{1BS1}, \cite{1BL1}, \cite{1BL2}. This also is towards some motivic generalization in the sense of \cite{1G}. 

\subsection{Perfectoid Witt-Prisms}

\noindent Now we consider the parametrization in some other foundation, namely the corresponding Robba stacks. The story goes over some local perfectoid ring $K_i$ as above for the general $K$ over $L$. $L$ can be of mixed characteristic or equal characteristic. Over such $K_i$ as in \cite{1KL}, \cite{1KL1}, we have the parametrization space which is just defined as the adic spectrum of the perfect Robba rings defined with respect to $K_i$, where we do not take the corresponding Frobenius quotient:
\begin{align}
Y_{K_i}:=\mathrm{Union}_I\mathrm{SpecSpa}(P_{I,K_i},P_{I,K_i}^+).
\end{align}
One then consider the corresponding generalization in \cite{1BS1} and \cite{1F2} to contact with the context we consider here. Namely we have the following version generalization of the Fargues-Fontaine stacks (again in two different characteristic situations):
\begin{align}
&Y_{K_i,2}:=\mathrm{Union}_I\mathrm{SpecSpa}(P_{I,K_i}[b^{1/2}],P_{I,K_i}^+[b^{1/2}]).
\end{align}
Suppose we use the notation $Q$ to denote the structure sheaves of these spaces:
\begin{align}
Q_{Y_{K_i,2}}.
\end{align}
Now over $Q$ we have the corresponding solid quasicoherent sheaves which form $\infty$-categories:
\begin{align}
\mathrm{QC}^\mathrm{solid}Q_{Y_{K_i,2}}.
\end{align}

\begin{definition}
One can then define the corresponding de Rham version of the Robba stacks as in the corresponding prismatization in the following. Locally we consider the completion along all the untilts $\sharp$, which goes in the following way again we forget the corresponding underlying stacks:
\begin{align}
&\mathrm{QC}^\mathrm{solid}_{\mathrm{deRham}}Q_{Y_{K_i,2}}:=\varprojlim_{\sharp} \mathrm{QC}^\mathrm{solid}{Q_{Y_{K_i,2}}}^\wedge_\sharp.
\end{align}
Then let $K_i$ change in the $v$-topology we have the following:
\begin{align}
&\mathrm{QC}^\mathrm{solid}_{\mathrm{deRham}}Q_{Y_{K,2}}.
\end{align}
\end{definition}


\begin{definition}
One can then define the corresponding de Rham version of the Robba stacks as in the corresponding prismatization in the following. Locally we consider the completion along all the untilts $\sharp$, which goes in the following way again we forget the corresponding underlying stacks:
\begin{align}
&\mathrm{QC}^\mathrm{solid}_{\mathrm{deRham}}Q_{Y_{K_i,2}}:=\varprojlim_{\sharp} \mathrm{QC}^\mathrm{solid}{Q_{Y_{K_i,2}}}^\wedge_\sharp.
\end{align}
Then let $K_i$ change in the $v$-topology we have the following:
\begin{align}
&\mathrm{QC}^\mathrm{solid}_{\mathrm{deRham}}Q_{Y_{K,2}}.
\end{align}
We then have a well-defined functor $\mathcal{H}$ which is called generalized de Rhamization:
\begin{align}
\mathrm{QC}^\mathrm{solid}Q_{Y_{K_i,2}}
\longrightarrow
\mathrm{QC}^\mathrm{solid}_{\mathrm{deRham}}Q_{Y_{K,2}},
\end{align}
through the $\sharp$-completion through all the untilts in the coherent way as in the above.
\end{definition}


\begin{remark}
This definition is considering $u$-adic de Rham sheaves, though the internal structure of such sheaves can be simpler the construction is in a uniform framework as in the above. 
\end{remark}


\subsection{Discussion for a lisse chart over $\mathrm{Spd}L$}

\noindent We now assume that the stack is just a lisse chart $\mathcal{L}$ over $\mathrm{Spd}L$, where we have the geometrized generalized $\mathrm{Gamma}_{\mathcal{L},2}$-modules without the Frobenius actions:
\begin{align}
\mathrm{Gamma}_{\mathcal{L},2}\mathrm{QC}^\mathrm{solid}Q_{Y_{\mathrm{Spd}\mathcal{L},2}},
\end{align}
with
\begin{align}
\mathrm{Gamma}_{\mathcal{L},2}\mathrm{QC}_\mathrm{deRham}^\mathrm{solid}Q_{Y_{\mathrm{Spd}\mathcal{L},2}}.
\end{align}

\begin{theorem}
Assume we are in the $p$-adic setting. By projecting to $\sharp=b$ we have the generalized differential equations attached to finite-locally free sheaves in:
\begin{align}
\mathrm{Gamma}_{\mathcal{L},2}\mathrm{QC}^\mathrm{solid}Q_{Y_{\mathrm{Spd}\mathcal{L},2}},
\end{align}
which is further assumed to be generalized de Rham in the obvious generalized way. Here we \textbf{do not} assume the stability and compatibility of the rank throughout the noncompact Stein Robba stacks and we \textbf{do not} assume the finiteness of the rank when we reach the global sections of the Stein stacks. Namely for any such sheaf we can find a projective limit system $D=\varprojlim_w D_w$ to attach to this sheaf, over which we can have the structure of arithmetic $\mathcal{D}$-modules with the action from the group $\mathrm{Gamma}_{\mathcal{L},2}$. 
\end{theorem}

\begin{proof}
We only need to extend the corresponding map from the Robba rings (with respect to some radius in variable of $p^w$) to $L_w[[b]]\otimes L'$ ($L'$ large) into the corresponding situation where we have $b^{1/2}$, then the corresponding formation of the corresponding $w$-th level $p$-adic differential modules in \cite[See and follow the construction around 5.10, the Theorem]{1BA} can be applied directly in our setting, then after we have the construction the corresponding project limit will produce the mixed-parity differential modules. The current morphism here needs the further step of the corresponding deformation which maps the variables $*$ of the lisse chart to $[*^\flat]-1$, which is the difference we need to consider beyond the point situation here. Then the corresponding construction goes in a parallel way. There are two related modules attached to the original module over the Robba ring without Frobenius structure. One is the corresponding differential module as in \cite{1BA} and the other one is the corresponding de Rham module (as in the definition of the corresponding functor). The two modules can be reconctructed from each other by considering the infinite level:
\begin{align}
L_\infty[[b^{1/2}]]\{*^{\pm 1}\}\otimes L',
\end{align}
and by using the invariance of the group $\mathcal{L}$. Then each $D_w$ will be certain preimage (here we need to invert the element $b^{1/2}$) of the bundle $H_w$ for some radius $f(w)$ under the map from the Robba rings in the current setting. This needs to consider the tower:
\begin{align}
&L_w[[b^{1/2}]]\{*^{\pm 1}\}\otimes L',\\
&L_{w+1}[[b^{1/2}]]\{*^{\pm 1}\}\otimes L',\\
&L_{w+2}[[b^{1/2}]]\{*^{\pm 1}\}\otimes L',\\
&L_{w+3}[[b^{1/2}]]\{*^{\pm 1}\}\otimes L',\\
&L_{w+4}[[b^{1/2}]]\{*^{\pm 1}\}\otimes L',\\
&L_{w+5}[[b^{1/2}]]\{*^{\pm 1}\}\otimes L',\\
&...
\end{align}
Then for any such sheaf we can find a projective limit system $D=\varprojlim_w D_w$ to attach to this sheaf, over which we can have the structure of arithmetic $\mathcal{D}$-modules with the action from the group $\mathrm{Gamma}_{\mathcal{L},2}$. 
\end{proof}

\begin{remark}
Also see \cite{1AB}. However we do not have theorems after \cite{1M}, \cite{1K}, \cite{1A} in such generality.
\end{remark}


\newpage
\section{Fundamental Comparison on the Stackifications over Small Arc Stacks}


\subsection{Small Arc-Stacks via Small $v$-Stacks}


\begin{theorem}
Assume we are in our general setting by adding the element $b^{1/2}$. The de Rham-Robba stackification and the de Rham-prismatization stackification in our generalized setting by adding $b^{1/2}$ are equivalent, in both $p$-adic and $z$-adic settings, i.e. in the $p$-adic setting we consider the small $v$-stacks over $\mathrm{Spd}\mathbb{Q}_p$, and in the $z$-adic setting we consider the small $v$-stacks over $\mathrm{Spd}\mathbb{F}_p((t))$, in the $v$-topology. This applies immediately to rigid analytic varieties.
\end{theorem}

\begin{proof}
This is because in the local setting over the $v$-site of any small $v$-stack, what we have will be the corresponding stacks of all the untilts on the prismatization level. Then by taking the limit throughout all the de Rham period sheaves for all the untilts in our generalized setting, we reach the same $\infty$-categories. This construction is then in the local perfectoid setting identical in the both approaches.
\end{proof}


\noindent We can promote this equivalence to the categorical level. One can construct the following functor:

\begin{definition}
Assume we are in our general setting by adding the element $b^{1/2}$. Let $S$ be a small $v$-stack, which can be either over $\mathrm{Spd}\mathbb{Q}_p$ or $\mathrm{Spd}\mathbb{F}_p((u))$. We use the notation:
\begin{align}
\mathrm{deRhamRobba}_S
\end{align}
to denote the corresponding de Rham-Robba stackification from the FF stacks, in our generalized setting. And we use the notation
\begin{align}
\mathrm{deRhamPrismatization}_S
\end{align}
to denote the corresponding de Rham Prismatization stackification, in our generalized setting. And for $?= \mathrm{deRhamRobba}_S, \mathrm{deRhamPrismatization}_S$ we use the notation:
\begin{align}
\mathrm{SolidQuasiCoh}_?
\end{align}
to denote the corresponding condensed $\infty$-categories of the corresponding solid quasicoherent sheaves over $?$. Then we have a functor:
\begin{align}
\mathrm{SolidQuasiCoh}_{\mathrm{deRhamPrismatization}_S}\rightarrow \mathrm{SolidQuasiCoh}_{\mathrm{deRhamRobba}_S}
\end{align}
by taking the induced functor from identification of the de Rham functors on the perfectoids.
\end{definition}


\begin{theorem}
Assume we are in our general setting by adding the element $b^{1/2}$. The functor defined above:
\begin{align}
\mathrm{SolidQuasiCoh}_{\mathrm{deRhamPrismatization}_S}\rightarrow \mathrm{SolidQuasiCoh}_{\mathrm{deRhamRobba}_S}
\end{align}
is well-defined, and an equivalence of symmetrical monoidal $\infty$-categories which are stable.

\end{theorem}

\begin{proof}
Locally over perfectoid spaces the corresponding prismatization is actually the corresponding Witt vector in both $p$-adic and equal characteristic situations. Then this identification of the prismatization and the Witt vector rings in fact directly produces the desired equivalence, since after this identification the construction will be identical completely.
\end{proof}


\noindent We then contact Scholze's notion of small arc stacks in \cite{1S5} \cite{1S6} where local charts will be simply Banach rings, then we have the following definition:

\begin{definition}
Assume we are in our general setting by adding the element $b^{1/2}$. Let $S$ be a small arc-stack in \cite{1S5} over $\mathbb{Q}_p$ or $\mathbb{F}((u))$. We use the notation:
\begin{align}
\mathrm{deRhamRobba}_S
\end{align}
to denote the corresponding de Rham-Robba stackification from the FF stacks, in our generalized setting. And we use the notation
\begin{align}
\mathrm{deRhamPrismatization}_S
\end{align}
to denote the corresponding de Rham Prismatization stackification, in our generalized setting. And for $?= \mathrm{deRhamRobba}_S, \mathrm{deRhamPrismatization}_S$ we use the notation:
\begin{align}
\mathrm{SolidQuasiCoh}_?
\end{align}
to denote the corresponding condensed $\infty$-categories of the corresponding solid quasicoherent sheaves over $?$. When we consider the de-Rham Robba stackification we consider the corresponding $v$-stack associated to $S$, which is denoted by $\mathrm{Stack}_v(S)$ after \cite{1S5}. Then we have a functor:
\begin{align}
\mathrm{SolidQuasiCoh}_{\mathrm{deRhamPrismatization}_S}\rightarrow \mathrm{SolidQuasiCoh}_{\mathrm{deRhamRobba}_{\mathrm{Stack}_v(S)}}
\end{align}
by taking the induced functor from identification of the de Rham functors on the perfectoids, i.e. we set the Banach ring local chart to be perfectoid to reach the objects in the second $\infty$-category. 
\end{definition}


\begin{theorem}
Assume we are in our general setting by adding the element $b^{1/2}$. The functor defined above:
\begin{align}
\mathrm{SolidQuasiCoh}_{\mathrm{deRhamPrismatization}_S}\longrightarrow \mathrm{SolidQuasiCoh}_{\mathrm{deRhamRobba}_{\mathrm{Stack}_v(S)}}
\end{align}
is well-defined, as a symmetrical monoidal $\infty$-tensor functor.
\end{theorem}

The following theorem will then be highly nontrivial:

\begin{theorem}
Assume we are in our general setting by adding the element $b^{1/2}$. The functor defined above:
\begin{align}
\mathrm{SolidQuasiCoh}_{\mathrm{deRhamPrismatization}_S}\longrightarrow \mathrm{SolidQuasiCoh}_{\mathrm{deRhamRobba}_{\mathrm{Stack}_v(S)}}
\end{align}
is fully faithful functor of symmetrical monoidal $\infty$-categories which are stable.
\end{theorem}

\begin{proof}
Here we follow \cite{1S5} \cite{1S6}, where we pass the whole functor to the totally disconnected subspaces in our current nonarchimedean settings. On the both sides they are exactly perfectoid coverings. Then following \cite{1S5} \cite{1S6} we pass to spaces taking the forms of the adic spectrum of algebraically closed fields. Then there is nothing to prove then, since both sides have the same underlying spaces, then the theorem follows.
\end{proof}

\begin{remark}
However we want to mention that the de Rham Robba stackification can live over small arc-stacks by using the tiltings of local Banach subspaces, therefore we will have the corresponding de Rham Robba stackification over the small arc-stacks in some functorial way as well. This can also be achived by taking the perfectoidization of the corresponding de Rham prismatization over small arc-stacks directly.
\end{remark}


\subsection{6-Functor Formalism for Generalized Prismatization}\label{section7.2}


\begin{assumption}\label{assumption2}
In this \cref{section7.2}, all our considerations on the prismatization, de Rham Robba stackification and de Rham prismatization are assumed to be the generalized ones, i.e. we consider the corresponding prismatizations with $b^{1/2}$ added, de Rham Robba stackifications with $b^{1/2}$-added and de Rham prismatizations with $b^{1/2}$ added.
\end{assumption}


\begin{theorem}
Assume we are in the situation in \cref{assumption2}. $\mathrm{SolidQuasiCoh}_{\mathrm{deRhamPrismatization}_*}$ admits pullback functor and pushforward functor $F^\square$ and $F_\square$ in the 6-functor formalism where $*$ is varying in the category of all the small arc-stacks or all small $v$-stacks.
\end{theorem}

\begin{proof}
We follow the method of proof after Scholze in \cite{1S5} by passing to totally disconnected subspaces from the adic spectrum of corresponding algebraically closed fields. Then one can see that the corresponding $\infty$-category can be regarded as the corresponding generalized Galois representations with coefficients over those generalized de Rham period sheaves then the corresponding result will follow since the 6-functor will be reduced to 6-functors among condensed generalized Galois representations. But over the such geometrical points the Galois groups are trivial. Then we are done at least in the current situation where we only consider the pullback and pushforward functors.
\end{proof}

\begin{definition}
Assume we are in the situation in \cref{assumption2}. Since $\mathrm{SolidQuasiCoh}_{\mathrm{deRhamPrismatization}_*}$ are symmetrical monoidal $\infty$-categories, the corresponding motivic Galois group exists as the corresponding Tannakian groups we will use:
\begin{align}
\pi_{\mathrm{SolidQuasiCoh}_{\mathrm{deRhamPrismatization}_*}}
\end{align}
to denote the group in any particular situation over $*$.
\end{definition}



\begin{theorem}
Assume we are in the situation in \cref{assumption2}. $\mathrm{SolidQuasiCoh}_{\mathrm{deRhamRobba}_*}$ admits pullback functor and pushforward functor $F^\square$ and $F_\square$ in the 6-functor formalism where $*$ is varying in the category of all the small $v$-stacks.
\end{theorem}

\begin{proof}
We follow the method of proof after Scholze in \cite{1S5} by passing to totally disconnected subspaces from the adic spectrum of corresponding algebraically closed fields. Then one can see that the corresponding $\infty$-category can be regarded as the corresponding generalized Galois representations with coefficients over those generalized de Rham period sheaves then the corresponding result will follow since the 6-functor will be reduced to 6-functors among condensed generalized Galois representations with trivial Galois groups.
\end{proof}

\begin{definition}
Assume we are in the situation in \cref{assumption2}. Since $\mathrm{SolidQuasiCoh}_{\mathrm{deRhamRobba}_*}$ are symmetrical monoidal $\infty$-categories, the corresponding motivic Galois group exists as the corresponding Tannakian groups we will use:
\begin{align}
\pi_{\mathrm{SolidQuasiCoh}_{\mathrm{deRhamRobba}_*}}
\end{align}
to denote the group in any particular situation over $*$.
\end{definition}


\begin{theorem}
Assume we are in the situation in \cref{assumption2}. $\mathrm{PerfComplex}\mathrm{SolidQuasiCoh}_{\mathrm{deRhamRobba}_*}$ or $\mathrm{PerfComplex}\mathrm{SolidQuasiCoh}_{\mathrm{deRhamPrismatization}_*}$ admits pullback functor and pushforward functor $F^\square$ and $F_\square$ in the 6-functor formalism where $*$ is varying in the category of all the small $v$-stacks (while for the de Rham primatization $*$ can also be the corresponding small arc stacks). Here the notation means we consider all the perfect complexes in the solid quasicoherent sheaves. 
\end{theorem}

\begin{proof}
We follow the method of proof after Scholze in \cite{1S5} by passing to totally disconnected subspaces from the adic spectrum of corresponding algebraically closed fields. Then one can see that the corresponding category can be regarded as the corresponding generalized Galois representations with coefficients over those generalized de Rham period sheaves then the corresponding result will follow since the 6-functor will be reduced to 6-functors among condensed generalized Galois representations with trivial Galois groups. Then the corresponding functor will be either we take the base change over the genearalized de Rham rings with respect to different algebraically closed fields or we consider the compositions of homomorphisms from the Galois groups. But over the geometric points these are trivial. 
\end{proof}


\begin{remark} \mbox{\textbf{(Full 6-functor formalism)}}
The $!$-adjoint pairs in the current context are obviously those maps in \cite{1S5}, \cite{1S6} but the tricky part in the proof is to derived preservation of the corresponding local finiteness theorem for the perfect complexes. However one can easily prove this by using the corresponding idea presented above following \cite{1S5}, \cite{1S6} by consider the corresponding geometric points, which then eventually reduct to modules (with trivial Galois actions) over the integral generalized de Rham period rings for these geometric points. We use the notation $F_\blacksquare$ and $F^\blacksquare$ to denote the $!$-adjoint pairs as in \cite{1S5}, \cite{1S6}.
\end{remark}


\subsection{Application to Local Langlands} \label{subsection7.3}

\noindent We now follow \cite{1S5}, \cite{1S6}, \cite{L1}, \cite{1FS} to construct some generalized version of the local Langlands correspondence in \cite{1FS} by using the de Rham prismatization we constructed above in the generalized setting. 


\begin{assumption} \label{assumption3}
All our considerations below in \cref{subsection7.3} on the prismatization, de Rham Robba stackification and de Rham prismatization are assumed to be the generalized ones, i.e. we consider the corresponding prismatizations with $b^{1/2}$ added, de Rham Robba stackifications with $b^{1/2}$ added and de Rham prismatizations with $b^{1/2}$ added. 
\end{assumption}

\begin{assumption}
The de Rham prismatizations and de Rham Robba stackifications in the following are all $p$-adic.
\end{assumption}


\noindent Recall the corresponding context in \cite{1FS} we have the corresponding $p$/$z$-adic group $G(F)$ with some local field $F$, this will provide the corresponding small arc stacks as in \cite{1S5}, \cite{1S6}. Since we have the Tannakian groups defined above, we can tranform a representation of the Tannakian group into the corresponding category on the other side. This process will define the following correponding functor.


\begin{theorem}
Assume we are in the situation of \cref{assumption3}. We have well-defined functors which are well-defined and isomorphisms:
\begin{align}
\mathrm{Repre}(?) \overset{\sim}{\rightarrow}  ! 
\end{align}
$?$ = $\pi_{\mathrm{SolidQuasiCoh}_{\mathrm{deRhamPrismatization}_*}}$, or $\pi_{\mathrm{SolidQuasiCoh}_{\mathrm{deRhamRobba}_*}}$, $!$ = $\mathrm{SolidQuasiCoh}_{\mathrm{deRhamPrismatization}_*}$ or $\mathrm{SolidQuasiCoh}_{\mathrm{deRhamRobba}_*}$.
\end{theorem}


\begin{proof}
By Tannakian formalism.
\end{proof}


\begin{corollary}
Assume we are in the situation of \cref{assumption3}. For $p$-adic representations of the Langlands dual group with coefficients in $\overline{\mathbb{Q}}_p$, by taking composition with the representation of the Tannakian group $\pi_{\mathrm{SolidQuasiCoh}_{\mathrm{deRhamPrismatization}_*}}$, or $\pi_{\mathrm{SolidQuasiCoh}_{\mathrm{deRhamRobba}_*}}$, we end up with certain complexes in $\mathrm{SolidQuasiCoh}_{\mathrm{deRhamPrismatization}_*}$ or $\mathrm{SolidQuasiCoh}_{\mathrm{deRhamRobba}_*}$.
\end{corollary}

 
\indent Now we consider the moduli $v$-stacks in \cite{1FS}, we denote it by $Y_{\mathrm{FS},G}$ which is the $v$-stack of $G$-bundles in both the equal characteristic and mixed characteristic situations for some local field $F$. We now consider the following following \cite{1FS}, \cite{1GL}. We actually relying on \cite{1FS}, \cite{1S5}, \cite{1S6}  can derive the Hecke operators:

\begin{definition}
Assume we are in the situation of \cref{assumption3}. Consider the map from the Hecke stack in \cite{1FS} which we denote that as $Y_{\mathrm{Hecke},G,I}$, and consider the map from this to fiber product of the Cartier stack $Y_\mathrm{Cartier}$ with the corresponding stack $Y_{\mathrm{FS},G}$, and consider the map from this Hecke stack to the $Y_{\mathrm{FS},G}$. Pulling back along the second and push-forward the product with $\square_O$ will define the Hecke operator, where $\square_O$ is defined for some representation of the Langlands full-dual group in the coefficient $\overline{\mathbb{Q}}_p$. By result in \cite{1S5}, \cite{1S6} we have the construction does not depend on the choice of the primes, so we can in some equivalent way to derive a corresponding $\overline{\mathbb{Q}}_p$-complex over the Hecke stacks with some finite set $I$. For instance after \cite{1FS} we have $\overline{\mathbb{Q}}_\ell$-adic complex with $\ell$ away from $p$. Take any motivic sheaf in \cite{1S5}, \cite{1S6} with $\ell$-adic realization which is isomorphic to this complex\footnote{Many things can now be defined over $\mathbb{Z}$ after \cite{1S5}, \cite{1S6}. First the Satake isomorphism can now be defined over $\mathbb{Z}$, one then just take the $p$-adic realization to reach our definition here by taking the base change from $\mathbb{Z}$ to $\mathbb{Z}_p$ which provides the desired complex here over the Hecke stack in order to finish the definition of the Hecke operators in our current setting.}. Then we consider the $p$-adic realization which provides a corresponding $p$-adic complex. Then one can take the base change to the corresponding $\varprojlim_\alpha B^+_\mathrm{dR,\alpha}[b^{1/2}]$-period ring to achieve finally an object in the category we are considering. This gives us desired $p$-adic complex over the Hecke stack, then one defines the corresponding morphisms from the Hecke stack for each finite set as above to $Y_{\mathrm{FS},G}$ and to $Y_{\mathrm{FS},G}\times Y'$. Here $Y'$ is defined to be the $v$-stack of all the solid quasicoherent sheaves over the two de Rham stackifications in our setting, over the Cartier stack $Y_{\mathrm{Cartier}}$ and those products of this Cartier stack. This will produce the desired Hecke operators.
\end{definition}


\begin{theorem}
Assume we are in the situation of \cref{assumption3}. The Hecke operator sends the complexes to those complexes carrying the action from the products of Tannakian group of the Cartier stack for $F$.
\end{theorem}

\begin{proof}
By our definition we have that the corresponding image complexes are those complexes over the corresponding fiber product of $Y_{\mathrm{FS},G}$ with the corresponding classifying stack of the product of the Tannakian group as in the statement of this theorem. For instance one can check this following the idea in \cite{1S5}, \cite{1S6} where we consider each totally disconneted subspace taking the form of the adic spectrum of some algebraically closed field. Over these algebraically closed geometric points we can see that we end up with purely perfect complexes of modules over:
\begin{align}
\varprojlim_\alpha B^+_\mathrm{dR,\alpha}[b^{1/2}],
\end{align}
but we do have the corresponding lattices then, which reduces to the correponding $\overline{\mathbb{Q}}_p$-situation. Here the action of Tannakian group for $F$ will then factors through the action of corresponding Weil group for $F$. Then we are in a situation parallel to \cite{1FS} and we only have to consider the action from the products of the Weil groups. Then the same proof as in \cite{1FS} will derive the result stated. In fact there is nothing to prove here once one follow the same ideas in \cite{1FS}, in particular the proposition IX.1.1.
\end{proof}

\begin{theorem}
Assume we are in the situation of \cref{assumption3}. One direction of the local Langlands holds true in this context: from Schur-irreducible objects to the corresponding $L$-parameters from the $?$ in our current context. Here $?$ = 
\begin{align}
\pi_{\mathrm{SolidQuasiCoh}_{\mathrm{deRhamPrismatization}_*}}, 
\end{align}
or 
\begin{align}
\pi_{\mathrm{SolidQuasiCoh}_{\mathrm{deRhamRobba}_*}}. 
\end{align}
$*$ is just the Cartier stack attached to $F$. Note that all the coefficients on the both sides are $p$-adic. 
\end{theorem}

\begin{proof}
By our construction we do have the mapping to the Bernstein centers in this current setting. Then as in VIII.4.1 and IV.4.1 of \cite{1FS} we can build up the corresponding mapping after \cite{1VL}. To be more precise for each finite set $I$ we can build up the corresponding symmetrical monoidal $\infty$-categories and the Hecke functors as in the above in our current setting, and we have the corresponding equivariant actions from the Tannakian groups on the Target symmetrical monoidal $\infty$-categories over the moduli $v$-stack. Then the excursion operators are generated automatically after \cite{1FS} and \cite{1VL}, where all these general abstract formalism will apply in our setting directly.
\end{proof}

\begin{remark}
Our consideration might be not a \textit{correct} generalization for ultimate $p$-adic local Langlands correspondence after \cite{C}. As in \cite{EGH1} our consideration at least (we don't know more about this) at this moment covers the $p$-adic smooth representations of the $p$-adic reductive groups, since the Hecke operators rely on Scholze's motivicalization with realization into $p$-adic sheaves over $v$-stacks. This might be serious thing to genuinely consider our consideration as correct generalization of Colmez's work and the context in Colmez's work. But our consideration does provide certain local Langlands correspondence in $p$-adic coefficients, and note that our consideration goes beyond the pro-\'etale local systems. However it is less clear to us if the corresponding \textit{$p$-adic spectral action} in such solid quasi-coherent sheaf level will produce the equivalence of the derived $\infty$-categories on the both sides as at least conjectured in \cite{1FS}. This is because we do not at this moment put more requirement on the motivic categories.
\end{remark}


\begin{remark}
This direction of Langlands correspondence is already many to one, however we want to mention that one can have a larger package if one changes the motivic Galois groups on the Weil side. For instance as in \cite{1T} on the Weil side we considered just the Weil groups (and their covering groups), this means that the Weil groups act as if they act through any full motivic Galois group related to them (but only through subgroups or quotient groups). This may enlarge the $L$-packages and may make the categoricalized functor on the $\infty$-categoricalized level fully faithful. However the correspondence still makes sense.
\end{remark}




\newpage
\section{Motivic $p$-adic Local Langlands Correspondence in Families}


\indent Function fields over finite fields share kind of arithmetical similarity to the local fields with mixed-characteristics. However due to the complication of positivity of the characteristics, many parallel obvious analogs of results might be extremely different and difficult. This is a problem for instance in prismatization as in our work in \cite{2T1}. Following \cite{2LH} we give a compactness method by taking the infinite induction from mixed-characteristic fibers to reach the function field by compactification of the natural numbers. We then study prismatization in families following \cite{2G}, \cite{2BS}, \cite{2BL}, \cite{2D}, \cite{2BL2}. Our construction rely on the corresponding construction in the following sense. For each $n$ one can pick some $p$-adic field with ramification index expanding, this process will define all family of the corresponding prismatization, de Rham prismatization and the corresponding Hodge-Tate prismatization. Then we consider the closure of the $\infty$-categories along the corresponding identification that $\overline{\mathbb{N}}= \mathbb{N}^\wedge$. One has a correspondence between categories over some finite $n_0$ and categories over $\infty$ through the corresponding identification on the ring level, which is the key idea in the proof of the generalized local Langlands in families in \cite{2LH}. We follow this idea significantly in this paper to reach the definitions of prismatizations at $\infty$, i.e. a function field. We then have a uniformed strategy on the prismatization. We then apply our construction to motivic $p$-adic local Langlands after \cite{1S5}, \cite{1S6}, \cite{1RS}, \cite{1FS}.







\subsection{Absolute prismatization}

\begin{setting}
We now consider the foundation in \cite{2LH} on the generalized topological ring $R$, which is defined in the following. We fix a finite field $\mathbb{F}_q$ as a universal residue field. Then we fix a family over $\mathbb{N}^\wedge$ (where we add the corresponding infinity as one point to the set $\mathbb{N}$) of local fields:
\begin{align}
K_1,K_2,...,K_\infty.
\end{align}
Then for each $n\in \mathbb{N}^\wedge$ we just fix some uniformizer $t_n\in K_n$. Then we just put $t$ to be the uniformizer for the topological ring $R$ to be defined in the following which evaluates at each $n$ to be $t(n)=t_n$. Then we just consider:
\begin{align}
R = \prod_{n\in \mathbb{N}^\wedge} \mathcal{O}_{K_n}
\end{align}
where we have that:
\begin{align}
R = \varprojlim_k R/t^k R.
\end{align}
Then we have the corresponding ring $K$ by simply inverting all the $t_n,n\in \mathbb{N}^\wedge$. The $\infty$-local field is then required to be a function field with uniformizer $t_\infty$ over the field $\mathbb{F}_q$. 
\end{setting}

\begin{setting}
As in \cite{2LH} we have the corresponding Witt vector $\mathrm{Witt}$ which is a functor from the category of $R$-algebras to the category of all the $R$-algebras which are $t$-complete.
\end{setting}

\begin{definition}
We now define prisms in our current setting. A prism in families is a pair over $\mathbb{N}$, which is denoted by $(L,A)$ where $L$ is a Cartier divisor in $\mathrm{Spec}A$ such that we have $A$ is derived $(t,L)$-complete and we have that the corresponding requirement that for each $n\in \mathbb{N}$ we require that $(L_n,A_n)$ is a prism in \cite{2BS}. This can be regarded as a corresponding family of prisms in \cite{2BS} such as if we put:
\begin{align}
\mathbb{N} = \varinjlim_i \{x\in \mathbb{N}|x\leq i\}.
\end{align} 
Then we can restrict the whole family of prisms in our current definition to each subspaces in this limit. Note that here $t$ is in our notation is just $t|\mathbb{N}$. We use the notation:
\begin{align}
\mathrm{Prism}_\mathbb{N}
\end{align}
to denote all the prisms fibered over $\mathbb{N}$, then we take the fiber product along $\mathbb{N}\hookrightarrow \mathbb{N}^\wedge$ to reach the definition of the prisms over $\mathbb{N}^\wedge$ in particular we have the definition of the prisms at the $\infty$. We use the notation:
\begin{align}
\widetilde{\mathrm{Prism}_\mathbb{N}}
\end{align}
to denote the category of all the prisms over $\mathbb{N}^\wedge$. Here we consider first the following category by taking the quotient by some power of $t$:
\begin{align}
\mathrm{Prism}_{\mathbb{N},R/t^a}
\end{align}
By the identification consider in \cite{2LH} we can take some $n_0\in \mathbb{N}$ and set the corresponding 
\begin{align}
\widetilde{\mathrm{Prism}_\mathbb{N}}(\infty)
\end{align}
to be just:
\begin{align}
\mathrm{Prism}_{\mathbb{N},\mathcal{O}_{K_{n_0}}/t_{n_0}^a}
\end{align}
which produces the corresponding mod $t^a$ category of all the prisms:
\begin{align}
\widetilde{\mathrm{Prism}_\mathbb{N}}_{R/t^a}
\end{align}
and then we define:
\begin{align}
\widetilde{\mathrm{Prism}_\mathbb{N}} = \varprojlim_a \widetilde{\mathrm{Prism}_\mathbb{N}}_{R/t^a}.
\end{align}
\end{definition}


\begin{definition}
We now define the corresponding Cartier stack which is the version of preprismatization in our setting. Consider first the Witt vector functor $\mathrm{Witt}$ fibered over the following category of all the $t$-nilpotent rings over $R$ in a family over $\mathbb{N}^\wedge$:
\begin{align}
\mathrm{Ring}_{\mathrm{Nil},\mathbb{N}^\wedge,t,R}.
\end{align}
Then consider another functor in the following sense which is called the Cartier stack in our setting:
\begin{align}
\mathrm{Cartier},
\end{align}
which parametrize over each $X\in \mathrm{Ring}_{\mathrm{Nil},\mathbb{N}^\wedge,t,R}$ the corresponding generalized Cartier ideal $(L,f)$ mapping under $f$ to $\mathrm{Witt}(X)$.
\end{definition}


\begin{definition}
We now define the corresponding Cartier-Witt stack which is the version of prismatization in our setting. Consider first the Witt vector functor $\mathrm{Witt}$ fibered over the following category of all the $t$-nilpotent rings over $R$ in a family over $\mathbb{N}^\wedge$:
\begin{align}
\mathrm{Ring}_{\mathrm{Nil},\mathbb{N}^\wedge,t,R}.
\end{align}
Then consider another functor in the following sense which is called the Cartier-Witt stack in our setting:
\begin{align}
\mathrm{Cartier}_W,
\end{align}
which parametrize over each $X\in \mathrm{Ring}_{\mathrm{Nil},\mathbb{N}^\wedge,t,R}$ the corresponding generalized Cartier ideal $(L,f)$ mapping under $f$ to $\mathrm{Witt}(X)$. We then furthermore require that the image of $(L,f)$ is principle generated by element which evaluates to a distinguished element for each $n\in \mathbb{N}^\wedge$, as in the usual situation in \cite{2BL}. As in \cite{2BL} we put the structure sheaf $\mathcal{F}$ as for each $X$ we set the section over $(L,f)$ with $f:L\rightarrow \mathrm{Witt}(X)$ to be just $X$ itself. We also consider the prismatization with filtration as in the usual situation, to do this we first set the notation $\mathrm{Cartier}_{W,\mathrm{NFil}}$ for the prismatization carrying the corresponding Nygaard filtration. We define this to be first the prismatization fibered over $\mathbb{N}$. Then we have the morphism:
\begin{align}
\mathrm{Cartier}_{W,\mathrm{NFil}} \rightarrow \mathrm{Cartier}_{W},
\end{align}
fibered over $\mathbb{N}$, by projection to the families over $\mathbb{N}$. Then we take the fibered product from $\mathrm{Cartier}_W|_{\mathbb{N}}$ to $\mathrm{Cartier}_{W}$ to reach the full stack: $\widetilde{\mathrm{Cartier}_{W,\mathrm{NFil}}}$ fibered over $\mathbb{N}^\wedge$.
\end{definition}

\begin{proposition}
We have a natural isomorphism between the following two formal stacks. The first one is the Cartier-Witt stack defined over the $\mathbb{N}^\wedge$ The second one is the compactification along $\mathbb{N}\hookrightarrow \mathbb{N}^\wedge$ of the Cartier-Witt stack over $\mathbb{N}$ after the restriction for instance. This means that we quotient out some $t^a$ and set the section of the stack at $\infty$ to be the section over some point $n'$, then take the inverse limit over $a$.
\end{proposition}

\begin{proof}
In fact we only have to consider what happens when we evaluate both stacks at $\infty$. In such a situation the first stack goes to be exactly the definition we have for the function field. Then for the corresponding compactification onto $\mathbb{N}^\wedge$, along the lifting of the ramification degrees, the corresponding distinguished element participate into the Witt wector to define each finite $n$ prismatization will eventual give rise to the need distinguished element as at $\infty$. This finishes the proof.
\end{proof}







\begin{remark}
By restricting to $\mathbb{N}$ of our prismatization from $\mathbb{N}^\wedge$ we have the following equivalence on the derived $\infty$-categories of complexes:
\begin{align}
\mathrm{Der}_{\mathcal{F}|_{\mathbb{N}}}\overset{\sim}{\rightarrow} \projlim_{(L,A)}\overline{\mathrm{Der}}_{(t,L)}(A)
\end{align}
where $(L,A)$ varies in the set of all the prisms in our setting. $\overline{Der}$ means the completion in the derived sense with respect to $t$ and $L$ over the complexes on the right of this equivalence.
\end{remark}


\begin{definition}
Now here we restrict to the space $\mathbb{N}$ is because that we have no relationship like this at the $\infty$ since that is a function field. But we can then take the corresponding compactification in the following sense. We call the above equivalence open prismatization equivalence $\mathrm{ope}_\mathbb{N}$:
\begin{align}
\mathrm{Der}_{\mathcal{F}|_{\mathbb{N}}}\overset{\sim}{\rightarrow} \projlim_{(L,A)}\overline{\mathrm{Der}}_{(t,L)}(A).
\end{align}
Then we define the following compact prismaization equivalence $\mathrm{cpe}$:
\begin{align}
\mathrm{Der}_{\mathcal{F}}\overset{\sim}{\rightarrow} \widetilde{\projlim_{(L,A)}\overline{\mathrm{Der}}_{(t,L)}(A)}.
\end{align}
Here the right hand side is the $\infty$-category which is the fiber product of $\mathrm{Der}_{\mathcal{F}}$ and $\projlim_{(L,A)}\overline{\mathrm{Der}}_{(t,L)}(A)$ along the map to the restriction over $\mathbb{N}$. 
\end{definition}

\begin{definition}
This also has a filtered version from Nygaard filtraion. Now here we restrict to the space $\mathbb{N}$ is because that we have no relationship like this at the $\infty$ since that is a function field. But we can then take the corresponding compactification in the following sense. We call the above equivalence open prismatization equivalence $\mathrm{ope}_\mathbb{N}$:
\begin{align}
\mathrm{Der}_{\mathcal{F}|_{\mathbb{N}}}\overset{\sim}{\rightarrow} \projlim_{(L,A,\mathrm{NFil})}\overline{\mathrm{Der}}_{(t,L)}(A).
\end{align}
Then we define the following compact prismaization equivalence $\mathrm{cpe}$:
\begin{align}
\mathrm{Der}_{\mathcal{F}}\overset{\sim}{\rightarrow} \widetilde{\projlim_{(L,A,\mathrm{NFil})}\overline{\mathrm{Der}}_{(t,L)}(A)}.
\end{align}
Here the right hand side is the $\infty$-category which is the fiber product of $\mathrm{Der}_{\mathcal{F}}$ and $\projlim_{(L,A)}\overline{\mathrm{Der}}_{(t,L)}(A)$ along the map to the restriction over $\mathbb{N}$. \end{definition}


\begin{proposition}
The fiber product resulted isomorphism:
\begin{align}
\mathrm{Der}_{\mathcal{F}}\overset{\sim}{\rightarrow} \widetilde{\projlim_{(L,A)}\overline{\mathrm{Der}}_{(t,L)}(A)},
\end{align}
is the same as the $\mathbb{N}\hookrightarrow \mathbb{N}^\wedge$ compactification of the restriction over $\mathbb{N}$ of this isomorphism.
\end{proposition}

\begin{proof}
This is because we have that the category $\mathrm{Der}_{\mathcal{F}}$ has the exactly the same behavior, though it can be defined directly using the Witt vector in \cite{2LH}.
\end{proof}








\subsection{Prismatization for $R$-formal schemes}

\begin{setting}
We now consider a $R$-formal ring which is $t$-adic complete, which is denoted by $H$. We assume that $H$ is fibered over $\mathbb{N}^\wedge$.
\end{setting}

\begin{definition}
We now define the corresponding Cartier stack which is the version of preprismatization in our setting. Consider first the Witt vector functor $\mathrm{Witt}$ fibered over the following category of all the $t$-nilpotent rings over $R$ in a family over $\mathbb{N}^\wedge$:
\begin{align}
\mathrm{Ring}_{\mathrm{Nil},\mathbb{N}^\wedge,t,R}.
\end{align}
Then consider another functor in the following sense which is called the Cartier stack in our setting:
\begin{align}
\mathrm{Cartier}_H,
\end{align}
which parametrize over each $X\in \mathrm{Ring}_{\mathrm{Nil},\mathbb{N}^\wedge,t,R}$ the corresponding generalized Cartier ideal $(L,f)$ mapping under $f$ to $\mathrm{Witt}(X)$. We then require further that there is open immersion from $\mathrm{Spec}\mathrm{Witt}(X)/L$. 
\end{definition}


\begin{definition}
We now define the corresponding Cartier-Witt stack which is the version of prismatization in our setting. Consider first the Witt vector functor $\mathrm{Witt}$ fibered over the following category of all the $t$-nilpotent rings over $R$ in a family over $\mathbb{N}^\wedge$:
\begin{align}
\mathrm{Ring}_{\mathrm{Nil},\mathbb{N}^\wedge,t,R}.
\end{align}
Then consider another functor in the following sense which is called the Cartier-Witt  stack in our setting:
\begin{align}
\mathrm{Cartier}_{W,H},
\end{align}
which parametrize over each $X\in \mathrm{Ring}_{\mathrm{Nil},\mathbb{N}^\wedge,t,R}$ the corresponding generalized Cartier ideal $(L,f)$ mapping under $f$ to $\mathrm{Witt}(X)$. We then furthermore require that the image of $(L,f)$ is principle generated by element which evaluates to a distinguished element for each $n\in \mathbb{N}^\wedge$, as in the usual situation in \cite{2BL}. As in \cite{2BL} we put the structure sheaf $\mathcal{F}$ as for each $X$ we set the section over $(L,f)$ with $f:L\rightarrow \mathrm{Witt}(X)$ to be just $X$ itself. We then require further that there is open immersion from $\mathrm{Spec}\mathrm{Witt}(X)/L$. We also consider the prismatization with filtration as in the usual situation, to do this we first set the notation $\mathrm{Cartier}_{W,H,\mathrm{NFil}}$ for the prismatization carrying the corresponding Nygaard filtration. We define this to be first the prismatization fibered over $\mathbb{N}$. Then we have the morphism:
\begin{align}
\mathrm{Cartier}_{W,H,\mathrm{NFil}} \rightarrow \mathrm{Cartier}_{W,H},
\end{align}
fibered over $\mathbb{N}$, by projection to the families over $\mathbb{N}$. Then we take the fibered product from $\mathrm{Cartier}_{W,H}|_{\mathbb{N}}$ to $\mathrm{Cartier}_{W,H}$ to reach the full stack: $\widetilde{\mathrm{Cartier}_{W,H,\mathrm{NFil}}}$ fibered over $\mathbb{N}^\wedge$.
\end{definition}

\begin{remark}
By restricting to $\mathbb{N}$ of our prismatization from $\mathbb{N}^\wedge$ we have the following equivalence on the derived $\infty$-categories of complexes:
\begin{align}
\mathrm{Der}_{\mathcal{F}|_{\mathbb{N}}}\overset{\sim}{\rightarrow} \projlim_{(L,A)/H}\overline{\mathrm{Der}}_{(t,L)}(A)
\end{align}
where $(L,A)$ varies in the set of all the prisms in our setting, over $H$. $\overline{Der}$ means the completion in the derived sense with respect to $t$ and $L$ over the complexes on the right of this equivalence.
\end{remark}


\begin{definition}
Now here we restrict to the space $\mathbb{N}$ is because that we have no relationship like this at the $\infty$ since that is a function field. But we can then take the corresponding compactification in the following sense. We call the above equivalence open prismatization equivalence $\mathrm{ope}_\mathbb{N}$:
\begin{align}
\mathrm{Der}_{\mathcal{F}|_{\mathbb{N}}}\overset{\sim}{\rightarrow} \projlim_{(L,A)/H}\overline{\mathrm{Der}}_{(t,L)}(A).
\end{align}
Then we define the following compact prismaization equivalence $\mathrm{cpe}$:
\begin{align}
\mathrm{Der}_{\mathcal{F}}\overset{\sim}{\rightarrow} \widetilde{\projlim_{(L,A)/H}\overline{\mathrm{Der}}_{(t,L)}(A)}.
\end{align}
Here the right hand side is the $\infty$-category which is the fiber product of $\mathrm{Der}_{\mathcal{F}}$ and $\projlim_{(L,A)}\overline{\mathrm{Der}}_{(t,L)}(A)$ along the map to the restriction over $\mathbb{N}$. \end{definition}


\begin{theorem}
The definitions above for $\mathrm{Cartier}_*$ and $\mathrm{Cartier}_{W,*}$ for $*$ a $R$ formal ring $t$-adic fibered over $\mathbb{N}^\wedge$ admit descent over $R$ formal scheme $S$ $t$-adic fibered over $\mathbb{N}^\wedge$. 
\end{theorem}

\begin{proof}
We reduce to \cite{2BL}.
\end{proof}

\begin{remark}
By restricting to $\mathbb{N}$ of our prismatization from $\mathbb{N}^\wedge$ we have the following equivalence on the derived $\infty$-categories of complexes:
\begin{align}
\mathrm{Der}_{\mathcal{F}|_{\mathbb{N}}}\overset{\sim}{\rightarrow} \projlim_{(L,A)/S}\overline{\mathrm{Der}}_{(t,L)}(A)
\end{align}
where $(L,A)$ varies in the set of all the prisms in our setting, over $H$. $\overline{Der}$ means the completion in the derived sense with respect to $t$ and $L$ over the complexes on the right of this equivalence.
\end{remark}


\begin{definition}
Now here we restrict to the space $\mathbb{N}$ is because that we have no relationship like this at the $\infty$ since that is a function field. But we can then take the corresponding compactification in the following sense. We call the above equivalence open prismatization equivalence $\mathrm{ope}_\mathbb{N}$:
\begin{align}
\mathrm{Der}_{\mathcal{F}|_{\mathbb{N}}}\overset{\sim}{\rightarrow} \projlim_{(L,A)/S}\overline{\mathrm{Der}}_{(t,L)}(A).
\end{align}
Then we define the following compact prismaization equivalence $\mathrm{cpe}$:
\begin{align}
\mathrm{Der}_{\mathcal{F}}\overset{\sim}{\rightarrow} \widetilde{\projlim_{(L,A)/S}\overline{\mathrm{Der}}_{(t,L)}(A)}.
\end{align}
Here the right hand side is the $\infty$-category which is the fiber product of $\mathrm{Der}_{\mathcal{F}}$ and $\projlim_{(L,A)}\overline{\mathrm{Der}}_{(t,L)}(A)$ along the map to the restriction over $\mathbb{N}$. 
\end{definition}

\begin{definition}
We have the definition in the Nygaard filtered situation. Now here we restrict to the space $\mathbb{N}$ is because that we have no relationship like this at the $\infty$ since that is a function field. But we can then take the corresponding compactification in the following sense. We call the above equivalence open prismatization equivalence $\mathrm{ope}_\mathbb{N}$:
\begin{align}
\mathrm{Der}_{\mathcal{F}|_{\mathbb{N}}}\overset{\sim}{\rightarrow} \projlim_{(L,A,\mathrm{NFil})/S}\overline{\mathrm{Der}}_{(t,L)}(A).
\end{align}
Then we define the following compact prismaization equivalence $\mathrm{cpe}$:
\begin{align}
\mathrm{Der}_{\mathcal{F}}\overset{\sim}{\rightarrow} \widetilde{\projlim_{(L,A,\mathrm{NFil})/S}\overline{\mathrm{Der}}_{(t,L)}(A)}.
\end{align}
Here the right hand side is the $\infty$-category which is the fiber product of $\mathrm{Der}_{\mathcal{F}}$ and $\projlim_{(L,A)}\overline{\mathrm{Der}}_{(t,L)}(A)$ along the map to the restriction over $\mathbb{N}$. 
\end{definition}

\begin{proposition}
We have a natural isomorphism between the following two formal stacks. The first one is the Cartier-Witt stack defined over the $\mathbb{N}^\wedge$ The second one is the compactification along $\mathbb{N}\hookrightarrow \mathbb{N}^\wedge$ of the Cartier-Witt stack over $\mathbb{N}$ after the restriction for instance.
\end{proposition}

\begin{proof}
In fact we only have to consider what happens when we evaluate both stacks at $\infty$. In such a situation the first stack goes to be exactly the definition we have for the function field. Then for the corresponding compactification onto $\mathbb{N}^\wedge$, along the lifting of the ramification degrees, the corresponding distinguished element participate into the Witt wector to define each finite $n$ prismatization will eventual give rise to the need distinguished element as at $\infty$. This finishes the proof.
\end{proof}





\subsection{Absolute de Rham prismatization in families}



By restricting to $\mathbb{N}$ of our prismatization from $\mathbb{N}^\wedge$ we have the following equivalence on the derived $\infty$-categories of complexes:
\begin{align}
\mathrm{Der}_{\mathcal{F}|_{\mathbb{N}}}\overset{\sim}{\rightarrow} \projlim_{(L,A)}\overline{\mathrm{Der}}_{(t,L)}(A)
\end{align}
where $(L,A)$ varies in the set of all the prisms in our setting. $\overline{\mathrm{Der}}$ means the completion in the derived sense with respect to $t$ and $L$ over the complexes on the right of this equivalence. Under this equivalence we define the corresponding de Rham prismatization to be the prismatization over $\mathbb{N}$ associated with $\projlim_{(L,A)}\overline{\mathrm{Der}}_{(t,L)}(\widehat{A[1/t]}^{L})$.


\begin{definition}
Now here we restrict to the space $\mathbb{N}$ is because that we have no relationship like this at the $\infty$ since that is a function field. But we can then take the corresponding compactification in the following sense. We call the above open de Rham prismatization $\mathrm{odrp}_\mathbb{N}$:
\begin{align}
\projlim_{(L,A)}\overline{\mathrm{Der}}_{(t,L)}(\widehat{A[1/t]}^{L}).
\end{align}
Then we define the following compact de Rham prismaization $\mathrm{cdrp}$:
\begin{align}
\widetilde{\projlim_{(L,A)}\overline{\mathrm{Der}}_{(t,L)}(\widehat{A[1/t]}^{L})}.
\end{align}
Here the write hand side is the $\infty$-category which is the fiber product of $\mathbb{N}^\wedge$ and $\projlim_{(L,A)}\overline{\mathrm{Der}}_{(t,L)}(\widehat{A[1/t]}^{L})$ along the map to the restriction over $\mathbb{N}$. 
This compactification means the process defined in the following. First consider the corresponding functors away from the infinity:
\begin{align}
\mathrm{Der}_{\mathcal{F}|_{\mathbb{N}}}\overset{\sim}{\rightarrow} \projlim_{(L,A)}\overline{\mathrm{Der}}_{(t,L)}(A) \overset{}{\rightarrow} \projlim_{(L,A)}\overline{\mathrm{Der}}_{(t,L)}(\widehat{A[1/t]}^L)
\end{align}
Then we just take the corresponding fiber product along this map from $\mathrm{Der}_{\mathcal{F}|_{\mathbb{N}}}$ to the full stack over $\mathbb{N}^\wedge$.
\end{definition}

\begin{definition}
We have the Nygaard filtered version. Now here we restrict to the space $\mathbb{N}$ is because that we have no relationship like this at the $\infty$ since that is a function field. But we can then take the corresponding compactification in the following sense. We call the above open de Rham prismatization $\mathrm{odrp}_\mathbb{N}$:
\begin{align}
\projlim_{(L,A,\mathrm{NFil})}\overline{\mathrm{Der}}_{(t,L)}(\widehat{A[1/t]}^{L}).
\end{align}
Then we define the following compact de Rham prismaization $\mathrm{cdrp}$:
\begin{align}
\widetilde{\projlim_{(L,A,\mathrm{NFil})}\overline{\mathrm{Der}}_{(t,L)}(\widehat{A[1/t]}^{L})}.
\end{align}
Here the write hand side is the $\infty$-category which is the fiber product of $\mathbb{N}^\wedge$ and $\projlim_{(L,A)}\overline{\mathrm{Der}}_{(t,L)}(\widehat{A[1/t]}^{L})$ along the map to the restriction over $\mathbb{N}$. 
This compactification means the process defined in the following. First consider the corresponding functors away from the infinity:
\begin{align}
\mathrm{Der}_{\mathcal{F}|_{\mathbb{N}}}\overset{\sim}{\rightarrow} \projlim_{(L,A,\mathrm{NFil})}\overline{\mathrm{Der}}_{(t,L)}(A) \overset{}{\rightarrow} \projlim_{(L,A,\mathrm{NFil})}\overline{\mathrm{Der}}_{(t,L,\mathrm{NFil})}(\widehat{A[1/t]}^L)
\end{align}
Then we just take the corresponding fiber product along this map from $\mathrm{Der}_{\mathcal{F}|_{\mathbb{N}}}$ to the full stack over $\mathbb{N}^\wedge$.
\end{definition}


\begin{proposition}
This definition is equivalent to the following definition over $\mathbb{N}^\wedge$. For each $k$, We consider the Cartier-Witt Stack over $R$ then we consider those ideals vanishes through the projection $\mathrm{Witt}(\square)\rightarrow \square$ after taking the power $k$. Then we consider the project limit over $k$.
\end{proposition}

\begin{proof}
This is obviously true away from $\infty$ by taking the corresponding restriction to each finite $n\in \mathbb{N}$. Then we have the result since for some $n_0$ the construction on the Witt vectors mode some power $t^a$ will establish some identification of Witt vectors from $n_0$ to $\infty$ along our construction above. 
\end{proof}


\subsection{de Rham Prismatization for $R$-formal schemes}

\begin{setting}
We now consider a $R$-formal ring which is $t$-adic complete, which is denoted by $H$. We assume that $H$ is fibered over $\mathbb{N}^\wedge$.
\end{setting}


By restricting to $\mathbb{N}$ of our prismatization from $\mathbb{N}^\wedge$ we have the following equivalence on the derived $\infty$-categories of complexes:
\begin{align}
\mathrm{Der}_{\mathcal{F}|_{\mathbb{N}}}\overset{\sim}{\rightarrow} \projlim_{(L,A)/H}\overline{\mathrm{Der}}_{(t,L)}(A)
\end{align}
where $(L,A)/H$ varies in the set of all the prisms in our setting. $\overline{\mathrm{Der}}$ means the completion in the derived sense with respect to $t$ and $L$ over the complexes on the right of this equivalence. Under this equivalence we define the corresponding de Rham prismatization to be the prismatization over $\mathbb{N}$ associated with $\projlim_{(L,A)}\overline{\mathrm{Der}}_{(t,L)}(\widehat{A[1/t]}^{L})$.




\begin{definition}
Now here we restrict to the space $\mathbb{N}$ is because that we have no relationship like this at the $\infty$ since that is a function field. But we can then take the corresponding compactification in the following sense. We call the above open de Rham prismatization $\mathrm{odrp}_\mathbb{N}$:
\begin{align}
\projlim_{(L,A)/H}\overline{\mathrm{Der}}_{(t,L)}(\widehat{A[1/t]}^{L}).
\end{align}
Then we define the following compact de Rham prismaization $\mathrm{cdrp}$:
\begin{align}
\widetilde{\projlim_{(L,A)/H}\overline{\mathrm{Der}}_{(t,L)}(\widehat{A[1/t]}^{L})}.
\end{align}
Here the right hand side is the $\infty$-category which is the fiber product of $\mathbb{N}^\wedge$ and $\projlim_{(L,A)/H}\overline{\mathrm{Der}}_{(t,L)}(\widehat{A[1/t]}^{L})$ along the map to the restriction over $\mathbb{N}$.  
One then glues over $S$ by reducing to \cite{2BL}. This compactification means the process defined in the following. First consider the corresponding functors away from the infinity:
\begin{align}
\mathrm{Der}_{\mathcal{F}|_{\mathbb{N}}}\overset{\sim}{\rightarrow} \projlim_{(L,A)/H}\overline{\mathrm{Der}}_{(t,L)}(A) \overset{}{\rightarrow} \projlim_{(L,A)/H}\overline{\mathrm{Der}}_{(t,L)}(\widehat{A[1/t]}^L)
\end{align}
Then we just take the corresponding fiber product along this map from $\mathrm{Der}_{\mathcal{F}|_{\mathbb{N}}}$ to the full stack over $\mathbb{N}^\wedge$.
\end{definition}


\begin{definition}
When we have the Nygaard filtration we can define the following by setting all objects to carry such filtrations. Now here we restrict to the space $\mathbb{N}$ is because that we have no relationship like this at the $\infty$ since that is a function field. But we can then take the corresponding compactification in the following sense. We call the above open de Rham prismatization $\mathrm{odrp}_\mathbb{N}$:
\begin{align}
\projlim_{(L,A,\mathrm{NFil})/H}\overline{\mathrm{Der}}_{(t,L)}(\widehat{A[1/t]}^{L}).
\end{align}
Then we define the following compact de Rham prismaization $\mathrm{cdrp}$:
\begin{align}
\widetilde{\projlim_{(L,A,\mathrm{NFil})/H}\overline{\mathrm{Der}}_{(t,L)}(\widehat{A[1/t]}^{L})}.
\end{align}
Here the right hand side is the $\infty$-category which is the fiber product of $\mathbb{N}^\wedge$ and $\projlim_{(L,A)/H}\overline{\mathrm{Der}}_{(t,L)}(\widehat{A[1/t]}^{L})$ along the map to the restriction over $\mathbb{N}$. 
One then glues over $S$ by reducing to \cite{2BL}. This compactification means the process defined in the following. First consider the corresponding functors away from the infinity:
\begin{align}
\mathrm{Der}_{\mathcal{F}|_{\mathbb{N}}}\overset{\sim}{\rightarrow} \projlim_{(L,A,\mathrm{NFil})/H}\overline{\mathrm{Der}}_{(t,L)}(A) \overset{}{\rightarrow} \projlim_{(L,A,\mathrm{NFil})/H}\overline{\mathrm{Der}}_{(t,L)}(\widehat{A[1/t]}^L)
\end{align}
Then we just take the corresponding fiber product along this map from $\mathrm{Der}_{\mathcal{F}|_{\mathbb{N}}}$ to the full stack over $\mathbb{N}^\wedge$.
\end{definition}


\begin{remark}
The strategy which is sort of indirect here we use here for constructing the de Rham prismatization over the whole space $\mathbb{N}^\wedge$ is kind of compact method. The idea is sort of infinite induction by taking the compactification of the construction along the compactification:
\begin{align}
\mathbb{N} \hookrightarrow \mathbb{N}^\wedge.
\end{align} 
This is non-trivial in many situation, where the above discussion on the function field prismatization and finding direct relationship with function field prisms is a very typical example, where we can reach some results on the $\infty$ function field at the $\infty$ by taking limit of results over the compactification, especially when the function field objects are hard to study. Even if they are not hard to study this strategy can provide many direct convience for the constructions for function fields by using the results in mixed-characteristic situations.
\end{remark}






\subsection{Absolute Hodge-Tate prismatization in families}



By restricting to $\mathbb{N}$ of our prismatization from $\mathbb{N}^\wedge$ we have the following equivalence on the derived $\infty$-categories of complexes:
\begin{align}
\mathrm{Der}_{\mathcal{F}|_{\mathbb{N}}}\overset{\sim}{\rightarrow} \projlim_{(L,A)}\overline{\mathrm{Der}}_{(t,L)}(A)
\end{align}
where $(L,A)$ varies in the set of all the prisms in our setting. $\overline{\mathrm{Der}}$ means the completion in the derived sense with respect to $t$ and $L$ over the complexes on the right of this equivalence. Under this equivalence we define the corresponding de Rham prismatization to be the prismatization over $\mathbb{N}$ associated with $\projlim_{(L,A)}\overline{\mathrm{Der}}_{(t,L)}(\widehat{A[1/t]}^{L})$. Then we take the corresponding first degree in the graded of the de Rham completion in the definition, which we call them Hodge-Tate prismatization in family over $\mathbb{N}$.




\begin{definition}
Now here we restrict to the space $\mathbb{N}$ is because that we have no relationship like this at the $\infty$ since that is a function field. But we can then take the corresponding compactification in the following sense. We call the above open de Rham prismatization $\mathrm{odrp}_\mathbb{N}$:
\begin{align}
\projlim_{(L,A)}\overline{\mathrm{Der}}_{(t,L)}(\widehat{A[1/t]}^{L}).
\end{align}
Then we define the following compact de Rham prismaization $\mathrm{cdrp}$:
\begin{align}
\widetilde{\projlim_{(L,A)}\overline{\mathrm{Der}}_{(t,L)}(\widehat{A[1/t]}^{L})}.
\end{align}
Here the write hand side is the $\infty$-category which is the fiber product of $\mathbb{N}^\wedge$ and $\projlim_{(L,A)}\overline{\mathrm{Der}}_{(t,L)}(\widehat{A[1/t]}^{L})$ along the map to the restriction over $\mathbb{N}$. Then we take the corresponding first degree in the graded of the de Rham completion in the definition, which we call them Hodge-Tate prismatization in family over $\mathbb{N}^\wedge$. This is equivalent to the following definition: one can define this by first consider the subspace $\mathbb{N}$ where one can take the product throughout the whole $\mathbb{N}$ from the Hodge-Tate prismatization in the usual situation for each $n\in \mathbb{N}$, then take the fiber product along the following functors:
\begin{align}
\mathrm{Der}_{\mathcal{F}|_{\mathbb{N}}}\overset{\sim}{\rightarrow} \projlim_{(L,A)}\overline{\mathrm{Der}}_{(t,L)}(A) \overset{}{\rightarrow} \projlim_{(L,A)}\overline{\mathrm{Der}}_{(t,L)}(\widehat{A[1/t]}^L) \overset{}{\rightarrow} \projlim_{(L,A)}\overline{\mathrm{Der}}_{(t,L)}(\widehat{A[1/t]}/L).
\end{align} 
\end{definition}


\begin{proposition}
One can equivalently define the corresponding following Hodge-Tate prismatization in the following different way, i.e. just consider those substacks of $\mathrm{Cartier}_{W}$ parametrizing those Cartier-Witt ideals which are Hodge-Tate ones over $\mathbb{N}^\wedge$ as in \cite{2BL}. That is to say those ideals in the kernel of the map $\mathrm{Witt}(.)\rightarrow .$ in our current setting over $\mathbb{N}^\wedge$.
\end{proposition}

\begin{proof}
The reason for this to be true is that this is just the underlying stackification for the corresponding Hodge-Tate quasi-coherent complexes over the Hodge-Tate stacks in our setting. Our definition makes the set of ideals implicit. Since the such stack will be limit of the one from $\mathbb{N}$ where we do have the corresponding equivalence, then we are done for the same limiting process.
\end{proof}










\subsection{Hodge-Tate Prismatization for $R$-formal schemes}

\begin{setting}
We now consider a $R$-formal ring which is $t$-adic complete, which is denoted by $H$. We assume that $H$ is fibered over $\mathbb{N}^\wedge$.
\end{setting}


By restricting to $\mathbb{N}$ of our prismatization from $\mathbb{N}^\wedge$ we have the following equivalence on the derived $\infty$-categories of complexes:
\begin{align}
\mathrm{Der}_{\mathcal{F}|_{\mathbb{N}}}\overset{\sim}{\rightarrow} \projlim_{(L,A)/H}\overline{\mathrm{Der}}_{(t,L)}(A)
\end{align}
where $(L,A)/H$ varies in the set of all the prisms in our setting. $\overline{Der}$ means the completion in the derived sense with respect to $t$ and $L$ over the complexes on the right of this equivalence. Under this equivalence we define the corresponding de Rham prismatization to be the prismatization over $\mathbb{N}$ associated with $\projlim_{(L,A)}\overline{\mathrm{Der}}_{(t,L)}(\widehat{A[1/t]}^{L})$. Then we take the graded pieces and take the first degree one to define Hodge-Tate prismatization fibered over $\mathbb{N}^\wedge$. Note that we have the following functors by base change:
\begin{align}
\mathrm{Der}_{\mathcal{F}|_{\mathbb{N}}}\overset{\sim}{\rightarrow} \projlim_{(L,A)/H}\overline{\mathrm{Der}}_{(t,L)}(A) \overset{}{\rightarrow} \projlim_{(L,A)/H}\overline{\mathrm{Der}}_{(t,L)}(\widehat{A[1/t]}^L) \overset{}{\rightarrow} \projlim_{(L,A)/H}\overline{\mathrm{Der}}_{(t,L)}(\widehat{A[1/t]}/L).
\end{align}





\begin{definition}
Now here we restrict to the space $\mathbb{N}$ is because that we have no relationship like this at the $\infty$ since that is a function field. But we can then take the corresponding compactification in the following sense. We call the above open de Rham prismatization $\mathrm{odrp}_\mathbb{N}$:
\begin{align}
\projlim_{(L,A)/H}\overline{\mathrm{Der}}_{(t,L)}(\widehat{A[1/t]}^{L}).
\end{align}
Then we define the following compact de Rham prismaization $\mathrm{cdrp}$:
\begin{align}
\widetilde{\projlim_{(L,A)/H}\overline{\mathrm{Der}}_{(t,L)}(\widehat{A[1/t]}^{L})}.
\end{align}
Here the right hand side is the $\infty$-category which is the fiber product of $\mathbb{N}^\wedge$ and $\projlim_{(L,A)/H}\overline{\mathrm{Der}}_{(t,L)}(\widehat{A[1/t]}^{L})$ along the map to the restriction over $\mathbb{N}$. 
One then glues over $S$ by reducing to \cite{2BL}. Then we take the graded pieces and take the first degree one to define Hodge-Tate prismatization fibered over $\mathbb{N}^\wedge$. This is equivalent to the following definition: one can define this by first consider the subspace $\mathbb{N}$ where one can take the product throughout the whole $\mathbb{N}$ from the Hodge-Tate prismatization in the usual situation for each $n\in \mathbb{N}$, then take the fiber product along the following functors:
\begin{align}
\mathrm{Der}_{\mathcal{F}|_{\mathbb{N}}}\overset{\sim}{\rightarrow} \projlim_{(L,A)/H}\overline{\mathrm{Der}}_{(t,L)}(A) \overset{}{\rightarrow} \projlim_{(L,A)/H}\overline{\mathrm{Der}}_{(t,L)}(\widehat{A[1/t]}^L) \overset{}{\rightarrow} \projlim_{(L,A)/H}\overline{\mathrm{Der}}_{(t,L)}(\widehat{A[1/t]}/L).
\end{align}

\end{definition}



\subsection{$p$-adic Motives in families and $6$-functor formalism}

\noindent In this section we consider the motivic point of view after \cite{2G} and \cite{2A}. The goal is to first consider the motives in our setting over families and then apply to the $B^+_{\mathrm{dR},\mathbb{N}^\wedge}$-cohomology theory dated back to \cite{2F} and the motivic theories for function fields under the foundation we are considering. Followig \cite{2A} we will consider for each $n\in \mathbb{N}$ a corresponding $p$-adic motivic theory and the corresponding Weil sheaves of any $t$-adic formal schemes $\mathbb{Y}$ over $R = \prod R_i$:
\begin{align}
(\mathrm{MotiveTheory}_{n}, \mathrm{WeilSheaves}_n)_{\mathrm{Ayoub}, R_n,\mathbb{Y}}
\end{align}
corresponding to the following three prismatizations:
\begin{itemize}
\item[1] Prismatization for each $n$, which is a motivic theory in the sense of \cite{2A};
\item[2] de Rham prismatization for each $n$, which is a motivic theory in the sense of \cite{2A};
\item[3] Hodge-Tate prismatization for each $n$, which is a motivic theory in the sense of \cite{2A};
\item[4] Other type prismatizaton such as crystalline ones, Laurent ones and so on...
\subitem[4A]  $\mathrm{Der}_{\mathcal{F}|_{\mathbb{N}}}\overset{\sim}{\rightarrow} \projlim_{(L,A)}\overline{\mathrm{Der}}_{(t,L)}(A) \overset{}{\rightarrow} \projlim_{(L,A)}\overline{\mathrm{Der}}_{(t,L)}(\widehat{A[1/L]}^t)$ through compactification at $\infty$; 
\subitem[4B]  $\mathrm{Der}_{\mathcal{F}|_{\mathbb{N}}}\overset{\sim}{\rightarrow} \projlim_{(L,A)}\overline{\mathrm{Der}}_{(t,L)}(A) \overset{}{\rightarrow} \projlim_{(L,A)}\overline{\mathrm{Der}}_{(t,L)}({A[1/t]})$ through compactification at $\infty$;
\subitem[4C] ...
\item[5] Prismatization carrying filtration such as Nygaard filtrations.
\end{itemize}

\begin{definition} \mbox{\textbf{(Motivic Prismatization in Families)}}
In our setting we consider the following motivic theory in families over $K$ by taking the product:
\begin{align}
\prod_{n\in \mathbb{N}} (\mathrm{MotiveTheory}_{n}, \mathrm{WeilSheaves}_n)_{\mathrm{Ayoub}, R_n,\mathbb{Y}}
\end{align}
Then we consider the corresponding compactifiction along the corresponding compactification of the Cartier-Witt stacks in families to get:
\begin{align}
\widetilde{\prod_{n\in \mathbb{N}} (\mathrm{MotiveTheory}_{n}, \mathrm{WeilSheaves}_n)_{\mathrm{Ayoub}, R_n,\mathbb{Y}}}.
\end{align}
This is a motivic theory over $R$ in the sense of \cite{2A}. We use the notation:
\begin{align}
\mathrm{Hopf}_{\widetilde{\prod_{n\in \mathbb{N}} (\mathrm{MotiveTheory}_{n}, \mathrm{WeilSheaves}_n)_{\mathrm{Ayoub}, R_n,\mathbb{Y}}}}
\end{align}
to denote the corresponding Hopf algebra sheaf over $\mathbb{Y}$ for this motivic theory over $R$ in families. Then we define the motivic Galois group sheaf to be the spectrum of this big Hopf algebra sheaf:
\begin{align}
\mathrm{Gal}_{\widetilde{\prod_{n\in \mathbb{N}} (\mathrm{MotiveTheory}_{n}, \mathrm{WeilSheaves}_n)_{\mathrm{Ayoub}, R_n,\mathbb{Y}}}}:= \mathrm{Spec}(\mathrm{Hopf}_{\widetilde{\prod_{n\in \mathbb{N}} (\mathrm{MotiveTheory}_{n}, \mathrm{WeilSheaves}_n)_{\mathrm{Ayoub}, R_n,\mathbb{Y}}}}).
\end{align}
\end{definition}

\begin{definition}  \mbox{\textbf{(Motivic de Rham Prismatization in Families)}}
In our setting we consider the following motivic theory in families over $K$ by taking the product:
\begin{align}
\prod_{n\in \mathbb{N}} (\mathrm{dRMotiveTheory}_{n}, \mathrm{dRWeilSheaves}_n)_{\mathrm{Ayoub}, R_n,\mathbb{Y}}
\end{align}
Then we consider the corresponding compactifiction along the corresponding compactification of the Cartier-Witt stacks in families to get:
\begin{align}
\widetilde{\prod_{n\in \mathbb{N}} (\mathrm{dRMotiveTheory}_{n}, \mathrm{dRWeilSheaves}_n)_{\mathrm{Ayoub}, R_n,\mathbb{Y}}}.
\end{align}
This is a motivic theory over $R$ in the sense of \cite{2A}. We use the notation:
\begin{align}
\mathrm{Hopf}_{\widetilde{\prod_{n\in \mathbb{N}} (\mathrm{dRMotiveTheory}_{n}, \mathrm{dRWeilSheaves}_n)_{\mathrm{Ayoub}, R_n,\mathbb{Y}}}}
\end{align}
to denote the corresponding Hopf algebra sheaf for this motivic theory over $R$ in families. Then we define the motivic Galois group sheaf to be the spectrum of this big Hopf algebra sheaf:
\begin{align}
\mathrm{Gal}_{\widetilde{\prod_{n\in \mathbb{N}} (\mathrm{dRMotiveTheory}_{n}, \mathrm{dRWeilSheaves}_n)_{\mathrm{Ayoub}, R_n,\mathbb{Y}}}}:= \mathrm{Spec}(\mathrm{Hopf}_{\widetilde{\prod_{n\in \mathbb{N}} (\mathrm{dRMotiveTheory}_{n}, \mathrm{dRWeilSheaves}_n)_{\mathrm{Ayoub}, R_n,\mathbb{Y}}}}).
\end{align}
\end{definition}

\begin{definition}  \mbox{\textbf{(Motivic Hodge-Tate Prismatization in Families)}}
In our setting we consider the following motivic theory in families over $K$ by taking the product:
\begin{align}
\prod_{n\in \mathbb{N}} (\mathrm{HTMotiveTheory}_{n}, \mathrm{HTWeilSheaves}_n)_{\mathrm{Ayoub}, R_n,\mathbb{Y}}
\end{align}
Then we consider the corresponding compactifiction along the corresponding compactification of the Cartier-Witt stacks in families to get:
\begin{align}
\widetilde{\prod_{n\in \mathbb{N}} (\mathrm{HTMotiveTheory}_{n}, \mathrm{HTWeilSheaves}_n)_{\mathrm{Ayoub}, R_n,\mathbb{Y}}}.
\end{align}
This is a motivic theory over $R$ in the sense of \cite{2A}. We use the notation:
\begin{align}
\mathrm{Hopf}_{\widetilde{\prod_{n\in \mathbb{N}} (\mathrm{HTMotiveTheory}_{n}, \mathrm{HTWeilSheaves}_n)_{\mathrm{Ayoub}, R_n,\mathbb{Y}}}}
\end{align}
to denote the corresponding Hopf algebra sheaf for this motivic theory over $R$ in families. Then we define the motivic Galois group sheaf to be the spectrum of this big Hopf algebra sheaf:
\begin{align}
\mathrm{Gal}_{\widetilde{\prod_{n\in \mathbb{N}} (\mathrm{HTMotiveTheory}_{n}, \mathrm{HTWeilSheaves}_n)_{\mathrm{Ayoub}, R_n,\mathbb{Y}}}}:= \mathrm{Spec}(\mathrm{Hopf}_{\widetilde{\prod_{n\in \mathbb{N}} (\mathrm{HTMotiveTheory}_{n}, \mathrm{HTWeilSheaves}_n)_{\mathrm{Ayoub}, R_n,\mathbb{Y}}}}).
\end{align}
\end{definition}


\begin{theorem}
Let $X$ be a $v$-stack over $\mathrm{Spd}R$ in family over $\mathbb{N}^\wedge$. Then the corresponding $\mathbb{B}_{\mathrm{HT},\mathbb{N}^\wedge}^+$-cohomology theory and the corresponding $\mathbb{B}_{\mathrm{dR},\mathbb{N}^\wedge}^+$-cohomology theory are motivic theory in the sense of \cite{2A}. 
\end{theorem}

\begin{proof}
Over the $v$-site of $X$, we apply the constructions above, which implies consequence directly away from $\infty$, then our construction will then imply the result through the whole family by taking the compactification.
\end{proof}



\begin{theorem}
Let $X$ be a $v$-stack over $\mathrm{Spd}R$ in family over $\mathbb{N}^\wedge$. Then the corresponding $\mathbb{B}_{\mathrm{HT},\mathbb{N}^\wedge}^+$-cohomology theory and the corresponding $\mathbb{B}_{\mathrm{dR},\mathbb{N}^\wedge}^+$-cohomology theory are motivic theory in the sense of \cite{2A}. We have th $6$-functor formalism in this motivic setting over families. 
\end{theorem}

\begin{proof}
Over the $v$-site of $X$, we apply the constructions above, which implies consequence directly away from $\infty$, then our construction will then imply the result through the whole family by taking the compactification. Then the 6-functor formalism reduces to formal schemes (locally we then reduce the $v$-topology to arc topology after \cite{1S5}, \cite{1S6}), then reduces to schemes. $*$-adjoint pairs are the obvious ones, and $!$-adjoint pairs are those $!$-able morphisms for schemes, for instance one consider the inductive coherent sheaves in \cite{GRI}, \cite{GRII} where a full 6-functor formalism is established. Then foundation from \cite{2A} will apply in this setting.
\end{proof}


\subsection{Application to motivic $p$-adic Local Langlands in families} 

\noindent We now follow \cite{1S5}, \cite{1S6}, \cite{L1}, \cite{1FS}, \cite{2LH} to construct some generalized version of the local Langlands correspondence in \cite{1FS} by using the motivic construction we constructed above in the generalized setting. Recall the corresponding context in \cite{2LH} we have the corresponding $p$/$z$-adic group $G(K)$ for our ring in family $K$, this will provide the corresponding small arc stacks as in \cite{1S5}, \cite{1S6}. Since we have the motivic groups defined above, we can tranform a representation of the motivic group into the corresponding category on the other side. This process will define the following correponding functor.


\begin{theorem}
We have well-defined functor which is well-defined isomorphism:
\begin{align}
\mathrm{Repre}(?) \overset{\sim}{\rightarrow}  ! 
\end{align}
$?$ = $\mathrm{Gal}_{\widetilde{\prod_{n\in \mathbb{N}} (\mathrm{dRMotiveTheory}_{n}, \mathrm{dRWeilSheaves}_n)_{\mathrm{Ayoub}, R_n,\mathbb{Y}}}}$, $!$ = $\widetilde{\prod_{n\in \mathbb{N}} (\mathrm{dRMotiveTheory}_{n}, \mathrm{dRWeilSheaves}_n)_{\mathrm{Ayoub}, R_n,\mathbb{Y}}}$.
\end{theorem}


\begin{proof}
By motivic Hopf formalism in \cite{2A}.
\end{proof}




\indent Now we consider the moduli $v$-stacks in \cite{2LH}, we denote it by $Y_{\mathrm{FS},G,R}$ which is the $v$-stack of $G$-bundles for our family version local ring $K$ and $R$. We now consider the following following \cite{1FS}, \cite{1GL}. We actually relying on \cite{1FS}, \cite{1S5}, \cite{1S6}  can derive the Hecke operators:

\begin{definition}
Consider the map from the Hecke stack in \cite{2LH} which we denote that as $Y_{\mathrm{Hecke},G,R,I}$, and consider the map from this to fiber product of the Cartier stack $Y_{\mathrm{Cartier},R}$ with the corresponding stack $Y_{\mathrm{FS},G,R}$, and consider the map from this Hecke stack to the $Y_{\mathrm{FS},G,R}$. Pulling back along the second and push-forward the product with $\square_O$ will define the Hecke operator, where $\square_O$ is defined for some representation of the Langlands full-dual group in the coefficient $R=\prod R_n$. By result in \cite{1S5}, \cite{1S6} we have the construction does not depend on the choice of the primes, so we can in some equivalent way to derive a corresponding $R$-complex over the Hecke stacks with some finite set $I$. For instance after \cite{1FS} we have ${\mathbb{Q}}_\ell$-adic complex with $\ell$ away from $p$. Take any motivic sheaf in \cite{1S5}, \cite{1S6} with $\ell$-adic realization which is isomorphic to this complex, i.e. we can always work with $\mathbb{Z}$-algebra coefficients. Then we consider the $p$-adic realization which provides a corresponding $p$-adic complex. So in such a way we can first find a $\prod_{n\in \mathbb{N}}R_n$-complex over the Hecke stack by considering each fiber over $n\in \mathbb{N}$, then we can base change to $R$ to reach the function field at $\infty$. Then one can take the base change to the corresponding $\varprojlim_\alpha B^+_\mathrm{dR,\mathbb{N}^\wedge,\alpha}$-period ring to achieve finally an object in the category we are considering. This gives us desired $p$-adic complex over the Hecke stack, then one defines the corresponding morphisms from the Hecke stack for each finite set as above to $Y_{\mathrm{FS},G,R}$ and to $Y_{\mathrm{FS},G,R}\times Y'$. Here $Y'$ is defined to be the $v$-stack of all the solid quasicoherent sheaves over the de Rham stackifications in our family motivic setting, over the Cartier stack $Y_{\mathrm{Cartier},R}$ and those products of this Cartier stack. This will produce the desired Hecke operators.
\end{definition}


\begin{theorem}
The Hecke operator sends the complexes to those complexes carrying the action from the products of motivic Galois group (in the de Rham setting) of the Cartier stack for $K$ fibered over $\mathbb{N}^\wedge$.
\end{theorem}

\begin{proof}
By our definition we have that the corresponding image complexes are those complexes over the corresponding fiber product of $Y_{\mathrm{FS},G,R}$ with the corresponding classifying stack of the product of the motivic Galois group as in the statement of this theorem. For instance one can check this following the idea in \cite{1S5}, \cite{1S6} where we consider each totally disconneted subspace taking the form of the adic spectrum of some algebraically closed field. Over these algebraically closed geometric points we can see that we end up with purely perfect complexes of modules over:
\begin{align}
\varprojlim_\alpha B^+_\mathrm{dR,\mathbb{N}^\wedge,\alpha},
\end{align}
but we do have the corresponding lattices then, which reduces to the correponding $\overline{K}:=\prod_n \overline{K_n}^\wedge$-situation. Here the action of motivic de Rham Galois group for $K$ will then factors through the action of corresponding Weil group for $K$. Then we are in a situation parallel to \cite{2LH} and we only have to consider the action from the products of the Weil groups. Then the same proof as in \cite{1FS}, \cite{2LH} will derive the result stated. In fact there is nothing to prove here once one follows the same ideas in \cite{2LH}, in particular the proposition IX.1.1 in \cite{1FS}. Note that here by \cite{1S5}, \cite{1S6} for each $n\in \mathbb{N}^\wedge$ we can take the base change from $\mathbb{Z}$ to $R_n$ of the corresponding complexes on the smooth representation side.
\end{proof}


\begin{definition}
For quasi-split groups, fixing some $t$-adic character of the a chosen parabolic with the chosen unipotent radical, taking the induction from this radical of the character, one has the $t$-adic Whittaker sheaf $\mathrm{Whittaker}$, which can be regarded as a sheaf over the stack of $L$-parameter over the motivic Galois group in our current setting. Then we have from this theorem, as in \cite{1FS} the \textit{$t$-adic spectral action} from the perfect complexes over the stack of $L$-parameter\footnote{This stack is well-defined as in \cite{1FS}, where construction for any discrete group works.} of our motivic Galois group (discrete algebraic group scheme) to the corresponding derived $\infty$-category of the de Rham prismatized motivic sheaves over $Y_{\mathrm{FS},G,R}$. We denote this as the corresponding $\mathrm{SAction}_{\mathrm{Whittaker}}$. We then use the notation $D\mathrm{SAction}_{\mathrm{Whittaker}}$ to denote the derived $\infty$-categorical action induced as in \cite{HJ}. As in \cite{HJ}, we call the derived $\infty$-category of motivic sheaves in our setting in family:
\begin{align}
\mathrm{Repre}(\mathrm{Gal}_{\widetilde{\prod_{n\in \mathbb{N}} (\mathrm{dRMotiveTheory}_{n}, \mathrm{dRWeilSheaves}_n)_{\mathrm{Ayoub}, R_n,Y_{\mathrm{FS},G,R}}}})
\end{align}
an $\infty$-module over $D\mathrm{SAction}_{\mathrm{Whittaker}}$, in the higher categorical sense.
\end{definition}


\begin{theorem}
The $t$-adic motivic spectral action in our setting is well-defined. As in \cite{HJ}, we call the derived $\infty$-category of motivic sheaves in our setting in family:
\begin{align}
\mathrm{Repre}(\mathrm{Gal}_{\widetilde{\prod_{n\in \mathbb{N}} (\mathrm{dRMotiveTheory}_{n}, \mathrm{dRWeilSheaves}_n)_{\mathrm{Ayoub}, R_n,Y_{\mathrm{FS},G,R}}}})
\end{align}
an $\infty$-module over $D\mathrm{SAction}_{\mathrm{Whittaker}}$, in the higher categorical sense. This is also well-defined.
\end{theorem}



\begin{theorem}
One direction of the local Langlands in families holds true in this context: from Schur-irreducible objects to the corresponding $L$-parameters from the $?$ in our current context. Here $?$ is the motivic de Rham Galois group in familes in our setting over $\mathbb{N}^\wedge$. Note that all the coefficients on the both sides are $t$-adic. Each $t_n,n\in\mathbb{N}^\wedge-{\infty}$ comes from certain $p$-adic local fields, with $p$-fixed. 
\end{theorem}

\begin{proof}
By our construction we do have the mapping to the Bernstein centers in this current setting. Then as in VIII.4.1 and IV.4.1 of \cite{1FS} we can build up the corresponding mapping after \cite{1VL}. To be more precise for each finite set $I$ we can build up the corresponding symmetrical monoidal $\infty$-categories and the Hecke functors as in the above in our current setting, and we have the corresponding equivariant actions from the motivic Galois groups on the target symmetrical monoidal $\infty$-categories over the moduli $v$-stack. Then the excursion operators are generated automatically after \cite{1FS} and \cite{1VL}, where all these general abstract formalism will apply in our setting directly. This directly generalize the work \cite{2LH} as well. One follows the proof of VIII 4.1 in \cite{1FS} to derive the desired mapping in our current situation, with coefficient over $R=\prod_{n\in \mathbb{N}^\wedge} R_n$. The excursion operator space will then be the based change from $\mathbb{Z}$ to $R=\prod_{n\in \mathbb{N}^\wedge} R_n$ of the corresponding space over $\mathbb{Z}$ in \cite{1S5}, \cite{1S6}:
\begin{align}
\underline{S_{\mathrm{cocyc},G_\mathrm{dual},G_\mathrm{dual}-\mathrm{inv}}}\otimes_\mathbb{Z} \left(\prod_{n\in \mathbb{N}^\wedge} R_n\right).  
\end{align}
\end{proof}


\begin{remark}
This theorem considers not only the spaces in families (i.e. the corresponding moduli stacks of $G$-bundles in this setting is fibered over $\mathbb{N}^\wedge$) but also considers the corresponding motivic cohomology theory in families. For instance if the underlying ring is just a single local field, then there is no need to enlarge the motivic cohomological $\infty$-categories, since the motivic Galois group for this single motivic cohomological category will have then zero action on the other motivic cohomolgical components. On the other hand we consider also the motivic coefficient $\infty$-category to be a family version fibered over $\mathbb{N}^\wedge$. One may not have to do this but we choose to consider the family version of the motivic coefficient theory since the obvious reason from $p$-adic Hodge theory, i.e. if we consider Galois group $G_K$ of $K=\prod_{n\in \mathbb{N}^\wedge} K_n$, then the period rings which can be used to study the representation of $G_K$ have to be fibered over $\mathbb{N}^\wedge$, which includes the following: Robba rings in families, de Rham rings in families, cristalline rings in families. 
\end{remark}


\newpage
\section{Motivic Cohomology Theory for Prismatizations in Families}\label{section9}


\indent We now consider the corresponding prismatization in families in the three settings in \cite{3BL}, \cite{3D}. The first one is the prismatization in families, the second one is the filtration prismatization and finally we have the corresponding syntomization prismatization. They can be defined all in the corresponding families way as we did before. In the first two situations we recall our constrution before which will also be put into our current general consideration of motivic cohomology theories. As in \cite{3BSI} we have all the parallel definitions of $\delta$-rigns in function field situation which directly give rise to the \textit{prisms for function fields}.

\subsection{Prismatization, filtration prismatization and syntomization prismatization}


\begin{definition}\label{definition39}
We consider the corresponding prismatization in familes. Recall the definition goes in the following way. Eventually the definition is applied to $z$-adic $A$-formal schemes. In the absolute manner we consider the category of all $z$-nilpotent $A$-algebras as the underlying ring categorie $\mathrm{Nil}_{z,A}$, where all such algebras are assumed to be fibered over $\mathbb{N}_\infty$. Then we use the Witt vector functor $W_A$ from \cite{3LH}. Then the prismaization is the stackification over the ring category $\mathrm{Nil}_{z,A}$ where for each such ring we have the groupoid of all the Cartier-Witt ideals fibered over $\mathbb{N}_\infty$, i.e. those maps locally principally generated by distinguished elements in the big Witt vectors (for $w_0$ we require nilpotency and for $w_1$ we require unitality when we have the coordinates $(w_0,w_1,......)$)\footnote{

\begin{theorem}
Following \cite{3BL}, \cite{3D} one can also equivalently define Cartier-Witt ideals in families over $\mathbb{N}_\infty$ as the ideals in the following sense. Recall we have two maps on $W_A$: one is the map $W_A(\square)\rightarrow \square$, and the other one is the $\delta$-map for this generalized Witt vector functor (for instance see \cite{3BSI}), where we do have the $\delta$-ring structure for the Witt vector ring for function fields. Then one can define equivalently the Cartier-Witt ideals in families over $\mathbb{N}_\infty$ by giving requirement on the image ideals of Cartier ideals in families over $\mathbb{N}_\infty$: again for the second map we require the image ideals to have the property of being unit (unitality), and for the first map we require the image ideals to have the property of being nilpotent (nilpotency). Then this will lead to equivalent definitions in \cref{definition39} of all the prismatizations in families over $\mathbb{N}_\infty$, all the filtration prismatizations in families over $\mathbb{N}_\infty$, and all the syntomization prismatizations in families over $\mathbb{N}_\infty$.
\end{theorem}

\begin{proof}
After using general Witt vector $W_A$ we can see that this holds for all the fibers over $\mathbb{N}_\infty$, where in the function field situation we just use the corresponding $\delta$-ring structure in the function field situation (just replace $p$ by the uniformizater $z_\infty$, for instance see \cite{3BSI}). Then the results over these fibers will provide the result in the correspondig family over $\mathbb{N}_\infty$. Then the definition of the primatization in families over $\mathbb{N}_\infty$ can be given as in \cref{definition39}, equivalently.
\end{proof}

Then this will produce the equivalent definitions of all the \textit{prismatizations in families over $\mathbb{N}_\infty$/filtration prismatizations in families over $\mathbb{N}_\infty$/syntomization prismatizations in families over $\mathbb{N}_\infty$} and \textit{analytic prismatizations in families over $\mathbb{N}_\infty$/analytic filtration prismatizations in families over $\mathbb{N}_\infty$/analytic sytomization  prismatizations in families over $\mathbb{N}_\infty$} below in \cref{section9} and \cref{section10}, including those de Rham ones in families over $\mathbb{N}_\infty$, de Rham-Hodge-Tate ones in families over $\mathbb{N}_\infty$.

}. In this paper we use the notation 
\begin{align}
\underline{\mathrm{Prismatization}}_{\mathrm{abs},\mathbb{N}_\infty}
\end{align}
to denote the absolute prismatization in families in our current setting. Then we have the corresponding filtration prismatization:
\begin{align}
\overline{\underline{\mathrm{Prismatization}}}_{\mathrm{abs},\mathrm{Nygaard},\mathbb{N}}
\end{align}
which is defined to be compactification from $\mathbb{N}$ to $\mathbb{N}_\infty$ of the corresponding open version over $\mathbb{N}$:
\begin{align}
{\underline{\mathrm{Prismatization}}}_{\mathrm{abs},\mathrm{Nygaard},\mathbb{N}}
\end{align}
where this open version is defined to be taking the product over the finite fibers over $\mathbb{N}_\infty$ of the usual filtration prismatizations over $z_n$-nilpotent $A_n$-algebras. The compactification process needs to be using the following morphisms to reach our definition as the corresponding fiber product in families through the compactification:
\begin{align}
\prod_{n\in \mathbb{N}} {\underline{\mathrm{Prismatization}}}_{\mathrm{abs},\mathrm{Nygaard},n}\rightarrow  \prod_{n\in \mathbb{N}} {\underline{\mathrm{Prismatization}}}_{\mathrm{abs},n}
\end{align}
by taking the products along all $\mathbb{N}$ of the usual projection from the corresponding filtration prismatization to the prismatization, and:
\begin{align}
{\underline{\mathrm{Prismatization}}}_{\mathrm{abs},\mathbb{N}_\infty}\rightarrow  \prod_{n\in \mathbb{N}} {\underline{\mathrm{Prismatization}}}_{\mathrm{abs},n}.
\end{align}
Recall that the filtration prismatization is actually certain fibration version of the usual prismatization, then we take further fibration over $\mathbb{N}$. Recall how finally the syntomization prismatization is constructed, which is by taking the descent of the Hodge-Tate morphism and de Rham morphism in the filtration prismatization, along the diagonal morphism from the prismatization with itself. These maps are obviously admitting the corresponding compactification versions in our current setting. Therefore we consider the following three morphisms:
\begin{align}
f_\mathrm{dR}: {\underline{\mathrm{Prismatization}}}_{\mathrm{abs},\mathbb{N}_\infty}\rightarrow \overline{\underline{\mathrm{Prismatization}}}_{\mathrm{abs},\mathrm{Nygaard},\mathbb{N}},
\end{align}
\begin{align}
f_\mathrm{HT}: {\underline{\mathrm{Prismatization}}}_{\mathrm{abs},\mathbb{N}_\infty}\rightarrow \overline{\underline{\mathrm{Prismatization}}}_{\mathrm{abs},\mathrm{Nygaard},\mathbb{N}},
\end{align}
\begin{align}
d:  {\underline{\mathrm{Prismatization}}}_{\mathrm{abs},\mathbb{N}_\infty}\times {\underline{\mathrm{Prismatization}}}_{\mathrm{abs},\mathbb{N}_\infty} \rightarrow {\underline{\mathrm{Prismatization}}}_{\mathrm{abs},\mathbb{N}_\infty}.
\end{align}
All these three morphisms can be also constructed by taking the compactification from the morphism over $\mathbb{N}$ to $\mathbb{N}_\infty$. Then we take the product of $f_\mathrm{dR}$ and $f_\mathrm{HT}$ together and take the corresponding descent along $d$ we have the definition of the syntomization prismatization in families in our current setting over $\mathbb{N}_\infty$:
\begin{align}
\overline{\underline{\mathrm{Prismatization}}}_{\mathrm{abs},\mathrm{syntomization},\mathbb{N}}. 
\end{align}
This finishes the definition. 
\end{definition}


\begin{definition} 
We consider the corresponding prismatization in families. Recall the definition goes in the following way. Eventually the definition is applied to $z$-adic $A$-formal schemes. In the absolute manner we consider the category of all $z$-nilpotent $A$-algebras as the underlying ring categorie $\mathrm{Nil}_{z,A}$, where all such algebras are assumed to be fibered over $\mathbb{N}_\infty$. Then we use the Witt vector functor $W_A$ from \cite{3LH}. Then the prismaization is the stackification over the ring category $\mathrm{Nil}_{z,A}$ where for each such ring we have the groupoid of all the Cartier-Witt ideals fibered over $\mathbb{N}_\infty$, i.e. those maps locally principally generated by distinguished elements in the big Witt vectors (for $w_0$ we require nilpotency and for $w_1$ we require unitality). After the discussion above, we can now apply the whole definition for the 3 prismatization to any $z$-adic $A$-formal scheme $M$, by taking the immersion from the closure of the Cartier-Witt ideals into $M$ in the compatible way. In this paper we use the notation 
\begin{align}
\underline{\mathrm{Prismatization}}_{\mathrm{abs},\mathbb{N}_\infty,M}
\end{align}
to denote the absolute prismatization in families in our current setting. Then we have the corresponding filtration prismatization:
\begin{align}
\overline{\underline{\mathrm{Prismatization}}}_{\mathrm{abs},\mathrm{Nygaard},\mathbb{N},M}
\end{align}
which is defined to be compactification from $\mathbb{N}$ to $\mathbb{N}_\infty$ of the corresponding open version over $\mathbb{N}$:
\begin{align}
{\underline{\mathrm{Prismatization}}}_{\mathrm{abs},\mathrm{Nygaard},\mathbb{N},M}
\end{align}
where this open version is defined to be taking the product over the finite fibers over $\mathbb{N}_\infty$ of the usual filtration prismatizations over $z_n$-nilpotent $A_n$-algebras. The compactification process needs to be using the following morphisms to reach our definition as the corresponding fiber product in families through the compactification:
\begin{align}
\prod_{n\in \mathbb{N}} {\underline{\mathrm{Prismatization}}}_{\mathrm{abs},\mathrm{Nygaard},n,M}\rightarrow  \prod_{n\in \mathbb{N}} {\underline{\mathrm{Prismatization}}}_{\mathrm{abs},n,M}
\end{align}
by taking the products along all $\mathbb{N}$ of the usual projection from the corresponding filtration prismatization to the prismatization, and:
\begin{align}
{\underline{\mathrm{Prismatization}}}_{\mathrm{abs},\mathbb{N}_\infty,M}\rightarrow  \prod_{n\in \mathbb{N}} {\underline{\mathrm{Prismatization}}}_{\mathrm{abs},n,M}.
\end{align}
Recall that the filtration prismatization is actually certain fibration version of the usual prismatization, then we take further fibration over $\mathbb{N}$. Recall how finally the syntomization prismatization is constructed, which is by taking the descent of the Hodge-Tate morphism and de Rham morphism in the filtration prismatization, along the diagonal morphism from the prismatization with itself. These maps are obviously admitting the corresponding compactification versions in our current setting. Therefore we consider the following three morphisms:
\begin{align}
f_\mathrm{dR}: {\underline{\mathrm{Prismatization}}}_{\mathrm{abs},\mathbb{N}_\infty,M}\rightarrow \overline{\underline{\mathrm{Prismatization}}}_{\mathrm{abs},\mathrm{Nygaard},\mathbb{N},M},
\end{align}
\begin{align}
f_\mathrm{HT}: {\underline{\mathrm{Prismatization}}}_{\mathrm{abs},\mathbb{N}_\infty,M}\rightarrow \overline{\underline{\mathrm{Prismatization}}}_{\mathrm{abs},\mathrm{Nygaard},\mathbb{N},M},
\end{align}
\begin{align}
d:  {\underline{\mathrm{Prismatization}}}_{\mathrm{abs},\mathbb{N}_\infty,M}\times {\underline{\mathrm{Prismatization}}}_{\mathrm{abs},\mathbb{N}_\infty,M} \rightarrow {\underline{\mathrm{Prismatization}}}_{\mathrm{abs},\mathbb{N}_\infty,M}.
\end{align}
Then we take the product of $f_\mathrm{dR}$ and $f_\mathrm{HT}$ together and take the corresponding descent along $d$ we have the definition of the syntomization prismatization in families in our current setting over $\mathbb{N}_\infty$:
\begin{align}
\overline{\underline{\mathrm{Prismatization}}}_{\mathrm{abs},\mathrm{syntomization},\mathbb{N},M}.
\end{align}
\end{definition}



\begin{theorem}
There is a version of K\"unneth theorem in this context. To be more precise we have a K\"unneth theorem at least by passing to the corresponding quasicoherent sheaves over the corresponding prismatization for derived formal schemes (i.e. the corresponding stackifications). This holds for all three prismatization stackifications in the families over $\mathbb{N}_\infty$. 
\end{theorem}

\begin{proof}
We need to consider the compactification in this theorem. Any prismatization  stackification for a particular formal scheme $M$ is basically a stackification over $M$ and ringed we use the notation $\mathcal{P}$ to denote the structure sheaf. As in the usual situation we have that the derived $\infty$-category of all the $\mathcal{P}$-modules is equivalent to the prismatic site for $M$. However the derived prismatic cohomology theory in this setting does satisfy the derived K\"unneth theorem on the product of the corresponding $p$-adic formal ring locally (which can be glue over formal schemes like $M$). Then the result can be generalized to our setting over $\mathbb{N}$ away from $\infty$. Then we can take the corresponding compactification along:
\begin{align}
{\underline{\mathrm{Prismatization}}}_{\mathrm{abs},\mathbb{N}_\infty,M}\rightarrow  \prod_{n\in \mathbb{N}} {\underline{\mathrm{Prismatization}}}_{\mathrm{abs},n,M}.
\end{align}
The filtration prismatization and the syntomization prismatization have the same derived version of the K\"unneth theorems, which can be induced from the corresponding result for the prismatization. This finishes the proof. One can prove this directly by using the prisms in families over $\mathbb{N}_\infty$ to form the corresponding $z$-adic prismatic sites, where the $\mathrm{R}\Gamma$ will be quasi-isomorphic to the derived cohomology complex for $z$-adic prismatizations in families in our current consideration, where one just replaces $p$ by $z$. $R\Gamma$ for filtration prismatization in families is just derived base change of the prismatization in families over $\mathbb{N}_\infty$, then one considers the equilizers to reduce the proof for the syntomization prismatization to the the proof for the prismatization and the filtration prismatization.
 \end{proof}

\begin{theorem}
All three prismatization, filtration prismatization and syntomization prismatization over $\mathbb{N}_\infty$ satisfy our \cref{situation1} conditions: (A1), (A2), (A3), the formalism pullback and pushforward in the category of all the formal schemes in (A4), (A5).
\end{theorem}


\begin{proof}
We check this one by one. First for the first condition it is true after we consider the corresponding structure sheaves. The second condition is proved above. For the third condition we consider the corresponding derived $\infty$-categories of quasicoherent sheaves. A4 holds for pullbacks and pushforward. This is true over $\mathbb{N}$ then we consider the fiber product through:
\begin{align}
{\underline{\mathrm{Prismatization}}}_{\mathrm{abs},\mathbb{N}_\infty,M}\rightarrow  \prod_{n\in \mathbb{N}} {\underline{\mathrm{Prismatization}}}_{\mathrm{abs},n,M}
\end{align}
to reach the corresponding compactification. Finally A5 holds see \cite[Chapter 4, in particular 4.7, 4.8, 4.9, 4.10]{3A}. A5 in this setting is a stacky version where the base formal scheme will carry the stackifications over itself whose structure sheaf will provide the desired $\infty$-sheaf of ring after taking the $\infty$-level suspension, where we regard this base formal scheme as a small arc stack.
\end{proof}


\begin{definition}
We in this context use the notations:
\begin{align}
&\mathrm{Galois}(\underline{\mathrm{Prismatization}}_{\mathrm{abs},\mathbb{N}_\infty,M})\\
&\mathrm{Galois}(\overline{\underline{\mathrm{Prismatization}}}_{\mathrm{abs},\mathrm{Nygaard},\mathbb{N}_\infty,M})\\
&\mathrm{Galois}(\overline{\underline{\mathrm{Prismatization}}}_{\mathrm{abs},\mathbb{N}_\infty,\mathrm{syntomization},M})
\end{align}
to denote the corresponding Hopf algebraic motivic Galois fundamental groups, which is the defined to be the spectra of the associated Hopf algebra.
\end{definition}


\begin{theorem}
There are morphisms from these motivic Galois fundamental groups to the corresponding motivic Galois groups of $A$, and moreover $L$: 
\begin{align}
&\mathrm{Galois}(\underline{\mathrm{Prismatization}}_{\mathrm{abs},\mathbb{N}_\infty,M})\rightarrow \mathrm{Gal}(\overline{A}/A),\\
&\mathrm{Galois}(\overline{\underline{\mathrm{Prismatization}}}_{\mathrm{abs},\mathrm{Nygaard},\mathbb{N}_\infty,M})\rightarrow \mathrm{Gal}(\overline{A}/A),\\
&\mathrm{Galois}(\overline{\underline{\mathrm{Prismatization}}}_{\mathrm{abs},\mathbb{N}_\infty,\mathrm{syntomization},M})\rightarrow \mathrm{Gal}(\overline{A}/A).
\end{align}
\end{theorem}

\begin{proof}
This is formal since we just apply the construction and definition to the point situation (regard the scheme $\mathrm{Spec}A$ or $\mathrm{Spec}L$ as the corresponding small arc-stacks), then the theorem follows by the functoriality of the construction of Hopf algebras.
\end{proof}



\subsection{Three prismatizations in the de Rham setting}


\begin{definition}
We consider the corresponding de Rham prismatization in families. Recall the definition goes in the following way. Eventually the definition is applied to $z$-adic $A$-formal schemes. In the absolute manner we consider the category of all $z$-nilpotent $A$-algebras as the underlying ring categorie $\mathrm{Nil}_{z,A}$, where all such algebras are assumed to be fibered over $\mathbb{N}_\infty$. Then we use the Witt vector functor $W_A$ from \cite{3LH}. Then the prismaization is the stackification over the ring category $\mathrm{Nil}_{z,A}$ where for each such ring we have the groupoid of all the Cartier-Witt ideals fibered over $\mathbb{N}_\infty$, i.e. those maps locally principally generated by distinguished elements in the big Witt vectors (for $w_0$ we require nilpotency and for $w_1$ we require unitality). In this paper we use the notation 
\begin{align}
\underline{\mathrm{Prismatization}}_{\mathrm{abs},\mathrm{dR},\mathbb{N}_\infty}
\end{align}
to denote the absolute de Rham prismatization in families in our current setting. The definition for this is through the corresponding compactification from the stackification over $\mathbb{N}$, since then we have the morphism:
\begin{align}
\mathcal{P}_{\underline{\mathrm{Prismatization}}_{\mathrm{abs},\mathbb{N}_\infty}(\mathbb{N})} \overset{\sim}{\rightarrow} \varprojlim_{(Q_P,P)} D\mathrm{Cat}(P)^\mathrm{comp}\rightarrow \varprojlim_{(Q_P,P)} D\mathrm{Cat}(P[1/z]_{Q_P})^\mathrm{comp}
\end{align}
where $\mathrm{comp}$ is the completion with respect to the natural toplogy induced from the prisms involved along the inverse limit. Then we take the compactification to reach the de Rham prismatization in families $\mathbb{N}_\infty$:
\begin{align}
\mathcal{P}_{\underline{\mathrm{Prismatization}}_{\mathrm{abs},\mathbb{N}_\infty}} \overset{\sim}{\rightarrow} \overline{\varprojlim_{(Q_P,P)} D\mathrm{Cat}(P)^\mathrm{comp}}\rightarrow \overline{\varprojlim_{(Q_P,P)} D\mathrm{Cat}(P[1/z]_{Q_P})^\mathrm{comp}}.
\end{align}
Then we have the corresponding filtration de Rham  prismatization:
\begin{align}
\overline{\underline{\mathrm{Prismatization}}}_{\mathrm{abs},\mathrm{Nygaard},\mathrm{dR},\mathbb{N}}
\end{align}
which is defined to be compactification from $\mathbb{N}$ to $\mathbb{N}_\infty$ of the corresponding open version over $\mathbb{N}$:
\begin{align}
{\underline{\mathrm{Prismatization}}}_{\mathrm{abs},\mathrm{Nygaard},\mathrm{dR},\mathbb{N}}
\end{align}
where this open version is defined to be taking the product over the finite fibers over $\mathbb{N}_\infty$ of the usual filtration de Rham  prismatizations over $z_n$-nilpotent $A_n$-algebras. The compactification process needs to be using the following morphisms to reach our definition as the corresponding fiber product in families through the compactification:
\begin{align}
\prod_{n\in \mathbb{N}} {\underline{\mathrm{Prismatization}}}_{\mathrm{abs},\mathrm{Nygaard},\mathrm{dR},n}\rightarrow  \prod_{n\in \mathbb{N}} {\underline{\mathrm{Prismatization}}}_{\mathrm{abs},\mathrm{dR},n}
\end{align}
by taking the products along all $\mathbb{N}$ of the usual projection from the corresponding filtration de Rham prismatization to the de Rham prismatization, and:
\begin{align}
{\underline{\mathrm{Prismatization}}}_{\mathrm{abs},\mathrm{dR},\mathbb{N}_\infty}\rightarrow  \prod_{n\in \mathbb{N}} {\underline{\mathrm{Prismatization}}}_{\mathrm{abs},\mathrm{dR},n}.
\end{align}
Here the morphism
\begin{align}
\prod_{n\in \mathbb{N}} {\underline{\mathrm{Prismatization}}}_{\mathrm{abs},\mathrm{Nygaard},\mathrm{dR},n}\rightarrow  \prod_{n\in \mathbb{N}} {\underline{\mathrm{Prismatization}}}_{\mathrm{abs},\mathrm{dR},n}
\end{align}
is the base change of the morphism:
\begin{align}
\prod_{n\in \mathbb{N}} {\underline{\mathrm{Prismatization}}}_{\mathrm{abs},\mathrm{Nygaard},n}\rightarrow  \prod_{n\in \mathbb{N}} {\underline{\mathrm{Prismatization}}}_{\mathrm{abs},n}
\end{align}
along the following morphism defining the de Rham prismatizatio in families:
\begin{align}
\prod_{n\in \mathbb{N}} {\underline{\mathrm{Prismatization}}}_{\mathrm{abs},\mathrm{dR},n}\rightarrow  \prod_{n\in \mathbb{N}} {\underline{\mathrm{Prismatization}}}_{\mathrm{abs},n}.
\end{align}
Recall from \cite{3D} that the filtration de Rham prismatization is actually certain fibration version of the usual prismatization, then we take further fibration over $\mathbb{N}$. Recall how finally the syntomization de Rham prismatization is constructed, which is by taking the descent of the Hodge-Tate morphism and de Rham morphism in the filtration prismatization, along the diagonal morphism from the de Rham prismatization with itself. These maps are obviously admitting the corresponding compactification versions in our current setting. Therefore we consider the following three morphisms:
\begin{align}
f_\mathrm{dR}: {\underline{\mathrm{Prismatization}}}_{\mathrm{abs},\mathrm{dR},\mathbb{N}_\infty}\rightarrow \overline{\underline{\mathrm{Prismatization}}}_{\mathrm{abs},\mathrm{Nygaard},\mathrm{dR},\mathbb{N}_\infty},
\end{align}
\begin{align}
f_\mathrm{HT}: {\underline{\mathrm{Prismatization}}}_{\mathrm{abs},\mathrm{dR},\mathbb{N}_\infty}\rightarrow \overline{\underline{\mathrm{Prismatization}}}_{\mathrm{abs},\mathrm{Nygaard},\mathrm{dR},\mathbb{N}_\infty},
\end{align}
\begin{align}
d:  {\underline{\mathrm{Prismatization}}}_{\mathrm{abs},\mathrm{dR},\mathbb{N}_\infty}\times {\underline{\mathrm{Prismatization}}}_{\mathrm{abs},\mathrm{dR},\mathbb{N}_\infty} \rightarrow {\underline{\mathrm{Prismatization}}}_{\mathrm{abs},\mathrm{dR},\mathbb{N}_\infty}.
\end{align}
Then we take the product of $f_\mathrm{dR}$ and $f_\mathrm{HT}$ together and take the corresponding descent along $d$ we have the definition of the syntomization prismatization in families in our current setting over $\mathbb{N}_\infty$:
\begin{align}
\overline{\underline{\mathrm{Prismatization}}}_{\mathrm{abs},\mathrm{dR},\mathrm{syntomization},\mathbb{N}}. 
\end{align}
This finishes the definition. $\square$
\end{definition}

\begin{remark}
The de Rham prismatization, de Rham filtration prismatization, and de Rham syntomization prismatization can all be defined by taking limit along certain substacks of Cartier-Witt stacks, in families over $\mathbb{N}_\infty$, which is parametrized by integer $k$. 
\end{remark}

\begin{definition}
We consider the corresponding de Rham prismatization in families. Recall the definition goes in the following way. Eventually the definition is applied to $z$-adic $A$-formal schemes. In the absolute manner we consider the category of all $z$-nilpotent $A$-algebras as the underlying ring categorie $\mathrm{Nil}_{z,A}$, where all such algebras are assumed to be fibered over $\mathbb{N}_\infty$. Then we use the Witt vector functor $W_A$ from \cite{3LH}. Then the prismaization is the stackification over the ring category $\mathrm{Nil}_{z,A}$ where for each such ring we have the groupoid of all the Cartier-Witt ideals fibered over $\mathbb{N}_\infty$, i.e. those maps locally principally generated by distinguished elements in the big Witt vectors (for $w_0$ we require nilpotency and for $w_1$ we require unitality). Now map all the construction and definitions in the de Rham setting to $z$-adic $A$-formal scheme $M$. In this paper we use the notation 
\begin{align}
\underline{\mathrm{Prismatization}}_{\mathrm{abs},\mathrm{dR},\mathbb{N}_\infty,M}
\end{align}
to denote the absolute de Rham prismatization in families in our current setting. The definition for this is through the corresponding compactification from the stackification over $\mathbb{N}$, since then we have the morphism:
\begin{align}
\mathcal{P}_{\underline{\mathrm{Prismatization}}_{\mathrm{abs},\mathbb{N}_\infty,M}(\mathbb{N})} \overset{\sim}{\rightarrow} \varprojlim_{(Q_P,P)/M} D\mathrm{Cat}(P)^\mathrm{comp}\rightarrow \varprojlim_{(Q_P,P)/M} D\mathrm{Cat}(P[1/z]_{Q_P})^\mathrm{comp}
\end{align}
where $\mathrm{comp}$ is the completion with respect to the natural toplogy induced from the prisms involved along the inverse limit. Then we take the compactification to reach the de Rham prismatization in families $\mathbb{N}_\infty$:
\begin{align}
\mathcal{P}_{\underline{\mathrm{Prismatization}}_{\mathrm{abs},\mathbb{N}_\infty},M} \overset{\sim}{\rightarrow} \overline{\varprojlim_{(Q_P,P)/M} D\mathrm{Cat}(P)^\mathrm{comp}}\rightarrow \overline{\varprojlim_{(Q_P,P)/M} D\mathrm{Cat}(P[1/z]_{Q_P})^\mathrm{comp}}.
\end{align}
Then we have the corresponding filtration de Rham prismatization:
\begin{align}
\overline{\underline{\mathrm{Prismatization}}}_{\mathrm{abs},\mathrm{Nygaard},\mathrm{dR},\mathbb{N},M}
\end{align}
which is defined to be compactification from $\mathbb{N}$ to $\mathbb{N}_\infty$ of the corresponding open version over $\mathbb{N}$:
\begin{align}
{\underline{\mathrm{Prismatization}}}_{\mathrm{abs},\mathrm{Nygaard},\mathrm{dR},\mathbb{N},M}
\end{align}
where this open version is defined to be taking the product over the finite fibers over $\mathbb{N}_\infty$ of the usual filtration de Rham prismatizations over $z_n$-nilpotent $A_n$-algebras. The compactification process needs to be using the following morphisms to reach our definition as the corresponding fiber product in families through the compactification:
\begin{align}
\prod_{n\in \mathbb{N}} {\underline{\mathrm{Prismatization}}}_{\mathrm{abs},\mathrm{Nygaard},\mathrm{dR},n,M}\rightarrow  \prod_{n\in \mathbb{N}} {\underline{\mathrm{Prismatization}}}_{\mathrm{abs},\mathrm{dR},n,M}
\end{align}
by taking the products along all $\mathbb{N}$ of the usual projection from the corresponding filtration de Rham prismatization to the de Rham prismatization, and:
\begin{align}
{\underline{\mathrm{Prismatization}}}_{\mathrm{abs},\mathrm{dR},\mathbb{N}_\infty,M}\rightarrow  \prod_{n\in \mathbb{N}} {\underline{\mathrm{Prismatization}}}_{\mathrm{abs},\mathrm{dR},n,M}.
\end{align}
Recall that the filtration de Rham prismatization is actually certain fibration version of the usual de Rham prismatization, then we take further fibration over $\mathbb{N}$. Recall how finally the syntomization prismatization is constructed, which is by taking the descent of the Hodge-Tate morphism and de Rham morphism in the filtration prismatization, along the diagonal morphism from the prismatization with itself. These maps are obviously admitting the corresponding compactification versions in our current setting. Therefore we consider the following three morphisms:
\begin{align}
f_\mathrm{dR}: {\underline{\mathrm{Prismatization}}}_{\mathrm{abs},\mathrm{dR},\mathbb{N}_\infty,M}\rightarrow \overline{\underline{\mathrm{Prismatization}}}_{\mathrm{abs},\mathrm{Nygaard},\mathrm{dR},\mathbb{N}_\infty,M},
\end{align}
\begin{align}
f_\mathrm{HT}: {\underline{\mathrm{Prismatization}}}_{\mathrm{abs},\mathrm{dR},\mathbb{N}_\infty,M}\rightarrow \overline{\underline{\mathrm{Prismatization}}}_{\mathrm{abs},\mathrm{Nygaard},\mathrm{dR},\mathbb{N}_\infty,M},
\end{align}
\begin{align}
d:  {\underline{\mathrm{Prismatization}}}_{\mathrm{abs},\mathrm{dR},\mathbb{N}_\infty,M}\times {\underline{\mathrm{Prismatization}}}_{\mathrm{abs},\mathrm{dR},\mathbb{N}_\infty,M} \rightarrow {\underline{\mathrm{Prismatization}}}_{\mathrm{abs},\mathrm{dR},\mathbb{N}_\infty,M}.
\end{align}
Then we take the product of $f_\mathrm{dR}$ and $f_\mathrm{HT}$ together and take the corresponding descent along $d$ we have the definition of the syntomization de Rham prismatization in families in our current setting over $\mathbb{N}_\infty$:
\begin{align}
\overline{\underline{\mathrm{Prismatization}}}_{\mathrm{abs},\mathrm{dR},\mathrm{syntomization},\mathbb{N},M}. 
\end{align}
This finishes the definition. $\square$
\end{definition}

\begin{theorem}
There is a version of K\"unneth theorem in this context. To be more precise we have a K\"unneth theorem at least by passing to the corresponding quasicoherent sheaves over the corresponding prismatization for derived formal schemes (i.e. the corresponding stackifications). This holds for all three stackifications in the families over $\mathbb{N}_\infty$. 
\end{theorem}

\begin{proof}
We need to consider the compactification in this theorem. Any prismatization   stackification for a particular formal scheme $M$ is basically a stackification over $M$ and ringed we use the notation $\mathcal{P}$ to denote the structure sheaf. As in the usual situation we have that the derived $\infty$-category of all the $\mathcal{P}$-modules is equivalent to the prismatic site for $M$. However the derived prismatic cohomology theory in this setting does satisfy the K\"unneth theorem in the derived version on the product of the corresponding $p$-adic formal ring locally (which can be glue over formal schemes like $M$). Then the result can be generalized to our setting over $\mathbb{N}$ away from $\infty$. Then we can take the corresponding compactification along:
\begin{align}
{\underline{\mathrm{Prismatization}}}_{\mathrm{abs},\mathbb{N}_\infty,\mathrm{dR},M}\rightarrow  \prod_{n\in \mathbb{N}} {\underline{\mathrm{Prismatization}}}_{\mathrm{abs},n,\mathrm{dR},M}.
\end{align}
The filtration prismatization and the syntomization prismatization have the same derived version of the K\"unneth theorems, which can be induced from the corresponding result for the prismatization. In the current de Rham situation, we just consider the restriction to the de Rham prismatization. This finishes the proof. This amounts to saying that as in \cite{2BL2} one takes the base change to some prism $P$, which then reduces the proof for derived cohomology complexes for the three de Rham prismatizations to the proof for the derived base change of the derived cohomology complexes for three prismatizations to some de Rham prism from this prism $P$, i.e. $P[1/z]_{I_P}$. Then we just have the result from the result for the three prismatizations. Note that the three prismatizations are integral versions, but the three de Rham prismatizations can be derived from them through inverting $z$, or one can extract the integral versions of the de Rham prismatization directly first (that is to say without inverting $z$), then derive the result for the three integral de Rham prismatizations from the three integral prismatizations, then invert $z$.
\end{proof}

\begin{theorem}
All three prismatization, filtration prismatization and syntomization prismatization over $\mathbb{N}_\infty$ satisfy our \cref{situation1} conditions: (A1), (A2), (A3), the formalism pullback and pushforward in the category of all the formal schemes in (A4), (A5).
\end{theorem}


\begin{proof}
We check this one by one. First for the first condition it is true after we consider the corresponding structure sheaves. The second condition is proved above. For the third condition we consider the corresponding derived $\infty$-categories of quasicoherent sheaves. A4 holds for pullbacks and pushforward. This is true over $\mathbb{N}$ then we consider the fiber product through:
\begin{align}
{\underline{\mathrm{Prismatization}}}_{\mathrm{abs},\mathbb{N}_\infty,\mathrm{dR},M}\rightarrow  \prod_{n\in \mathbb{N}} {\underline{\mathrm{Prismatization}}}_{\mathrm{abs},n,\mathrm{dR},M}
\end{align}
to reach the corresponding compactification. Finally A5 holds see \cite[Chapter 4, in particular 4.7, 4.8, 4.9, 4.10]{3A}. For A5 we consider the big Grothendieck site over the base derived formal stack (regarded as a derived small arc stack) in families, with coverings from derived small arc stacks in families.
\end{proof}





\subsection{Three prismatization in the de Rham-Hodge-Tate setting}


\begin{definition}
We consider the corresponding de Rham-Hodge-Tate prismatization in families. Recall the definition goes in the following way. Eventually the definition is applied to $z$-adic $A$-formal schemes. In the absolute manner we consider the category of all $z$-nilpotent $A$-algebras as the underlying ring categorie $\mathrm{Nil}_{z,A}$, where all such algebras are assumed to be fibered over $\mathbb{N}_\infty$. Then we use the Witt vector functor $W_A$ from \cite{3LH}. Then the prismaization is the stackification over the ring category $\mathrm{Nil}_{z,A}$ where for each such ring we have the groupoid of all the Cartier-Witt ideals fibered over $\mathbb{N}_\infty$, i.e. those maps locally principally generated by distinguished elements in the big Witt vectors (for $w_0$ we require nilpotency and for $w_1$ we require unitality). In this paper we use the notation 
\begin{align}
\underline{\mathrm{Prismatization}}_{\mathrm{abs},\mathrm{dRHT},\mathbb{N}_\infty}
\end{align}
to denote the absolute de Rham-Hodge-Tate prismatization in families in our current setting. The definition for this is through the corresponding compactification from the stackification over $\mathbb{N}$, since then we have the morphism:
\begin{align}
\mathcal{P}_{\underline{\mathrm{Prismatization}}_{\mathrm{abs},\mathbb{N}_\infty}(\mathbb{N})} \overset{\sim}{\rightarrow} \varprojlim_{(Q_P,P)} D\mathrm{Cat}(P)^\mathrm{comp}\rightarrow \varprojlim_{(Q_P,P)} D\mathrm{Cat}(P[1/z]_{Q_P})^\mathrm{comp}\rightarrow \varprojlim_{(Q_P,P)} D\mathrm{Cat}(P[1/z]/{Q_P})^\mathrm{comp}
\end{align}
where $\mathrm{comp}$ is the completion with respect to the natural toplogy induced from the prisms involved along the inverse limit. Then we take the compactification to reach the de Rham-Hodge-Tate prismatization in families $\mathbb{N}_\infty$:
\begin{align}
\mathcal{P}_{\underline{\mathrm{Prismatization}}_{\mathrm{abs},\mathbb{N}_\infty}} \overset{\sim}{\rightarrow} \overline{\varprojlim_{(Q_P,P)} D\mathrm{Cat}(P)^\mathrm{comp}}\rightarrow \overline{\varprojlim_{(Q_P,P)} D\mathrm{Cat}(P[1/z]_{Q_P})^\mathrm{comp}}\rightarrow \overline{\varprojlim_{(Q_P,P)} D\mathrm{Cat}(P[1/z]/{Q_P})^\mathrm{comp}}.
\end{align}
Then we have the corresponding filtration de Rham-Hodge-Tate prismatization:
\begin{align}
\overline{\underline{\mathrm{Prismatization}}}_{\mathrm{abs},\mathrm{Nygaard},\mathrm{dRHT},\mathbb{N}}
\end{align}
which is defined to be compactification from $\mathbb{N}$ to $\mathbb{N}_\infty$ of the corresponding open version over $\mathbb{N}$:
\begin{align}
{\underline{\mathrm{Prismatization}}}_{\mathrm{abs},\mathrm{Nygaard},\mathrm{dRHT},\mathbb{N}}
\end{align}
where this open version is defined to be taking the product over the finite fibers over $\mathbb{N}_\infty$ of the usual filtration prismatizations over $z_n$-nilpotent $A_n$-algebras. The compactification process needs to be using the following morphisms to reach our definition as the corresponding fiber product in families through the compactification:
\begin{align}
\prod_{n\in \mathbb{N}} {\underline{\mathrm{Prismatization}}}_{\mathrm{abs},\mathrm{Nygaard},\mathrm{dRHT},n}\rightarrow  \prod_{n\in \mathbb{N}} {\underline{\mathrm{Prismatization}}}_{\mathrm{abs},\mathrm{dRHT},n}
\end{align}
by taking the products along all $\mathbb{N}$ of the usual projection from the corresponding filtration prismatization to the prismatization in the de Rham-Hodge-Tate situation, and:
\begin{align}
{\underline{\mathrm{Prismatization}}}_{\mathrm{abs},\mathrm{dRHT},\mathbb{N}_\infty}\rightarrow  \prod_{n\in \mathbb{N}} {\underline{\mathrm{Prismatization}}}_{\mathrm{abs},\mathrm{dRHT},n}.
\end{align}
Recall that the filtration prismatization is actually certain fibration version of the usual prismatization, then we take further fibration over $\mathbb{N}$, in the de Rham-Hodge-Tate situation. Recall how finally the syntomization prismatization is constructed, which is by taking the descent of the Hodge-Tate morphism and de Rham morphism in the filtration prismatization, along the diagonal morphism from the prismatization with itself. These maps are obviously admitting the corresponding compactification versions in our current setting. Therefore we consider the following three morphisms:
\begin{align}
f_\mathrm{dR}: {\underline{\mathrm{Prismatization}}}_{\mathrm{abs},\mathrm{dRHT},\mathbb{N}_\infty}\rightarrow \overline{\underline{\mathrm{Prismatization}}}_{\mathrm{abs},\mathrm{Nygaard},\mathrm{dRHT},\mathbb{N}_\infty},
\end{align}
\begin{align}
f_\mathrm{HT}: {\underline{\mathrm{Prismatization}}}_{\mathrm{abs},\mathrm{dRHT},\mathbb{N}_\infty}\rightarrow \overline{\underline{\mathrm{Prismatization}}}_{\mathrm{abs},\mathrm{Nygaard},\mathrm{dRHT},\mathbb{N}_\infty},
\end{align}
\begin{align}
d:  {\underline{\mathrm{Prismatization}}}_{\mathrm{abs},\mathrm{dRHT},\mathbb{N}_\infty}\times {\underline{\mathrm{Prismatization}}}_{\mathrm{abs},\mathrm{dRHT},\mathbb{N}_\infty} \rightarrow {\underline{\mathrm{Prismatization}}}_{\mathrm{abs},\mathrm{dRHT},\mathbb{N}_\infty}.
\end{align}
Then we take the product of $f_\mathrm{dR}$ and $f_\mathrm{HT}$ together and take the corresponding descent along $d$ we have the definition of the syntomization prismatization in families in our current setting over $\mathbb{N}_\infty$:
\begin{align}
\overline{\underline{\mathrm{Prismatization}}}_{\mathrm{abs},\mathrm{dRHT},\mathrm{syntomization},\mathbb{N}}. 
\end{align}
This finishes the definition. $\square$
\end{definition}

\begin{definition}
We consider the corresponding de Rham-Hodge-Tate prismatization in families. Recall the definition goes in the following way. Eventually the definition is applied to $z$-adic $A$-formal schemes. In the absolute manner we consider the category of all $z$-nilpotent $A$-algebras as the underlying ring categorie $\mathrm{Nil}_{z,A}$, where all such algebras are assumed to be fibered over $\mathbb{N}_\infty$. Then we use the Witt vector functor $W_A$ from \cite{3LH}. Then the prismaization is the stackification over the ring category $\mathrm{Nil}_{z,A}$ where for each such ring we have the groupoid of all the Cartier-Witt ideals fibered over $\mathbb{N}_\infty$, i.e. those maps locally principally generated by distinguished elements in the big Witt vectors (for $w_0$ we require nilpotency and for $w_1$ we require unitality). Now map all the construction and definitions in the de Rham-Hodge-Tate setting to $z$-adic $A$-formal scheme $M$. In this paper we use the notation 
\begin{align}
\underline{\mathrm{Prismatization}}_{\mathrm{abs},\mathrm{dRHT},\mathbb{N}_\infty,M}
\end{align}
to denote the absolute prismatization in families in our current setting. The definition for this is through the corresponding compactification from the stackification over $\mathbb{N}$, since then we have the morphism:
\begin{align}
&\mathcal{P}_{\underline{\mathrm{Prismatization}}_{\mathrm{abs},\mathbb{N}_\infty,M}(\mathbb{N})} \overset{\sim}{\rightarrow} \varprojlim_{(Q_P,P)/M} D\mathrm{Cat}(P)^\mathrm{comp}\\
&\rightarrow \varprojlim_{(Q_P,P)/M} D\mathrm{Cat}(P[1/z]/{Q_P})^\mathrm{comp}\rightarrow \varprojlim_{(Q_P,P)/M} D\mathrm{Cat}(P[1/z]/{Q_P})^\mathrm{comp}
\end{align}
where $\mathrm{comp}$ is the completion with respect to the natural toplogy induced from the prisms involved along the inverse limit. Then we take the compactification to reach the de Rham-Hodge-Tate prismatization in families $\mathbb{N}_\infty$:
\begin{align}
\mathcal{P}_{\underline{\mathrm{Prismatization}}_{\mathrm{abs},\mathbb{N}_\infty},M} \overset{\sim}{\rightarrow} \overline{\varprojlim_{(Q_P,P)/M} D\mathrm{Cat}(P)^\mathrm{comp}}
\rightarrow \overline{\varprojlim_{(Q_P,P)/M} D\mathrm{Cat}(P[1/z]/{Q_P})^\mathrm{comp}}\rightarrow \overline{\varprojlim_{(Q_P,P)/M} D\mathrm{Cat}(P[1/z]/{Q_P})^\mathrm{comp}}.
\end{align}
Then we have the corresponding filtration prismatization:
\begin{align}
\overline{\underline{\mathrm{Prismatization}}}_{\mathrm{abs},\mathrm{Nygaard},\mathrm{dRHT},\mathbb{N},M}
\end{align}
which is defined to be compactification from $\mathbb{N}$ to $\mathbb{N}_\infty$ of the corresponding open version over $\mathbb{N}$:
\begin{align}
{\underline{\mathrm{Prismatization}}}_{\mathrm{abs},\mathrm{Nygaard},\mathrm{dRHT},\mathbb{N},M}
\end{align}
where this open version is defined to be taking the product over the finite fibers over $\mathbb{N}_\infty$ of the usual filtration prismatizations over $z_n$-nilpotent $A_n$-algebras. The compactification process needs to be using the following morphisms to reach our definition as the corresponding fiber product in families through the compactification:
\begin{align}
\prod_{n\in \mathbb{N}} {\underline{\mathrm{Prismatization}}}_{\mathrm{abs},\mathrm{Nygaard},\mathrm{dRHT},n,M}\rightarrow  \prod_{n\in \mathbb{N}} {\underline{\mathrm{Prismatization}}}_{\mathrm{abs},\mathrm{dRHT},n,M}
\end{align}
by taking the products along all $\mathbb{N}$ of the usual projection from the corresponding filtration prismatization to the prismatization, and:
\begin{align}
{\underline{\mathrm{Prismatization}}}_{\mathrm{abs},\mathrm{dRHT},\mathbb{N}_\infty,M}\rightarrow  \prod_{n\in \mathbb{N}} {\underline{\mathrm{Prismatization}}}_{\mathrm{abs},\mathrm{dRHT},n,M}.
\end{align}
Recall that the filtration prismatization is actually certain fibration version of the usual prismatization in the de Rham-Hodge-Tate situation, then we take further fibration over $\mathbb{N}$. Recall how finally the syntomization prismatization is constructed, which is by taking the descent of the Hodge-Tate morphism and de Rham morphism in the filtration prismatization, along the diagonal morphism from the prismatization with itself. These maps are obviously admitting the corresponding compactification versions in our current setting. Therefore we consider the following three morphisms:
\begin{align}
f_\mathrm{dR}: {\underline{\mathrm{Prismatization}}}_{\mathrm{abs},\mathrm{dRHT},\mathbb{N}_\infty,M}\rightarrow \overline{\underline{\mathrm{Prismatization}}}_{\mathrm{abs},\mathrm{Nygaard},\mathrm{dRHT},\mathbb{N}_\infty,M},
\end{align}
\begin{align}
f_\mathrm{HT}: {\underline{\mathrm{Prismatization}}}_{\mathrm{abs},\mathrm{dRHT},\mathbb{N}_\infty,M}\rightarrow \overline{\underline{\mathrm{Prismatization}}}_{\mathrm{abs},\mathrm{Nygaard},\mathrm{dRHT},\mathbb{N}_\infty,M},
\end{align}
\begin{align}
d:  {\underline{\mathrm{Prismatization}}}_{\mathrm{abs},\mathrm{dRHT},\mathbb{N}_\infty,M}\times {\underline{\mathrm{Prismatization}}}_{\mathrm{abs},\mathrm{dRHT},\mathbb{N}_\infty,M} \rightarrow {\underline{\mathrm{Prismatization}}}_{\mathrm{abs},\mathrm{dRHT},\mathbb{N}_\infty,M}.
\end{align}
Then we take the product of $f_\mathrm{dR}$ and $f_\mathrm{HT}$ together and take the corresponding descent along $d$ we have the definition of the syntomization prismatization in families in our current setting over $\mathbb{N}_\infty$:
\begin{align}
\overline{\underline{\mathrm{Prismatization}}}_{\mathrm{abs},\mathrm{dRHT},\mathrm{syntomization},\mathbb{N},M}. 
\end{align}
This finishes the definition. $\square$
\end{definition}

\begin{theorem}
There is a version of K\"unneth theorem in this context. To be more precise we have a K\"unneth theorem at least by passing to the corresponding quasicoherent sheaves over the corresponding prismatization for derived formal schemes (i.e. the corresponding stackifications). This holds for all three stackifications in the families fibered over $\mathbb{N}_\infty$. 
\end{theorem}

\begin{proof}
We need to consider the compactification in this theorem. Any prismatization  stackification for a particular formal scheme $M$ is basically a stackification over $M$ and ringed we use the notation $\mathcal{P}$ to denote the structure sheaf. As in the usual situation we have that the derived $\infty$-category of all the $\mathcal{P}$-modules is equivalent to the prismatic site for $M$. However the derived prismatic cohomology theory in this setting does satisfy the K\"unneth theorem in the derived version on the product of the corresponding $p$-adic formal ring locally (which can be glue over formal schemes like $M$). Then the result can be generalized to our setting over $\mathbb{N}$ away from $\infty$. Then we can take the corresponding compactification along:
\begin{align}
{\underline{\mathrm{Prismatization}}}_{\mathrm{abs},\mathbb{N}_\infty,\mathrm{dRHT},M}\rightarrow  \prod_{n\in \mathbb{N}} {\underline{\mathrm{Prismatization}}}_{\mathrm{abs},n,\mathrm{dRHT},M}.
\end{align}
The derived version of the K\"unneth theorem then holds for the filtration  prismatization and syntomization prismatization. Then the derived K\"unneth holds in the 3 de Rham prismatizations situation which implies the result in the current setting. This finishes the proof.
\end{proof}

\begin{theorem}
All three prismatization, filtration prismatization and syntomization prismatization over $\mathbb{N}_\infty$ satisfy our \cref{situation1} conditions: (A1), (A2), (A3), the formalism pullback and pushforward in the category of all the formal schemes in (A4), (A5).
\end{theorem}


\begin{proof}
We check this one by one. First for the first condition it is true after we consider the corresponding structure sheaves. The second condition is proved above. For the third condition we consider the corresponding derived $\infty$-categories of quasicoherent sheaves. A4 holds for pullbacks and pushforward. This is true over $\mathbb{N}$ then we consider the fiber product through:
\begin{align}
{\underline{\mathrm{Prismatization}}}_{\mathrm{abs},\mathbb{N}_\infty,\mathrm{dRHT},M}\rightarrow  \prod_{n\in \mathbb{N}} {\underline{\mathrm{Prismatization}}}_{\mathrm{abs},n,\mathrm{dRHT},M}
\end{align}
to reach the corresponding compactification. Finally A5 holds see \cite[Chapter 4, in particular 4.7, 4.8, 4.9, 4.10]{3A}. For A5 we consider the derived small arc stacks over the base formal scheme (regarded as a small arc stack) to form the corresponding Grothendieck site in families.
\end{proof}








\subsection{Three prismatizations in the Laurent setting}

\begin{definition}
We consider the corresponding Laurent prismatization in families. Recall the definition goes in the following way. Eventually the definition is applied to $z$-adic $A$-formal schemes. In the absolute manner we consider the category of all $z$-nilpotent $A$-algebras as the underlying ring categorie $\mathrm{Nil}_{z,A}$, where all such algebras are assumed to be fibered over $\mathbb{N}_\infty$. Then we use the Witt vector functor $W_A$ from \cite{3LH}. Then the prismaization is the stackification over the ring category $\mathrm{Nil}_{z,A}$ where for each such ring we have the groupoid of all the Cartier-Witt ideals fibered over $\mathbb{N}_\infty$, i.e. those maps locally principally generated by distinguished elements in the big Witt vectors (for $w_0$ we require nilpotency and for $w_1$ we require unitality). In this paper we use the notation 
\begin{align}
\underline{\mathrm{Prismatization}}_{\mathrm{abs},\mathrm{Laurent},\mathbb{N}_\infty}
\end{align}
to denote the absolute prismatization in families in our current setting. The definition for this is through the corresponding compactification from the stackification over $\mathbb{N}$, since then we have the morphism:
\begin{align}
\mathcal{P}_{\underline{\mathrm{Prismatization}}_{\mathrm{abs},\mathbb{N}_\infty}(\mathbb{N})} \overset{\sim}{\rightarrow} \varprojlim_{(Q_P,P)} D\mathrm{Cat}(P)^\mathrm{comp}\rightarrow \varprojlim_{(Q_P,P)} D\mathrm{Cat}(P[1/Q_P]_z)^\mathrm{comp}
\end{align}
where $\mathrm{comp}$ is the completion with respect to the natural toplogy induced from the prisms involved along the inverse limit. Then we take the compactification to reach the Laurent prismatization in families $\mathbb{N}_\infty$:
\begin{align}
\mathcal{P}_{\underline{\mathrm{Prismatization}}_{\mathrm{abs},\mathbb{N}_\infty}} \overset{\sim}{\rightarrow} \overline{\varprojlim_{(Q_P,P)} D\mathrm{Cat}(P)^\mathrm{comp}}\rightarrow \overline{\varprojlim_{(Q_P,P)} D\mathrm{Cat}(P[1/Q_P]_z)^\mathrm{comp}}.
\end{align}
Then we have the corresponding filtration prismatization:
\begin{align}
\overline{\underline{\mathrm{Prismatization}}}_{\mathrm{abs},\mathrm{Nygaard},\mathrm{Laurent},\mathbb{N}}
\end{align}
which is defined to be compactification from $\mathbb{N}$ to $\mathbb{N}_\infty$ of the corresponding open version over $\mathbb{N}$:
\begin{align}
{\underline{\mathrm{Prismatization}}}_{\mathrm{abs},\mathrm{Nygaard},\mathrm{Laurent},\mathbb{N}}
\end{align}
where this open version is defined to be taking the product over the finite fibers over $\mathbb{N}_\infty$ of the usual filtration prismatizations over $z_n$-nilpotent $A_n$-algebras. The compactification process needs to be using the following morphisms to reach our definition as the corresponding fiber product in families through the compactification:
\begin{align}
\prod_{n\in \mathbb{N}} {\underline{\mathrm{Prismatization}}}_{\mathrm{abs},\mathrm{Nygaard},\mathrm{Laurent},n}\rightarrow  \prod_{n\in \mathbb{N}} {\underline{\mathrm{Prismatization}}}_{\mathrm{abs},\mathrm{Laurent},n}
\end{align}
by taking the products along all $\mathbb{N}$ of the usual projection from the corresponding filtration prismatization to the prismatization, and:
\begin{align}
{\underline{\mathrm{Prismatization}}}_{\mathrm{abs},\mathrm{Laurent},\mathbb{N}_\infty}\rightarrow  \prod_{n\in \mathbb{N}} {\underline{\mathrm{Prismatization}}}_{\mathrm{abs},\mathrm{Laurent},n}.
\end{align}
Recall that the filtration prismatization is actually certain fibration version of the usual prismatization in the current setting, then we take further fibration over $\mathbb{N}$. Recall how finally the syntomization prismatization is constructed, which is by taking the descent of the Hodge-Tate morphism and de Rham morphism in the filtration prismatization, along the diagonal morphism from the prismatization with itself. These maps are obviously admitting the corresponding compactification versions in our current setting. Therefore we consider the following three morphisms:
\begin{align}
f_\mathrm{dR}: {\underline{\mathrm{Prismatization}}}_{\mathrm{abs},\mathrm{Laurent},\mathbb{N}_\infty}\rightarrow \overline{\underline{\mathrm{Prismatization}}}_{\mathrm{abs},\mathrm{Nygaard},\mathrm{Laurent},\mathbb{N}_\infty},
\end{align}
\begin{align}
f_\mathrm{HT}: {\underline{\mathrm{Prismatization}}}_{\mathrm{abs},\mathrm{Laurent},\mathbb{N}_\infty}\rightarrow \overline{\underline{\mathrm{Prismatization}}}_{\mathrm{abs},\mathrm{Nygaard},\mathrm{Laurent},\mathbb{N}_\infty},
\end{align}
\begin{align}
d:  {\underline{\mathrm{Prismatization}}}_{\mathrm{abs},\mathrm{Laurent},\mathbb{N}_\infty}\times {\underline{\mathrm{Prismatization}}}_{\mathrm{abs},\mathrm{Laurent},\mathbb{N}_\infty} \rightarrow {\underline{\mathrm{Prismatization}}}_{\mathrm{abs},\mathrm{Laurent},\mathbb{N}_\infty}.
\end{align}
Then we take the product of $f_\mathrm{dR}$ and $f_\mathrm{HT}$ together and take the corresponding descent along $d$ we have the definition of the syntomization prismatization in families in our current setting over $\mathbb{N}_\infty$:
\begin{align}
\overline{\underline{\mathrm{Prismatization}}}_{\mathrm{abs},\mathrm{Laurent},\mathrm{syntomization},\mathbb{N}}. 
\end{align}
This finishes the definition. $\square$
\end{definition}

\begin{definition}
We consider the corresponding Laurent prismatization in families. Recall the definition goes in the following way. Eventually the definition is applied to $z$-adic $A$-formal schemes. In the absolute manner we consider the category of all $z$-nilpotent $A$-algebras as the underlying ring categorie $\mathrm{Nil}_{z,A}$, where all such algebras are assumed to be fibered over $\mathbb{N}_\infty$. Then we use the Witt vector functor $W_A$ from \cite{3LH}. Then the prismaization is the stackification over the ring category $\mathrm{Nil}_{z,A}$ where for each such ring we have the groupoid of all the Cartier-Witt ideals fibered over $\mathbb{N}_\infty$, i.e. those maps locally principally generated by distinguished elements in the big Witt vectors (for $w_0$ we require nilpotency and for $w_1$ we require unitality). Now map all the construction and definitions in the de Rham setting to $z$-adic $A$-formal scheme $M$. In this paper we use the notation 
\begin{align}
\underline{\mathrm{Prismatization}}_{\mathrm{abs},\mathrm{Laurent},\mathbb{N}_\infty,M}
\end{align}
to denote the absolute Laurent prismatization in families in our current setting. The definition for this is through the corresponding compactification from the stackification over $\mathbb{N}$, since then we have the morphism:
\begin{align}
\mathcal{P}_{\underline{\mathrm{Prismatization}}_{\mathrm{abs},\mathbb{N}_\infty,M}(\mathbb{N})} \overset{\sim}{\rightarrow} \varprojlim_{(Q_P,P)/M} D\mathrm{Cat}(P)^\mathrm{comp}\rightarrow \varprojlim_{(Q_P,P)/M} D\mathrm{Cat}(P[1/Q_P]_z)^\mathrm{comp}
\end{align}
where $\mathrm{comp}$ is the completion with respect to the natural toplogy induced from the prisms involved along the inverse limit. Then we take the compactification to reach the Laurent prismatization in families $\mathbb{N}_\infty$:
\begin{align}
\mathcal{P}_{\underline{\mathrm{Prismatization}}_{\mathrm{abs},\mathbb{N}_\infty},M} \overset{\sim}{\rightarrow} \overline{\varprojlim_{(Q_P,P)/M} D\mathrm{Cat}(P)^\mathrm{comp}}\rightarrow \overline{\varprojlim_{(Q_P,P)/M} D\mathrm{Cat}(P[1/Q_P]_z)^\mathrm{comp}}.
\end{align}
Then we have the corresponding filtration prismatization:
\begin{align}
\overline{\underline{\mathrm{Prismatization}}}_{\mathrm{abs},\mathrm{Nygaard},\mathrm{Laurent},\mathbb{N},M}
\end{align}
which is defined to be compactification from $\mathbb{N}$ to $\mathbb{N}_\infty$ of the corresponding open version over $\mathbb{N}$:
\begin{align}
{\underline{\mathrm{Prismatization}}}_{\mathrm{abs},\mathrm{Nygaard},\mathrm{Laurent},\mathbb{N},M}
\end{align}
where this open version is defined to be taking the product over the finite fibers over $\mathbb{N}_\infty$ of the usual filtration prismatizations over $z_n$-nilpotent $A_n$-algebras. The compactification process needs to be using the following morphisms to reach our definition as the corresponding fiber product in families through the compactification:
\begin{align}
\prod_{n\in \mathbb{N}} {\underline{\mathrm{Prismatization}}}_{\mathrm{abs},\mathrm{Nygaard},\mathrm{Laurent},n,M}\rightarrow  \prod_{n\in \mathbb{N}} {\underline{\mathrm{Prismatization}}}_{\mathrm{abs},\mathrm{Laurent},n,M}
\end{align}
by taking the products along all $\mathbb{N}$ of the usual projection from the corresponding filtration prismatization to the prismatization, and:
\begin{align}
{\underline{\mathrm{Prismatization}}}_{\mathrm{abs},\mathrm{Laurent},\mathbb{N}_\infty,M}\rightarrow  \prod_{n\in \mathbb{N}} {\underline{\mathrm{Prismatization}}}_{\mathrm{abs},\mathrm{Laurent},n,M}.
\end{align}
Recall that the filtration prismatization is actually certain fibration version of the usual prismatization, then we take further fibration over $\mathbb{N}$ in the current setting. Recall how finally the syntomization prismatization is constructed, which is by taking the descent of the Hodge-Tate morphism and de Rham morphism in the filtration prismatization, along the diagonal morphism from the prismatization with itself. These maps are obviously admitting the corresponding compactification versions in our current setting. Therefore we consider the following three morphisms:
\begin{align}
f_\mathrm{dR}: {\underline{\mathrm{Prismatization}}}_{\mathrm{abs},\mathrm{Laurent},\mathbb{N}_\infty,M}\rightarrow \overline{\underline{\mathrm{Prismatization}}}_{\mathrm{abs},\mathrm{Nygaard},\mathrm{Laurent},\mathbb{N}_\infty,M},
\end{align}
\begin{align}
f_\mathrm{HT}: {\underline{\mathrm{Prismatization}}}_{\mathrm{abs},\mathrm{Laurent},\mathbb{N}_\infty,M}\rightarrow \overline{\underline{\mathrm{Prismatization}}}_{\mathrm{abs},\mathrm{Nygaard},\mathrm{Laurent},\mathbb{N}_\infty,M},
\end{align}
\begin{align}
d:  {\underline{\mathrm{Prismatization}}}_{\mathrm{abs},\mathrm{Laurent},\mathbb{N}_\infty,M}\times {\underline{\mathrm{Prismatization}}}_{\mathrm{abs},\mathrm{Laurent},\mathbb{N}_\infty,M} \rightarrow {\underline{\mathrm{Prismatization}}}_{\mathrm{abs},\mathrm{Laurent},\mathbb{N}_\infty,M}.
\end{align}
Then we take the product of $f_\mathrm{dR}$ and $f_\mathrm{HT}$ together and take the corresponding descent along $d$ we have the definition of the syntomization prismatization in families in our current setting over $\mathbb{N}_\infty$:
\begin{align}
\overline{\underline{\mathrm{Prismatization}}}_{\mathrm{abs},\mathrm{Laurent},\mathrm{syntomization},\mathbb{N},M}. 
\end{align}
This finishes the definition. $\square$
\end{definition}

\begin{theorem}
There is a version of K\"unneth theorem in this context. To be more precise we have a K\"unneth theorem at least by passing to the corresponding quasicoherent sheaves over the corresponding prismatization for derived formal schemes (i.e. the corresponding stackifications). This holds for all three stackifications in the families over $\mathbb{N}_\infty$. 
\end{theorem}

\begin{proof}
We need to consider the compactification in this theorem. Any prismatization  stackification for a particular formal scheme $M$ is basically a stackification over $M$ and ringed we use the notation $\mathcal{P}$ to denote the structure sheaf. As in the usual situation we have that the derived $\infty$-category of all the $\mathcal{P}$-modules is equivalent to the prismatic site for $M$. However the derived prismatic cohomology theory in this setting does satisfy the K\"unneth theorem in the derived version on the product of the corresponding $p$-adic formal ring locally (which can be glue over formal schemes like $M$). Then the result can be generalized to our setting over $\mathbb{N}$ away from $\infty$. Then we can take the corresponding compactification along:
\begin{align}
{\underline{\mathrm{Prismatization}}}_{\mathrm{abs},\mathbb{N}_\infty,\mathrm{Laurent},M}\rightarrow  \prod_{n\in \mathbb{N}} {\underline{\mathrm{Prismatization}}}_{\mathrm{abs},n,\mathrm{Laurent},M}.
\end{align}
Then this implies derived version of K\"unneth theorem holds for filtration prismatization and syntomization prismatization. One then have the K\"unneth theorem in a derived version holds for the Laurent prismatizations in the three situations. This finishes the proof.
\end{proof}

\begin{theorem}
All three prismatization, filtration prismatization and syntomization prismatization over $\mathbb{N}_\infty$ satisfy our \cref{situation1} conditions: (A1), (A2), (A3), the formalism pullback and pushforward in the category of all the formal schemes in (A4), (A5).
\end{theorem}


\begin{proof}
We check this one by one. First for the first condition it is true after we consider the corresponding structure sheaves. The second condition is proved above. For the third condition we consider the corresponding derived $\infty$-categories of quasicoherent sheaves. A4 holds for pullbacks and pushforward. This is true over $\mathbb{N}$ then we consider the fiber product through:
\begin{align}
{\underline{\mathrm{Prismatization}}}_{\mathrm{abs},\mathbb{N}_\infty,\mathrm{Laurent},M}\rightarrow  \prod_{n\in \mathbb{N}} {\underline{\mathrm{Prismatization}}}_{\mathrm{abs},n,\mathrm{Laurent},M}
\end{align}
to reach the corresponding compactification. Finally A5 holds see \cite[Chapter 4, in particular 4.7, 4.8, 4.9, 4.10]{3A}.
\end{proof}







\newpage
\section{Motivic Cohomology Theory for Analytic Prismatizations in Families}\label{section10}


\indent We now consider the corresponding prismatization in families in the three settings in \cite{3BL}, \cite{3D}, with further analytification as in \cite{3ALBRCS} and \cite{3S1}. The idea is to construct there analytic versions of the prismatization in families $\mathbb{N}_\infty$, then apply our previous results above to derive the corresponding motivic cohomology theories. The first one is the prismatization in families, the second one is the filtration prismatization and finally we have the corresponding syntomization prismatization. They can be defined all in the corresponding families way as we did before. In the first two situations we recall our construtions before which will also be put into the general consideration we are current considering on motivic cohomology theories.

\subsection{Prismatization, filtration prismatization and syntomization prismatization}

\begin{definition}
We consider the corresponding prismatization in families. Recall the definition goes in the following way. Eventually the definition is applied to $z$-adic $A$-formal schemes. In the absolute manner we consider the category of all $z$-nilpotent $A$-algebras as the underlying ring categorie $\mathrm{Nil}_{z,A}$, where all such algebras are assumed to be fibered over $\mathbb{N}_\infty$. Then we use the Witt vector functor $W_A$ from \cite{3LH}. Then the prismaization is the stackification over the ring category $\mathrm{Nil}_{z,A}$ where for each such ring we have the groupoid of all the Cartier-Witt ideals fibered over $\mathbb{N}_\infty$, i.e. those maps locally principally generated by distinguished elements in the big Witt vectors (for $w_0$ we require nilpotency and for $w_1$ we require unitality). In this paper we use the notation 
\begin{align}
\underline{\mathrm{Prismatization}}^L_{\mathrm{abs},\mathbb{N}_\infty}
\end{align}
to denote the absolute analytic prismatization in families in our current setting. We follow \cite{3S1}, \cite{3ALBRCS} to the following three step for any $L$-rigid affioid with formal local model:
\begin{itemize}
\item[S1] Over $\mathbb{N}$ we have the analytic prismatization from local formal model from \cite{3S1}, \cite{3ALBRCS}, which provide the analytic prismatization over the generic fiber over $L$;
\item[S2] We take the corresponding compactification of this to $\mathbb{N}_\infty$, by using the morphism to take the fiber product:
\begin{align}
\underline{\mathrm{Prismatization}}_{\mathrm{abs},\mathbb{N}_\infty}\rightarrow \underline{\mathrm{Prismatization}}_{\mathrm{abs},\mathbb{N}_\infty}(\mathbb{N}).
\end{align}
\item[3S3] Take condensed analytification from \cite{3CS},\cite{3CS1},\cite{3CS2}.
\end{itemize}
Then we have the corresponding filtration analytic  prismatization:
\begin{align}
\overline{\underline{\mathrm{Prismatization}}}^L_{\mathrm{abs},\mathrm{Nygaard},\mathbb{N}}
\end{align}
which is defined to be compactification from $\mathbb{N}$ to $\mathbb{N}_\infty$ of the corresponding open version over $\mathbb{N}$:
\begin{align}
{\underline{\mathrm{Prismatization}}}^L_{\mathrm{abs},\mathrm{Nygaard},\mathbb{N}}
\end{align}
where this open version is defined to be taking the product over the finite fibers over $\mathbb{N}_\infty$ of the usual filtration prismatizations over $z_n$-nilpotent $A_n$-algebras. The compactification process needs to be using the following morphisms to reach our definition as the corresponding fiber product in families through the compactification:
\begin{align}
\prod_{n\in \mathbb{N}} {\underline{\mathrm{Prismatization}}}^L_{\mathrm{abs},\mathrm{Nygaard},n}\rightarrow  \prod_{n\in \mathbb{N}} {\underline{\mathrm{Prismatization}}}^L_{\mathrm{abs},n}
\end{align}
by taking the products along all $\mathbb{N}$ of the usual projection from the corresponding filtration analytic prismatization to the prismatization, and:
\begin{align}
{\underline{\mathrm{Prismatization}}}^L_{\mathrm{abs},\mathbb{N}_\infty}\rightarrow  \prod_{n\in \mathbb{N}} {\underline{\mathrm{Prismatization}}}^L_{\mathrm{abs},n}.
\end{align}
Recall that the filtration prismatization is actually certain fibration version of the usual prismatization, then we take further fibration over $\mathbb{N}$. Recall how finally the syntomization prismatization is constructed, which is by taking the descent of the Hodge-Tate morphism and de Rham morphism in the filtration prismatization, along the diagonal morphism from the prismatization with itself. These maps are obviously admitting the corresponding compactification versions in our current setting. Therefore we consider the following three morphisms:
\begin{align}
f_\mathrm{dR}: {\underline{\mathrm{Prismatization}}}^L_{\mathrm{abs},\mathbb{N}_\infty}\rightarrow \overline{\underline{\mathrm{Prismatization}}}^L_{\mathrm{abs},\mathrm{Nygaard},\mathbb{N}_\infty},
\end{align}
\begin{align}
f_\mathrm{HT}: {\underline{\mathrm{Prismatization}}}^L_{\mathrm{abs},\mathbb{N}_\infty}\rightarrow \overline{\underline{\mathrm{Prismatization}}}^L_{\mathrm{abs},\mathrm{Nygaard},\mathbb{N}_\infty},
\end{align}
\begin{align}
d:  {\underline{\mathrm{Prismatization}}}^L_{\mathrm{abs},\mathbb{N}_\infty}\times {\underline{\mathrm{Prismatization}}}^L_{\mathrm{abs},\mathbb{N}_\infty} \rightarrow {\underline{\mathrm{Prismatization}}}^L_{\mathrm{abs},\mathbb{N}_\infty}.
\end{align}
Then we take the product of $f_\mathrm{dR}$ and $f_\mathrm{HT}$ together and take the corresponding descent along $d$ we have the definition of the syntomization analytic prismatization in families in our current setting over $\mathbb{N}_\infty$:
\begin{align}
\overline{\underline{\mathrm{Prismatization}}}^L_{\mathrm{abs},\mathrm{syntomization},\mathbb{N}}. 
\end{align}
This finishes the definition. $\square$
\end{definition}


\begin{definition} 
We consider the corresponding prismatization in families. Recall the definition goes in the following way. Eventually the definition is applied to $z$-adic $A$-formal schemes. In the absolute manner we consider the category of all $z$-nilpotent $A$-algebras as the underlying ring categorie $\mathrm{Nil}_{z,A}$, where all such algebras are assumed to be fibered over $\mathbb{N}_\infty$. Then we use the Witt vector functor $W_A$ from \cite{3LH}. Then the prismaization is the stackification over the ring category $\mathrm{Nil}_{z,A}$ where for each such ring we have the groupoid of all the Cartier-Witt ideals fibered over $\mathbb{N}_\infty$, i.e. those maps locally principally generated by distinguished elements in the big Witt vectors (for $w_0$ we require nilpotency and for $w_1$ we require unitality). After the discussion above, we can now apply the whole definition for the 3 prismatization to any $z$-adic $A$-formal scheme $M$, by taking the immersion from the closure of the Cartier-Witt ideals into $M$ in the compatible way. In this paper we use the notation 
\begin{align}
\underline{\mathrm{Prismatization}}^L_{\mathrm{abs},\mathbb{N}_\infty,M}
\end{align}
to denote the absolute analytic prismatization in families in our current setting. Then we have the corresponding filtration analytic prismatization:
\begin{align}
\overline{\underline{\mathrm{Prismatization}}}^L_{\mathrm{abs},\mathrm{Nygaard},\mathbb{N},M}
\end{align}
which is defined to be compactification from $\mathbb{N}$ to $\mathbb{N}_\infty$ of the corresponding open version over $\mathbb{N}$:
\begin{align}
{\underline{\mathrm{Prismatization}}}^L_{\mathrm{abs},\mathrm{Nygaard},\mathbb{N},M}
\end{align}
where this open version is defined to be taking the product over the finite fibers over $\mathbb{N}_\infty$ of the usual filtration prismatizations over $z_n$-nilpotent $A_n$-algebras. The compactification process needs to be using the following morphisms to reach our definition as the corresponding fiber product in families through the compactification:
\begin{align}
\prod_{n\in \mathbb{N}} {\underline{\mathrm{Prismatization}}}^L_{\mathrm{abs},\mathrm{Nygaard},n,M}\rightarrow  \prod_{n\in \mathbb{N}} {\underline{\mathrm{Prismatization}}}^L_{\mathrm{abs},n,M}
\end{align}
by taking the products along all $\mathbb{N}$ of the usual projection from the corresponding filtration analytic prismatization to the analytic  prismatization, and:
\begin{align}
{\underline{\mathrm{Prismatization}}}^L_{\mathrm{abs},\mathbb{N}_\infty,M}\rightarrow  \prod_{n\in \mathbb{N}} {\underline{\mathrm{Prismatization}}}^L_{\mathrm{abs},n,M}.
\end{align}
Recall that the filtration analytic prismatization is actually certain fibration version of the usual analytic prismatization, then we take further fibration over $\mathbb{N}$. Recall how finally the syntomization prismatization is constructed, which is by taking the descent of the Hodge-Tate morphism and de Rham morphism in the filtration prismatization, along the diagonal morphism from the prismatization with itself. These maps are obviously admitting the corresponding compactification versions in our current setting. Therefore we consider the following three morphisms:
\begin{align}
f_\mathrm{dR}: {\underline{\mathrm{Prismatization}}}^L_{\mathrm{abs},\mathbb{N}_\infty,M}\rightarrow \overline{\underline{\mathrm{Prismatization}}}^L_{\mathrm{abs},\mathrm{Nygaard},\mathbb{N}_\infty,M},
\end{align}
\begin{align}
f_\mathrm{HT}: {\underline{\mathrm{Prismatization}}}^L_{\mathrm{abs},\mathbb{N}_\infty,M}\rightarrow \overline{\underline{\mathrm{Prismatization}}}^L_{\mathrm{abs},\mathrm{Nygaard},\mathbb{N}_\infty,M},
\end{align}
\begin{align}
d:  {\underline{\mathrm{Prismatization}}}^L_{\mathrm{abs},\mathbb{N}_\infty,M}\times {\underline{\mathrm{Prismatization}}}^L_{\mathrm{abs},\mathbb{N}_\infty,M} \rightarrow {\underline{\mathrm{Prismatization}}}^L_{\mathrm{abs},\mathbb{N}_\infty,M}.
\end{align}
Then we take the product of $f_\mathrm{dR}$ and $f_\mathrm{HT}$ together and take the corresponding descent along $d$ we have the definition of the syntomization analytic prismatization in families in our current setting over $\mathbb{N}_\infty$:
\begin{align}
\overline{\underline{\mathrm{Prismatization}}}^L_{\mathrm{abs},\mathrm{syntomization},\mathbb{N},M}.
\end{align}
\end{definition}


\begin{remark}
One can also define the three analytic prismatizations by taking the direct analytification of the family verion of the three prismatizations before as in \cite{3S1}, \cite{3ALBRCS}.
\end{remark}


\begin{theorem}
There is a version of K\"unneth theorem in this context. To be more precise we have a K\"unneth theorem at least by passing to the corresponding quasicoherent sheaves over the corresponding prismatization for derived rigid analytic spaces in families (i.e. the corresponding stackifications). This holds for all three stackifications in the families over $\mathbb{N}_\infty$. 
\end{theorem}




\begin{proof}
One considers formal models in this setting. We need to consider the compactification in this theorem. Any prismatization stackification for a particular formal scheme $M$ is basically a stackification over $M$ and ringed we use the notation $\mathcal{P}$ to denote the structure sheaf. As in the usual situation we have that the derived $\infty$-category of all the $\mathcal{P}$-modules is equivalent to the prismatic site for $M$. However the derived prismatic cohomology theory in this setting does satisfy the K\"unneth theorem in the derived version on the product of the corresponding $p$-adic formal ring locally (which can be glue over formal schemes like $M$). Then the result can be generalized to our setting over $\mathbb{N}$ away from $\infty$. Then we can take the corresponding compactification along:
\begin{align}
{\underline{\mathrm{Prismatization}}}^L_{\mathrm{abs},\mathbb{N}_\infty,M}\rightarrow  \prod_{n\in \mathbb{N}} {\underline{\mathrm{Prismatization}}}^L_{\mathrm{abs},n,M}.
\end{align}
This implies the derived version of K\"unneth theorem holds for analytic filtration prismatization and anlytic syntomization prismatization. This finishes the proof.
\end{proof}

\begin{theorem}
All three analytic prismatization, analytic filtration prismatization and analytic syntomization prismatization over $\mathbb{N}_\infty$ satisfy our \cref{situation2} conditions: (A1), (A2), (A3), the formalism pullback and pushforward in (A4), (A5).
\end{theorem}


\begin{proof}
We check this one by one. First for the first condition it is true after we consider the corresponding structure sheaves. The second condition is proved above. For the third condition we consider the corresponding derived $\infty$-categories of quasicoherent sheaves. A4 holds for pullbacks and pushforward. This is true over $\mathbb{N}$ then we consider the fiber product through:
\begin{align}
{\underline{\mathrm{Prismatization}}}^L_{\mathrm{abs},\mathbb{N}_\infty,M}\rightarrow  \prod_{n\in \mathbb{N}} {\underline{\mathrm{Prismatization}}}^L_{\mathrm{abs},n,M}
\end{align}
to reach the corresponding compactification. Finally A5 holds see \cite[Chapter 4, in particular 4.7, 4.8, 4.9, 4.10]{3A}. However since we are talking about rigid analytic spaces, \cite{3A} can be directly applied, where one can derive all the results from the formalism in \cite{3A}, which allows us to reduce the corresponding proof to the formal scheme situation before. 
\end{proof}


\begin{definition}
We in this context use the notations:
\begin{align}
&\mathrm{Galois}(\underline{\mathrm{Prismatization}}^L_{\mathrm{abs},\mathbb{N}_\infty,M})\\
&\mathrm{Galois}(\overline{\underline{\mathrm{Prismatization}}}^L_{\mathrm{abs},\mathrm{Nygaard},\mathbb{N}_\infty,M})\\
&\mathrm{Galois}(\overline{\underline{\mathrm{Prismatization}}}^L_{\mathrm{abs},\mathbb{N}_\infty,\mathrm{syntomization},M})
\end{align}
to denote the corresponding Hopf algebraic motivic Galois fundamental groups, which is the defined to be the spectra of the associated Hopf algebra.
\end{definition}


\begin{theorem}
There are morphisms from these motivic Galois fundamental groups to the corresponding motivic Galois groups of $A$, and moreover $L$: 
\begin{align}
&\mathrm{Galois}(\underline{\mathrm{Prismatization}}^L_{\mathrm{abs},\mathbb{N}_\infty,M})\rightarrow \mathrm{Gal}(\overline{A}/A),\\
&\mathrm{Galois}(\overline{\underline{\mathrm{Prismatization}}}^L_{\mathrm{abs},\mathrm{Nygaard},\mathbb{N}_\infty,M})\rightarrow \mathrm{Gal}(\overline{A}/A),\\
&\mathrm{Galois}(\overline{\underline{\mathrm{Prismatization}}}^L_{\mathrm{abs},\mathbb{N}_\infty,\mathrm{syntomization},M})\rightarrow \mathrm{Gal}(\overline{A}/A).
\end{align}
\end{theorem}

\begin{proof}
This is formal since we just apply the construction and definition to the point situation (regard the scheme $\mathrm{Spec}A$ or $\mathrm{Spec}L$ as the corresponding formal scheme or rigid analytic space), then the theorem follows by the functoriality of the construction of Hopf algebras. Then we reduce to the results before for formal local model before following \cite{3A}. 
\end{proof}


\begin{definition}
The three analytic prismatizations:
\begin{align}
&\underline{\mathrm{Prismatization}}^L_{\mathrm{abs},\mathbb{N}_\infty,M}\\
&\overline{\underline{\mathrm{Prismatization}}}^L_{\mathrm{abs},\mathrm{Nygaard},\mathbb{N}_\infty,M}\\
&\overline{\underline{\mathrm{Prismatization}}}^L_{\mathrm{abs},\mathbb{N}_\infty,\mathrm{syntomization},M}
\end{align}
have a small arc-stack version as well as a small v-stack version, where the condensed structure sheaves are pull-back along the strictly totally disconnected coverings. For such switching of categories and consideration, we use the then the following different notation:
\begin{align}
&\underline{\mathrm{Prismatization}}^L_{\mathrm{abs},\mathbb{N}_\infty,-}\\
&\overline{\underline{\mathrm{Prismatization}}}^L_{\mathrm{abs},\mathrm{Nygaard},\mathbb{N}_\infty,-}\\
&\overline{\underline{\mathrm{Prismatization}}}^L_{\mathrm{abs},\mathbb{N}_\infty,\mathrm{syntomization},-}
\end{align}
where $-$ is a small arc stack or a small v-stack attached to a rigid analytic space over $L$ over $\mathbb{N}_\infty$, however locally the topology is arc-topology or $v$-topology.
\end{definition}

\begin{theorem}
There is a version of K\"unneth theorem in this context
\begin{align}
&\underline{\mathrm{Prismatization}}^L_{\mathrm{abs},\mathbb{N}_\infty,-}\\
&\overline{\underline{\mathrm{Prismatization}}}^L_{\mathrm{abs},\mathrm{Nygaard},\mathbb{N}_\infty,-}\\
&\overline{\underline{\mathrm{Prismatization}}}^L_{\mathrm{abs},\mathbb{N}_\infty,\mathrm{syntomization},-}.
\end{align}
To be more precise we have a K\"unneth theorem at least by passing to the corresponding quasicoherent sheaves over the corresponding prismatization (i.e. the corresponding stackifications). This holds for all three stackifications in the families over $\mathbb{N}_\infty$. 
\end{theorem}

\begin{proof}
We have to consider formal models in this setting. We need to consider the compactification in this theorem. Any prismatization stackification for a particular formal scheme $-$ is basically a stackification over $-$ and ringed we use the notation $\mathcal{P}$ to denote the structure sheaf. As in the usual situation we have that the derived $\infty$-category of all the $\mathcal{P}$-modules is equivalent to the prismatic site for $-$. However the derived prismatic cohomology theory in this setting does satisfy the K\"unneth theorem in the derived version on the product of the corresponding $p$-adic formal ring locally (which can be glue over formal schemes like $-$). Then the result can be generalized to our setting over $\mathbb{N}$ away from $\infty$. Then we can take the corresponding compactification along:
\begin{align}
{\underline{\mathrm{Prismatization}}}^L_{\mathrm{abs},\mathbb{N}_\infty,-}\rightarrow  \prod_{n\in \mathbb{N}} {\underline{\mathrm{Prismatization}}}^L_{\mathrm{abs},n,-}.
\end{align}
This implies the derived version of the K\"unneth theorem holds for analytic filtration prismatization and analytic syntomization prismatization. This finishes the proof.
\end{proof}

\begin{theorem}
All three analytic prismatization, analytic filtration prismatization and analytic syntomization prismatization over $\mathbb{N}_\infty$ satisfy our \cref{situation1} conditions: (A1), (A2), (A3), the formalism pullback and pushforward in (A4), (A5).
\end{theorem}


\begin{proof}
We check this one by one. First for the first condition it is true after we consider the corresponding structure sheaves. The second condition is proved above. For the third condition we consider the corresponding derived $\infty$-categories of quasicoherent sheaves. A4 holds for pullbacks and pushforward. This is true over $\mathbb{N}$ then we consider the fiber product through:
\begin{align}
{\underline{\mathrm{Prismatization}}}^L_{\mathrm{abs},\mathbb{N}_\infty,-}\rightarrow  \prod_{n\in \mathbb{N}} {\underline{\mathrm{Prismatization}}}^L_{\mathrm{abs},n,-}
\end{align}
to reach the corresponding compactification. Finally A5 holds see \cite[Chapter 4, in particular 4.7, 4.8, 4.9, 4.10]{3A}. However since we are talking about rigid analytic spaces by using perfectoid coverings, \cite{3A} can be directly applied, where one can derive all the results from the formalism in \cite{3A}, which allows us to reduce the corresponding proof to the formal scheme situation before. Again we need to consider the generalized version of \cite{3A} along \cite{3S}, i.e. regard the corresponding rigid analytic motivic cohomology theories as ones in our \cref{situation1}.
\end{proof}


\begin{definition}
We in this context use the notations:
\begin{align}
&\mathrm{Galois}(\underline{\mathrm{Prismatization}}^L_{\mathrm{abs},\mathbb{N}_\infty,-})\\
&\mathrm{Galois}(\overline{\underline{\mathrm{Prismatization}}}^L_{\mathrm{abs},\mathrm{Nygaard},\mathbb{N}_\infty,-})\\
&\mathrm{Galois}(\overline{\underline{\mathrm{Prismatization}}}^L_{\mathrm{abs},\mathbb{N}_\infty,\mathrm{syntomization},-})
\end{align}
to denote the corresponding Hopf algebraic motivic Galois fundamental groups, which is the defined to be the spectra of the associated Hopf algebra.
\end{definition}


\begin{theorem}
There are morphisms from these motivic Galois fundamental groups to the corresponding motivic Galois groups of $A$, and moreover $L$: 
\begin{align}
&\mathrm{Galois}(\underline{\mathrm{Prismatization}}^L_{\mathrm{abs},\mathbb{N}_\infty,-})\rightarrow \mathrm{Gal}(\overline{L}/L),\\
&\mathrm{Galois}(\overline{\underline{\mathrm{Prismatization}}}^L_{\mathrm{abs},\mathrm{Nygaard},\mathbb{N}_\infty,-})\rightarrow \mathrm{Gal}(\overline{L}/L),\\
&\mathrm{Galois}(\overline{\underline{\mathrm{Prismatization}}}^L_{\mathrm{abs},\mathbb{N}_\infty,\mathrm{syntomization},-})\rightarrow \mathrm{Gal}(\overline{L}/L).
\end{align}
\end{theorem}

\begin{proof}
This is formal since we just apply the construction and definition to the point situation (regard the scheme $\mathrm{Spec}A$ or $\mathrm{Spec}L$ as the corresponding formal scheme or rigid analytic space), then the theorem follows by the functoriality of the construction of Hopf algebras. Then we reduce to the results before for formal local model before following \cite{3A}. 
\end{proof}


\subsection{Three analytic prismatizations in the de Rham setting}


\begin{definition}
We consider the corresponding de Rham prismatization in families. Recall the definition goes in the following way. Eventually the definition is applied to $z$-adic $A$-formal schemes. In the absolute manner we consider the category of all $z$-nilpotent $A$-algebras as the underlying ring categorie $\mathrm{Nil}_{z,A}$, where all such algebras are assumed to be fibered over $\mathbb{N}_\infty$. Then we use the Witt vector functor $W_A$ from \cite{3LH}. Then the prismaization is the stackification over the ring category $\mathrm{Nil}_{z,A}$ where for each such ring we have the groupoid of all the Cartier-Witt ideals fibered over $\mathbb{N}_\infty$, i.e. those maps locally principally generated by distinguished elements in the big Witt vectors (for $w_0$ we require nilpotency and for $w_1$ we require unitality). We follow \cite{3S1}, \cite{3ALBRCS} to the following three step for any $L$-rigid affioid with formal local model:
\begin{itemize}
\item[S1] Over $\mathbb{N}$ we have the analytic prismatization from local formal model from \cite{3S1}, \cite{3ALBRCS}, which provide the analytic prismatization over the generic fiber over $L$;
\item[S2] We take the corresponding compactification of this to $\mathbb{N}_\infty$, by using the morphism to take the fiber product:
\begin{align}
\underline{\mathrm{Prismatization}}_{\mathrm{abs},\mathbb{N}_\infty}\rightarrow \underline{\mathrm{Prismatization}}_{\mathrm{abs},\mathbb{N}_\infty}(\mathbb{N}).
\end{align}
\item[S3] Take condensed analytification from \cite{3CS}, \cite{3CS1}, \cite{3CS2}.
\end{itemize} In this paper we use the notation 
\begin{align}
\underline{\mathrm{Prismatization}}^L_{\mathrm{abs},\mathrm{dR},\mathbb{N}_\infty}
\end{align}
to denote the absolute analytic prismatization in families in our current setting. The definition for this is through the corresponding compactification from the stackification over $\mathbb{N}$, since then we have the morphism:
\begin{align}
\mathcal{P}_{\underline{\mathrm{Prismatization}}_{\mathrm{abs},\mathbb{N}_\infty}(\mathbb{N})} \overset{\sim}{\rightarrow} \varprojlim_{(Q_P,P)} D\mathrm{Cat}(P)^\mathrm{comp}\rightarrow \varprojlim_{(Q_P,P)} D\mathrm{Cat}(P[1/z]_{Q_P})^\mathrm{comp}
\end{align}
where $\mathrm{comp}$ is the completion with respect to the natural toplogy induced from the prisms involved along the inverse limit. Then we take the compactification to reach the de Rham prismatization in families $\mathbb{N}_\infty$:
\begin{align}
\mathcal{P}_{\underline{\mathrm{Prismatization}}_{\mathrm{abs},\mathbb{N}_\infty}} \overset{\sim}{\rightarrow} \overline{\varprojlim_{(Q_P,P)} D\mathrm{Cat}(P)^\mathrm{comp}}\rightarrow \overline{\varprojlim_{(Q_P,P)} D\mathrm{Cat}(P[1/z]_{Q_P})^\mathrm{comp}}.
\end{align}
Then we have to take analytification when needed. Then we have the corresponding filtration analytic prismatization:
\begin{align}
\overline{\underline{\mathrm{Prismatization}}}^L_{\mathrm{abs},\mathrm{Nygaard},\mathrm{dR},\mathbb{N}}
\end{align}
which is defined to be compactification from $\mathbb{N}$ to $\mathbb{N}_\infty$ of the corresponding open version over $\mathbb{N}$:
\begin{align}
{\underline{\mathrm{Prismatization}}}^L_{\mathrm{abs},\mathrm{Nygaard},\mathrm{dR},\mathbb{N}}
\end{align}
where this open version is defined to be taking the product over the finite fibers over $\mathbb{N}_\infty$ of the usual filtration prismatizations over $z_n$-nilpotent $A_n$-algebras. The compactification process needs to be using the following morphisms to reach our definition as the corresponding fiber product in families through the compactification:
\begin{align}
\prod_{n\in \mathbb{N}} {\underline{\mathrm{Prismatization}}}^L_{\mathrm{abs},\mathrm{Nygaard},\mathrm{dR},n}\rightarrow  \prod_{n\in \mathbb{N}} {\underline{\mathrm{Prismatization}}}^L_{\mathrm{abs},\mathrm{dR},n}
\end{align}
by taking the products along all $\mathbb{N}$ of the usual projection from the corresponding filtration analytic prismatization to the prismatization, and:
\begin{align}
{\underline{\mathrm{Prismatization}}}^L_{\mathrm{abs},\mathrm{dR},\mathbb{N}_\infty}\rightarrow  \prod_{n\in \mathbb{N}} {\underline{\mathrm{Prismatization}}}^L_{\mathrm{abs},\mathrm{dR},n}.
\end{align}
Recall that the filtration analytic prismatization is actually certain fibration version of the usual analytic prismatization, then we take further fibration over $\mathbb{N}$, in our current setting. Recall how finally the syntomization prismatization is constructed, which is by taking the descent of the Hodge-Tate morphism and de Rham morphism in the filtration prismatization, along the diagonal morphism from the prismatization with itself. These maps are obviously admitting the corresponding compactification versions in our current setting. Therefore we consider the following three morphisms:
\begin{align}
f_\mathrm{dR}: {\underline{\mathrm{Prismatization}}}^L_{\mathrm{abs},\mathrm{dR},\mathbb{N}_\infty}\rightarrow \overline{\underline{\mathrm{Prismatization}}}^L_{\mathrm{abs},\mathrm{Nygaard},\mathrm{dR},\mathbb{N}},
\end{align}
\begin{align}
f_\mathrm{HT}: {\underline{\mathrm{Prismatization}}}^L_{\mathrm{abs},\mathrm{dR},\mathbb{N}_\infty}\rightarrow \overline{\underline{\mathrm{Prismatization}}}^L_{\mathrm{abs},\mathrm{Nygaard},\mathrm{dR},\mathbb{N}},
\end{align}
\begin{align}
d:  {\underline{\mathrm{Prismatization}}}^L_{\mathrm{abs},\mathrm{dR},\mathbb{N}_\infty}\times {\underline{\mathrm{Prismatization}}}^L_{\mathrm{abs},\mathrm{dR},\mathbb{N}_\infty} \rightarrow {\underline{\mathrm{Prismatization}}}^L_{\mathrm{abs},\mathrm{dR},\mathbb{N}_\infty}.
\end{align}
Then we take the product of $f_\mathrm{dR}$ and $f_\mathrm{HT}$ together and take the corresponding descent along $d$ we have the definition of the syntomization analytic prismatization in families in our current setting over $\mathbb{N}_\infty$:
\begin{align}
\overline{\underline{\mathrm{Prismatization}}}^L_{\mathrm{abs},\mathrm{dR},\mathrm{syntomization},\mathbb{N}}. 
\end{align}
This finishes the definition. $\square$
\end{definition}

\begin{definition}
We consider the corresponding de Rham prismatization in families. Recall the definition goes in the following way. Eventually the definition is applied to $z$-adic $A$-formal schemes. In the absolute manner we consider the category of all $z$-nilpotent $A$-algebras as the underlying ring categorie $\mathrm{Nil}_{z,A}$, where all such algebras are assumed to be fibered over $\mathbb{N}_\infty$. Then we use the Witt vector functor $W_A$ from \cite{3LH}. Then the prismaization is the stackification over the ring category $\mathrm{Nil}_{z,A}$ where for each such ring we have the groupoid of all the Cartier-Witt ideals fibered over $\mathbb{N}_\infty$, i.e. those maps locally principally generated by distinguished elements in the big Witt vectors (for $w_0$ we require nilpotency and for $w_1$ we require unitality). Now map all the construction and definitions in the de Rham setting to $z$-adic $A$-formal scheme $M$. In this paper we use the notation 
\begin{align}
\underline{\mathrm{Prismatization}}^L_{\mathrm{abs},\mathrm{dR},\mathbb{N}_\infty,M}
\end{align}
to denote the absolute analytic de Rham prismatization in families in our current setting. The definition for this is through the corresponding compactification from the stackification over $\mathbb{N}$, since then we have the morphism:
\begin{align}
\mathcal{P}_{\underline{\mathrm{Prismatization}}_{\mathrm{abs},\mathrm{dR},\mathbb{N}_\infty,M}(\mathbb{N})} \overset{\sim}{\rightarrow} \varprojlim_{(Q_P,P)/M} D\mathrm{Cat}(P)^\mathrm{comp}\rightarrow \varprojlim_{(Q_P,P)/M} D\mathrm{Cat}(P[1/z]_{Q_P})^\mathrm{comp}
\end{align}
where $\mathrm{comp}$ is the completion with respect to the natural toplogy induced from the prisms involved along the inverse limit. Then we take the compactification to reach the de Rham prismatization in families $\mathbb{N}_\infty$:
\begin{align}
\mathcal{P}_{\underline{\mathrm{Prismatization}}_{\mathrm{abs},\mathrm{dR},\mathbb{N}_\infty},M} \overset{\sim}{\rightarrow} \overline{\varprojlim_{(Q_P,P)/M} D\mathrm{Cat}(P)^\mathrm{comp}}\rightarrow \overline{\varprojlim_{(Q_P,P)/M} D\mathrm{Cat}(P[1/z]_{Q_P})^\mathrm{comp}}.
\end{align}
Then we have the corresponding filtration prismatization:
\begin{align}
\overline{\underline{\mathrm{Prismatization}}}^L_{\mathrm{abs},\mathrm{Nygaard},\mathrm{dR},\mathbb{N},M}
\end{align}
which is defined to be compactification from $\mathbb{N}$ to $\mathbb{N}_\infty$ of the corresponding open version over $\mathbb{N}$:
\begin{align}
{\underline{\mathrm{Prismatization}}}^L_{\mathrm{abs},\mathrm{Nygaard},\mathrm{dR},\mathbb{N},M}
\end{align}
where this open version is defined to be taking the product over the finite fibers over $\mathbb{N}_\infty$ of the usual filtration prismatizations over $z_n$-nilpotent $A_n$-algebras. The compactification process needs to be using the following morphisms to reach our definition as the corresponding fiber product in families through the compactification:
\begin{align}
\prod_{n\in \mathbb{N}} {\underline{\mathrm{Prismatization}}}^L_{\mathrm{abs},\mathrm{Nygaard},\mathrm{dR},n,M}\rightarrow  \prod_{n\in \mathbb{N}} {\underline{\mathrm{Prismatization}}}^L_{\mathrm{abs},\mathrm{dR},n,M}
\end{align}
by taking the products along all $\mathbb{N}$ of the usual projection from the corresponding filtration prismatization to the prismatization, and:
\begin{align}
{\underline{\mathrm{Prismatization}}}^L_{\mathrm{abs},\mathrm{dR},\mathbb{N}_\infty,M}\rightarrow  \prod_{n\in \mathbb{N}} {\underline{\mathrm{Prismatization}}}^L_{\mathrm{abs},\mathrm{dR},n,M}.
\end{align}
Recall that the filtration analytic prismatization is actually certain fibration version of the usual analytic prismatization, then we take further fibration over $\mathbb{N}$, in our current setting. Recall how finally the syntomization prismatization is constructed, which is by taking the descent of the Hodge-Tate morphism and de Rham morphism in the filtration prismatization, along the diagonal morphism from the prismatization with itself. These maps are obviously admitting the corresponding compactification versions in our current setting. Therefore we consider the following three morphisms:
\begin{align}
f_\mathrm{dR}: {\underline{\mathrm{Prismatization}}}^L_{\mathrm{abs},\mathrm{dR},\mathbb{N}_\infty,M}\rightarrow \overline{\underline{\mathrm{Prismatization}}}^L_{\mathrm{abs},\mathrm{Nygaard},\mathrm{dR},\mathbb{N}_\infty,M},
\end{align}
\begin{align}
f_\mathrm{HT}: {\underline{\mathrm{Prismatization}}}^L_{\mathrm{abs},\mathrm{dR},\mathbb{N}_\infty,M}\rightarrow \overline{\underline{\mathrm{Prismatization}}}^L_{\mathrm{abs},\mathrm{Nygaard},\mathrm{dR},\mathbb{N},M},
\end{align}
\begin{align}
d:  {\underline{\mathrm{Prismatization}}}^L_{\mathrm{abs},\mathrm{dR},\mathbb{N}_\infty,M}\times {\underline{\mathrm{Prismatization}}}^L_{\mathrm{abs},\mathrm{dR},\mathbb{N}_\infty,M} \rightarrow {\underline{\mathrm{Prismatization}}}^L_{\mathrm{abs},\mathrm{dR},\mathbb{N}_\infty,M}.
\end{align}
Then we take the product of $f_\mathrm{dR}$ and $f_\mathrm{HT}$ together and take the corresponding descent along $d$ we have the definition of the syntomization analytic prismatization in families in our current setting over $\mathbb{N}_\infty$:
\begin{align}
\overline{\underline{\mathrm{Prismatization}}}^L_{\mathrm{abs},\mathrm{dR},\mathrm{syntomization},\mathbb{N},M}. 
\end{align}
This finishes the definition. $\square$
\end{definition}

\begin{theorem}\label{theorem20}
There is a version of K\"unneth theorem in this context. To be more precise we have a K\"unneth theorem at least by passing to the corresponding quasicoherent sheaves over the corresponding prismatization (i.e. the corresponding stackifications). This holds for all three stackifications in the families over $\mathbb{N}_\infty$. 
\end{theorem}

\begin{proof}
After we reduce this to formal models, we need to consider the compactification in this theorem. Any prismatization stackification for a particular formal scheme $M$ is basically a stackification over $M$ and ringed we use the notation $\mathcal{P}$ to denote the structure sheaf. As in the usual situation we have that the derived $\infty$-category of all the $\mathcal{P}$-modules is equivalent to the prismatic site for $M$. However the derived prismatic cohomology theory in this setting does satisfy the derived K\"unneth theorem on the product of the corresponding $p$-adic formal ring locally (which can be glue over formal schemes like $M$). Then the result can be generalized to our setting over $\mathbb{N}$ away from $\infty$. Then we can take the corresponding compactification along:
\begin{align}
{\underline{\mathrm{Prismatization}}}^L_{\mathrm{abs},\mathbb{N}_\infty,\mathrm{dR},M}\rightarrow  \prod_{n\in \mathbb{N}} {\underline{\mathrm{Prismatization}}}^L_{\mathrm{abs},n,\mathrm{dR},M}.
\end{align}
This implies the derived K\"unneth theorem holds for analytic filtration prismatization and analytic syntomization prismatization. We then get the derived version of the K\"unneth theorem in the current setting by restricting to the 3 de Rham prismatizations. This finishes the proof.
\end{proof}

\begin{theorem}
All three prismatization, filtration prismatization and syntomization prismatization fibered over $\mathbb{N}_\infty$ satisfy our \cref{situation2} conditions: (A1), (A2), (A3), the formalism pullback and pushforward in (A4), (A5).
\end{theorem}


\begin{proof}
We check this one by one. First for the first condition it is true after we consider the corresponding structure sheaves. The second condition is proved above. For the third condition we consider the corresponding derived $\infty$-categories of quasicoherent sheaves. A4 holds for pullbacks and pushforward. This is true over $\mathbb{N}$ then we consider the fiber product through:
\begin{align}
{\underline{\mathrm{Prismatization}}}_{\mathrm{abs},\mathbb{N}_\infty,\mathrm{dR},M}\rightarrow  \prod_{n\in \mathbb{N}} {\underline{\mathrm{Prismatization}}}_{\mathrm{abs},n,\mathrm{dR},M}
\end{align}
to reach the corresponding compactification. Finally A5 holds see \cite[Chapter 4, in particular 4.7, 4.8, 4.9, 4.10]{3A}.
\end{proof}

\begin{definition}
The three analytic prismatizations:
\begin{align}
&\underline{\mathrm{Prismatization}}^L_{\mathrm{abs},\mathbb{N}_\infty,\mathrm{dR},M}\\
&\overline{\underline{\mathrm{Prismatization}}}^L_{\mathrm{abs},\mathrm{Nygaard},\mathbb{N}_\infty,\mathrm{dR},M}\\
&\overline{\underline{\mathrm{Prismatization}}}^L_{\mathrm{abs},\mathbb{N}_\infty,\mathrm{syntomization},\mathrm{dR},M}
\end{align}
have a small arc-stack version as well as a small v-stack version, where the condensed structure sheaves are pull-back along the strictly totally disconnected coverings. For such switching of categories and consideration, we use the then the following different notation:
\begin{align}
&\underline{\mathrm{Prismatization}}^L_{\mathrm{abs},\mathbb{N}_\infty,\mathrm{dR},-}\\
&\overline{\underline{\mathrm{Prismatization}}}^L_{\mathrm{abs},\mathrm{Nygaard},\mathbb{N}_\infty,\mathrm{dR},-}\\
&\overline{\underline{\mathrm{Prismatization}}}^L_{\mathrm{abs},\mathbb{N}_\infty,\mathrm{syntomization},\mathrm{dR},-}
\end{align}
where $-$ is a small arc stack or a small v-stack attached to a rigid analytic space over $L$ over $\mathbb{N}_\infty$, however locally the topology is arc-topology or $v$-topology.
\end{definition}

\begin{theorem}
There is a version of K\"unneth theorem in this context
\begin{align}
&\underline{\mathrm{Prismatization}}^L_{\mathrm{abs},\mathbb{N}_\infty,\mathrm{dR},-}\\
&\overline{\underline{\mathrm{Prismatization}}}^L_{\mathrm{abs},\mathrm{Nygaard},\mathbb{N}_\infty,\mathrm{dR},-}\\
&\overline{\underline{\mathrm{Prismatization}}}^L_{\mathrm{abs},\mathbb{N}_\infty,\mathrm{syntomization},\mathrm{dR},-}.
\end{align}
To be more precise we have a K\"unneth theorem at least by passing to the corresponding quasicoherent sheaves over the corresponding prismatization (i.e. the corresponding stackifications). This holds for all three stackifications in the families over $\mathbb{N}_\infty$. 
\end{theorem}

\begin{proof}
We need to consider the compactification in this theorem, after we reduce the consideration to formal models. Any prismatization stackification for a particular formal scheme $-$ is basically a stackification over $-$ and ringed we use the notation $\mathcal{P}$ to denote the structure sheaf. As in the usual situation we have that the derived $\infty$-category of all the $\mathcal{P}$-modules is equivalent to the prismatic site for $-$. However the derived prismatic cohomology theory in this setting does satisfy the K\"unneth theorem in the derived version on the product of the corresponding $p$-adic formal ring locally (which can be glue over formal schemes like $-$). Then the result can be generalized to our setting over $\mathbb{N}$ away from $\infty$. Then we can take the corresponding compactification along:
\begin{align}
{\underline{\mathrm{Prismatization}}}^L_{\mathrm{abs},\mathbb{N}_\infty,\mathrm{dR},-}\rightarrow  \prod_{n\in \mathbb{N}} {\underline{\mathrm{Prismatization}}}^L_{\mathrm{abs},n,\mathrm{dR},-}.
\end{align}
This implies the derived K\"unneth theorem holds for analytic filtration prismatization and analytic syntomization prismatization. We then get the derived version of the K\"unneth theorem in the current setting by restricting to the 3 de Rham prismatizations. This finishes the proof.
\end{proof}

\begin{theorem}
All three analytic prismatization, analytic filtration prismatization and analytic syntomization prismatization over $\mathbb{N}_\infty$ satisfy our \cref{situation1} conditions: (A1), (A2), (A3), the formalism pullback and pushforward in (A4), (A5).
\end{theorem}


\begin{proof}
We check this one by one. First for the first condition it is true after we consider the corresponding structure sheaves. The second condition is proved above. For the third condition we consider the corresponding derived $\infty$-categories of quasicoherent sheaves. A4 holds for pullbacks and pushforward. This is true over $\mathbb{N}$ then we consider the fiber product through:
\begin{align}
{\underline{\mathrm{Prismatization}}}^L_{\mathrm{abs},\mathbb{N}_\infty,\mathrm{dR},-}\rightarrow  \prod_{n\in \mathbb{N}} {\underline{\mathrm{Prismatization}}}^L_{\mathrm{abs},n,\mathrm{dR},-}
\end{align}
to reach the corresponding compactification. Finally A5 holds see \cite[Chapter 4, in particular 4.7, 4.8, 4.9, 4.10]{3A}. However since we are talking about rigid analytic spaces by using perfectoid coverings, \cite{3A} can be directly applied, where one can derive all the results from the formalism in \cite{3A}, which allows us to reduce the corresponding proof to the formal scheme situation before. Again we need to consider the generalized version of \cite{3A} along \cite{3S}, i.e. regard the corresponding rigid analytic motivic cohomology theories as ones in our \cref{situation1}.
\end{proof}


\begin{definition}
We in this context use the notations:
\begin{align}
&\mathrm{Galois}(\underline{\mathrm{Prismatization}}^L_{\mathrm{abs},\mathbb{N}_\infty,\mathrm{dR},-})\\
&\mathrm{Galois}(\overline{\underline{\mathrm{Prismatization}}}^L_{\mathrm{abs},\mathrm{Nygaard},\mathbb{N}_\infty,\mathrm{dR},-})\\
&\mathrm{Galois}(\overline{\underline{\mathrm{Prismatization}}}^L_{\mathrm{abs},\mathbb{N}_\infty,\mathrm{syntomization},\mathrm{dR},-})
\end{align}
to denote the corresponding Hopf algebraic motivic Galois fundamental groups, which is the defined to be the spectra of the associated Hopf algebra.
\end{definition}


\begin{theorem}
There are morphisms from these motivic Galois fundamental groups to the corresponding motivic Galois groups of $A$, and moreover $L$: 
\begin{align}
&\mathrm{Galois}(\underline{\mathrm{Prismatization}}^L_{\mathrm{abs},\mathbb{N}_\infty,\mathrm{dR},-})\rightarrow \mathrm{Gal}(\overline{L}/L),\\
&\mathrm{Galois}(\overline{\underline{\mathrm{Prismatization}}}^L_{\mathrm{abs},\mathrm{Nygaard},\mathbb{N}_\infty,\mathrm{dR},-})\rightarrow \mathrm{Gal}(\overline{L}/L),\\
&\mathrm{Galois}(\overline{\underline{\mathrm{Prismatization}}}^L_{\mathrm{abs},\mathbb{N}_\infty,\mathrm{syntomization},\mathrm{dR},-})\rightarrow \mathrm{Gal}(\overline{L}/L).
\end{align}
\end{theorem}

\begin{proof}
This is formal since we just apply the construction and definition to the point situation (regard the scheme $\mathrm{Spec}A$ or $\mathrm{Spec}L$ as the corresponding formal scheme or rigid analytic space), then the theorem follows by the functoriality of the construction of Hopf algebras. Then we reduce to the results before for formal local model before following \cite{3A}. 
\end{proof}




\subsection{Three analytic prismatizations in the de Rham-Hodge-Tate setting}


\begin{definition}
We consider the corresponding de Rham-Hodge-Tate prismatization in families. Recall the definition goes in the following way. Eventually the definition is applied to $z$-adic $A$-formal schemes. In the absolute manner we consider the category of all $z$-nilpotent $A$-algebras as the underlying ring categorie $\mathrm{Nil}_{z,A}$, where all such algebras are assumed to be fibered over $\mathbb{N}_\infty$. Then we use the Witt vector functor $W_A$ from \cite{3LH}. Then the prismaization is the stackification over the ring category $\mathrm{Nil}_{z,A}$ where for each such ring we have the groupoid of all the Cartier-Witt ideals fibered over $\mathbb{N}_\infty$, i.e. those maps locally principally generated by distinguished elements in the big Witt vectors (for $w_0$ we require nilpotency and for $w_1$ we require unitality). In this paper we use the notation 
\begin{align}
\underline{\mathrm{Prismatization}}^L_{\mathrm{abs},\mathrm{dRHT},\mathbb{N}_\infty}
\end{align}
to denote the absolute analytic de Rham-Hodge-Tate prismatization in families in our current setting. The definition for this is through the corresponding compactification from the stackification over $\mathbb{N}$, since then we have the morphism:
\begin{align}
\mathcal{P}_{\underline{\mathrm{Prismatization}}_{\mathrm{abs},\mathrm{dRHT},\mathbb{N}_\infty}(\mathbb{N})} \overset{\sim}{\rightarrow} \varprojlim_{(Q_P,P)} D\mathrm{Cat}(P)^\mathrm{comp}\rightarrow \varprojlim_{(Q_P,P)} D\mathrm{Cat}(P[1/z]_{Q_P})^\mathrm{comp}\rightarrow \varprojlim_{(Q_P,P)} D\mathrm{Cat}(P[1/z]/{Q_P})^\mathrm{comp}
\end{align}
where $\mathrm{comp}$ is the completion with respect to the natural toplogy induced from the prisms involved along the inverse limit. Then we take the compactification to reach the de Rham-Hodge-Tate prismatization in families $\mathbb{N}_\infty$:
\begin{align}
\mathcal{P}_{\underline{\mathrm{Prismatization}}_{\mathrm{abs},\mathrm{dRHT},\mathbb{N}_\infty}} \overset{\sim}{\rightarrow} \overline{\varprojlim_{(Q_P,P)} D\mathrm{Cat}(P)^\mathrm{comp}}\rightarrow \overline{\varprojlim_{(Q_P,P)} D\mathrm{Cat}(P[1/z]_{Q_P})^\mathrm{comp}}\rightarrow \overline{\varprojlim_{(Q_P,P)} D\mathrm{Cat}(P[1/z]/{Q_P})^\mathrm{comp}}.
\end{align}
Then we have the corresponding filtration prismatization:
\begin{align}
\overline{\underline{\mathrm{Prismatization}}}^L_{\mathrm{abs},\mathrm{Nygaard},\mathrm{dRHT},\mathbb{N}}
\end{align}
which is defined to be compactification from $\mathbb{N}$ to $\mathbb{N}_\infty$ of the corresponding open version over $\mathbb{N}$:
\begin{align}
{\underline{\mathrm{Prismatization}}}^L_{\mathrm{abs},\mathrm{Nygaard},\mathrm{dRHT},\mathbb{N}}
\end{align}
where this open version is defined to be taking the product over the finite fibers over $\mathbb{N}_\infty$ of the usual filtration prismatizations over $z_n$-nilpotent $A_n$-algebras. The compactification process needs to be using the following morphisms to reach our definition as the corresponding fiber product in families through the compactification:
\begin{align}
\prod_{n\in \mathbb{N}} {\underline{\mathrm{Prismatization}}}^L_{\mathrm{abs},\mathrm{Nygaard},\mathrm{dRHT},n}\rightarrow  \prod_{n\in \mathbb{N}} {\underline{\mathrm{Prismatization}}}^L_{\mathrm{abs},\mathrm{dRHT},n}
\end{align}
by taking the products along all $\mathbb{N}$ of the usual projection from the corresponding filtration prismatization to the prismatization, and:
\begin{align}
{\underline{\mathrm{Prismatization}}}^L_{\mathrm{abs},\mathrm{dRHT},\mathbb{N}_\infty}\rightarrow  \prod_{n\in \mathbb{N}} {\underline{\mathrm{Prismatization}}}^L_{\mathrm{abs},\mathrm{dRHT},n}.
\end{align}
Recall that the filtration analytic prismatization is actually certain fibration version of the usual analytic prismatization, then we take further fibration over $\mathbb{N}$, in our current setting. Recall how finally the syntomization prismatization is constructed, which is by taking the descent of the Hodge-Tate morphism and de Rham morphism in the filtration prismatization, along the diagonal morphism from the prismatization with itself. These maps are obviously admitting the corresponding compactification versions in our current setting. Therefore we consider the following three morphisms:
\begin{align}
f_\mathrm{dR}: {\underline{\mathrm{Prismatization}}}^L_{\mathrm{abs},\mathrm{dRHT},\mathbb{N}_\infty}\rightarrow \overline{\underline{\mathrm{Prismatization}}}^L_{\mathrm{abs},\mathrm{Nygaard},\mathrm{dRHT},\mathbb{N}_\infty},
\end{align}
\begin{align}
f_\mathrm{HT}: {\underline{\mathrm{Prismatization}}}^L_{\mathrm{abs},\mathrm{dRHT},\mathbb{N}_\infty}\rightarrow \overline{\underline{\mathrm{Prismatization}}}^L_{\mathrm{abs},\mathrm{Nygaard},\mathrm{dRHT},\mathbb{N}_\infty},
\end{align}
\begin{align}
d:  {\underline{\mathrm{Prismatization}}}^L_{\mathrm{abs},\mathrm{dRHT},\mathbb{N}_\infty}\times {\underline{\mathrm{Prismatization}}}^L_{\mathrm{abs},\mathrm{dRHT},\mathbb{N}_\infty} \rightarrow {\underline{\mathrm{Prismatization}}}^L_{\mathrm{abs},\mathrm{dRHT},\mathbb{N}_\infty}.
\end{align}
Then we take the product of $f_\mathrm{dR}$ and $f_\mathrm{HT}$ together and take the corresponding descent along $d$ we have the definition of the analytic syntomization prismatization in families in our current setting over $\mathbb{N}_\infty$:
\begin{align}
\overline{\underline{\mathrm{Prismatization}}}^L_{\mathrm{abs},\mathrm{dRHT},\mathrm{syntomization},\mathbb{N}}. 
\end{align}
This finishes the definition. $\square$
\end{definition}

\begin{definition}
We consider the corresponding de Rham-Hodge-Tate prismatization in families. Recall the definition goes in the following way. Eventually the definition is applied to $z$-adic $A$-formal schemes. In the absolute manner we consider the category of all $z$-nilpotent $A$-algebras as the underlying ring categorie $\mathrm{Nil}_{z,A}$, where all such algebras are assumed to be fibered over $\mathbb{N}_\infty$. Then we use the Witt vector functor $W_A$ from \cite{3LH}. Then the prismaization is the stackification over the ring category $\mathrm{Nil}_{z,A}$ where for each such ring we have the groupoid of all the Cartier-Witt ideals fibered over $\mathbb{N}_\infty$, i.e. those maps locally principally generated by distinguished elements in the big Witt vectors (for $w_0$ we require nilpotency and for $w_1$ we require unitality). Now map all the construction and definitions in the de Rham-Hodge-Tate setting to $z$-adic $A$-formal scheme $M$. In this paper we use the notation 
\begin{align}
\underline{\mathrm{Prismatization}}^L_{\mathrm{abs},\mathrm{dRHT},\mathbb{N}_\infty,M}
\end{align}
to denote the absolute analytic prismatization in families in our current setting. The definition for this is through the corresponding compactification from the stackification over $\mathbb{N}$, since then we have the morphism:
\begin{align}
&\mathcal{P}_{\underline{\mathrm{Prismatization}}_{\mathrm{abs},\mathrm{dRHT},\mathbb{N}_\infty,M}(\mathbb{N})} \overset{\sim}{\rightarrow} \varprojlim_{(Q_P,P)/M} D\mathrm{Cat}(P)^\mathrm{comp}\\
&\rightarrow \varprojlim_{(Q_P,P)/M} D\mathrm{Cat}(P[1/z]/{Q_P})^\mathrm{comp}\rightarrow \varprojlim_{(Q_P,P)/M} D\mathrm{Cat}(P[1/z]/{Q_P})^\mathrm{comp}
\end{align}
where $\mathrm{comp}$ is the completion with respect to the natural toplogy induced from the prisms involved along the inverse limit. Then we take the compactification to reach the de Rham prismatization in families $\mathbb{N}_\infty$:
\begin{align}
&\mathcal{P}_{\underline{\mathrm{Prismatization}}_{\mathrm{abs},\mathrm{dRHT},\mathbb{N}_\infty},M} \overset{\sim}{\rightarrow} \overline{\varprojlim_{(Q_P,P)/M} D\mathrm{Cat}(P)^\mathrm{comp}}\\
&\rightarrow \overline{\varprojlim_{(Q_P,P)/M} D\mathrm{Cat}(P[1/z]/{Q_P})^\mathrm{comp}}\rightarrow \overline{\varprojlim_{(Q_P,P)/M} D\mathrm{Cat}(P[1/z]/{Q_P})^\mathrm{comp}}.
\end{align}
Then we have the corresponding filtration prismatization:
\begin{align}
\overline{\underline{\mathrm{Prismatization}}}^L_{\mathrm{abs},\mathrm{Nygaard},\mathrm{dRHT},\mathbb{N},M}
\end{align}
which is defined to be compactification from $\mathbb{N}$ to $\mathbb{N}_\infty$ of the corresponding open version over $\mathbb{N}$:
\begin{align}
{\underline{\mathrm{Prismatization}}}^L_{\mathrm{abs},\mathrm{Nygaard},\mathrm{dRHT},\mathbb{N},M}
\end{align}
where this open version is defined to be taking the product over the finite fibers over $\mathbb{N}_\infty$ of the usual filtration prismatizations over $z_n$-nilpotent $A_n$-algebras. The compactification process needs to be using the following morphisms to reach our definition as the corresponding fiber product in families through the compactification:
\begin{align}
\prod_{n\in \mathbb{N}} {\underline{\mathrm{Prismatization}}}^L_{\mathrm{abs},\mathrm{Nygaard},\mathrm{dRHT},n,M}\rightarrow  \prod_{n\in \mathbb{N}} {\underline{\mathrm{Prismatization}}}^L_{\mathrm{abs},\mathrm{dRHT},n,M}
\end{align}
by taking the products along all $\mathbb{N}$ of the usual projection from the corresponding filtration prismatization to the prismatization, and:
\begin{align}
{\underline{\mathrm{Prismatization}}}^L_{\mathrm{abs},\mathrm{dRHT},\mathbb{N}_\infty,M}\rightarrow  \prod_{n\in \mathbb{N}} {\underline{\mathrm{Prismatization}}}^L_{\mathrm{abs},\mathrm{dRHT},n,M}.
\end{align}
Recall that the analytic filtration prismatization is actually certain fibration version of the usual analytic prismatization, then we take further fibration over $\mathbb{N}$. Recall how finally the syntomization prismatization is constructed, which is by taking the descent of the Hodge-Tate morphism and de Rham morphism in the filtration prismatization, along the diagonal morphism from the prismatization with itself. These maps are obviously admitting the corresponding compactification versions in our current setting. Therefore we consider the following three morphisms:
\begin{align}
f_\mathrm{dR}: {\underline{\mathrm{Prismatization}}}^L_{\mathrm{abs},\mathrm{dRHT},\mathbb{N}_\infty,M}\rightarrow \overline{\underline{\mathrm{Prismatization}}}^L_{\mathrm{abs},\mathrm{Nygaard},\mathrm{dRHT},\mathbb{N}_\infty,M},
\end{align}
\begin{align}
f_\mathrm{HT}: {\underline{\mathrm{Prismatization}}}^L_{\mathrm{abs},\mathrm{dRHT},\mathbb{N}_\infty,M}\rightarrow \overline{\underline{\mathrm{Prismatization}}}^L_{\mathrm{abs},\mathrm{Nygaard},\mathrm{dRHT},\mathbb{N}_\infty,M},
\end{align}
\begin{align}
d:  {\underline{\mathrm{Prismatization}}}^L_{\mathrm{abs},\mathrm{dRHT},\mathbb{N}_\infty,M}\times {\underline{\mathrm{Prismatization}}}^L_{\mathrm{abs},\mathrm{dRHT},\mathbb{N}_\infty,M} \rightarrow {\underline{\mathrm{Prismatization}}}^L_{\mathrm{abs},\mathrm{dRHT},\mathbb{N}_\infty,M}.
\end{align}
Then we take the product of $f_\mathrm{dR}$ and $f_\mathrm{HT}$ together and take the corresponding descent along $d$ we have the definition of the analytic syntomization prismatization in families in our current setting over $\mathbb{N}_\infty$:
\begin{align}
\overline{\underline{\mathrm{Prismatization}}}^L_{\mathrm{abs},\mathrm{dRHT},\mathrm{syntomization},\mathbb{N},M}. 
\end{align}
This finishes the definition. $\square$
\end{definition}

\begin{theorem}
There is a version of K\"unneth theorem in this context. To be more precise we have a K\"unneth theorem at least by passing to the corresponding quasicoherent sheaves over the corresponding prismatization (i.e. the corresponding stackifications). This holds for all three stackifications in the families over $\mathbb{N}_\infty$. 
\end{theorem}

\begin{proof}
See the proof of \cref{theorem20}.
\end{proof}

\begin{theorem}
All three prismatization, filtration prismatization and syntomization prismatization over $\mathbb{N}_\infty$ satisfy our \cref{situation2} conditions: (A1), (A2), (A3), the formalism pullback and pushforward in (A4), (A5).
\end{theorem}


\begin{proof}
We check this one by one. First for the first condition it is true after we consider the corresponding structure sheaves. The second condition is proved above. For the third condition we consider the corresponding derived $\infty$-categories of quasicoherent sheaves. A4 holds for pullbacks and pushforward. This is true over $\mathbb{N}$ then we consider the fiber product through:
\begin{align}
{\underline{\mathrm{Prismatization}}}^L_{\mathrm{abs},\mathbb{N}_\infty,\mathrm{dRHT},M}\rightarrow  \prod_{n\in \mathbb{N}} {\underline{\mathrm{Prismatization}}}^L_{\mathrm{abs},n,\mathrm{dRHT},M}
\end{align}
to reach the corresponding compactification. Finally A5 holds see \cite[Chapter 4, in particular 4.7, 4.8, 4.9, 4.10]{3A}.
\end{proof}















\newpage
\subsection*{Acknowledgements}

We thank Professor Kedlaya for discussion on Z\'abr\'adi's work. We then get the chance to write this paper down following some recently established foundation on analytic geometry. We thank Professor Kedlaya for learning the motivic cohomology theory related to Robba sheaves (such as the \textit{K\"unneth theorem} in this setting) from him. We in this paper generalize a previous consideration of Sorensen in the field $\mathbb{F}_p$-situation to the $\mathbb{Z}_p$-ring situation closely following the work of Schneider, Sorensen, Schneider-Sorensen and Heyer-Mann. We follow also closely the work \cite{Sa} on derived $E_1$-rings over any commutative rings and derived Hochschild homological consideration. We thank Professor Sorensen for correspondence during my writing of this paper and for inspiration on generalizing Hodge theory along Breuil-Schneider's key construction and observation. 










\newpage
\begin{thebibliography}{}

\bibitem[BBBK]{BBBK} Bambozzi, Federico, Oren Ben-Bassat, and Kobi Kremnizer. "Analytic geometry over F1 and the Fargues-Fontaine curve." Advances in Mathematics 356 (2019): 106815. 
\bibitem[BBK]{BBK} Ben-Bassat, Oren, and Kobi Kremnizer. "Fr\'echet modules and descent." Theory and Applications of Categories 39, no. 9 (2023): 207-266.
\bibitem[BBKK]{BBKK} Ben-Bassat, Oren, Jack Kelly, and Kobi Kremnizer. "A perspective on the foundations of derived analytic geometry." arXiv preprint arXiv:2405.07936 (2024).
\bibitem[CS1]{CS1} Dustin Clausen and Peter Scholze. "Lectures on condensed mathematics." Https://www.math.uni-bonn.de/people/scholze/Condensed.pdf. 
\bibitem[CS2]{CS2} Dustin Clausen and Peter Scholze. "Lectures on analytic geometry." Https://www.math.uni-bonn.de/people/scholze/Analytic.pdf.  
\bibitem[CS3]{CS3} Dustin Clausen and Peter Scholze. "Analytic stacks." Https://people.mpim-bonn.mpg.de/scholze/AnalyticStacks.html.  
 



\bibitem[G1]{G1} A. Grothendieck. "Letter to Serre (on motives)." 1964.
\bibitem[G2]{G2} A. Grothendieck. "Letter to Serre (on K-theory)." 1964.
\bibitem[V]{V} Voevodsky, Vladimir. "A1-homotopy theory." In Proceedings of the international congress of mathematicians, vol. 1, pp. 579-604. 1998. 
\bibitem[A]{A} Ayoub, Joseph. "Motifs des vari\'et\'es analytiques rigides." Soci\'et\'e Math\'ematique de France. Memoires 140 (2015): 1-386.
\bibitem[S1]{S1} Peter Scholze. "Berkovich motives." arXiv:2412.03882.
\bibitem[S2]{S2} Peter Scholze. "Geometrization of local langlands, motivically."


\bibitem[K1]{K1} Kiehl, Reinhardt. "Der Endlichkeitssatz f\"ur eigentliche Abbildungen in der nichtarchimedischen Funktionentheorie." Inventiones mathematicae 2 (1967): 191-214. 
\bibitem[KL]{KL} Kedlaya, Kiran S., and Ruochuan Liu. "Finiteness of cohomology of local systems on rigid analytic spaces." arXiv preprint arXiv:1611.06930 (2016).
\bibitem[T]{T} Tong, Xin. "Analytic Geometry and Hodge-Frobenius Structure." arXiv preprint arXiv:2011.08358 (2020).
\bibitem[CKZ]{CKZ} Carter, Annie, Kiran S. Kedlaya, and Gergely Z\'abr\'adi. "Drinfeld's lemma for perfectoid spaces and overconvergence of multivariate $(\varphi, \Gamma)$-modules." Doc. Math 26 (2021): 1329-1393. 
\bibitem[PZ]{PZ} Pal, Aprameyo, and Gergely Z\'abr\'adi. "Cohomology and overconvergence for representations of powers of Galois groups." Journal of the Institute of Mathematics of Jussieu 20, no. 2 (2021): 361-421.
\bibitem[C]{C} Colmez, Pierre. "Repr\'esentations de $GL_2(Q_p)$ et $(\varphi, \Gamma)$-modules." Ast\'erisque 330 (2010): 281-509.

\bibitem[KPX]{KPX} Kedlaya, Kiran, Jonathan Pottharst, and Liang Xiao. "Cohomology of arithmetic families of $(\varphi, \Gamma)$-modules." Journal of the American Mathematical Society 27, no. 4 (2014): 1043-1115.
\bibitem[ST]{ST} Schneider, Peter, and Jeremy Teitelbaum. "Algebras of p-adic distributions and admissible representations." Inventiones Mathematicae 153, no. 1 (2003).
\bibitem[Z1]{Z1} Z\'abr\'adi, Gergely. "Generalized Robba rings." Israel Journal of Mathematics 191 (2012): 817-887.


\bibitem[A2]{A2} Andreychev, Grigory. "$ K $-Theorie adischer R\"aume." arXiv preprint arXiv:2311.04394 (2023).
\bibitem[BGT]{BGT} Blumberg, Andrew J., David Gepner, and Gon\c{c}alo Tabuada. "A universal characterization of higher algebraic K-theory." Geometry $\&$ Topology 17, no. 2 (2013): 733-838.


\bibitem[Sc1]{Sc1} Schneider, Peter. "Smooth representations and Hecke modules in characteristic p." Pacific Journal of Mathematics 279, no. 1 (2015): 447-464.
\bibitem[SS1]{SS1} Schneider, Peter, and Claus Sorensen. "Duals and admissibility in natural characteristic." Representation Theory of the American Mathematical Society 27, no. 02 (2023): 30-50.
\bibitem[SS2]{SS2} Schneider, Peter, and Claus Sorensen. "Derived smooth induction with applications." arXiv preprint arXiv:2402.01504 (2024).
\bibitem[So1]{So1} Sorensen, Claus. "Koszul duality for Iwasawa algebras modulo $p$." Representation Theory of the American Mathematical Society 24, no. 5 (2020): 151-177.
\bibitem[Sa]{Sa} Sagave, Steffen. "DG-algebras and derived $A_\infty$-algebras." Journal f\"ur die Reine und Angewandte Mathematik 2010, no. 639 (2010).
\bibitem[HM]{HM} Heyer, Claudius, and Lucas Mann. "6-Functor Formalisms and Smooth Representations." arXiv preprint arXiv:2410.13038 (2024).

\bibitem[D2]{D2} Drinfel'd, Vladimir G. "Elliptic modules." Mathematics of the USSR-Sbornik 23, no. 4 (1974): 561. 

\bibitem[L1]{L1} R. Langlands. "Letter to Weil." 1967.

\bibitem[D]{D} Drinfeld, Vladimir Gershonovich. "Langlands' conjecture for $GL (2)$ over functional fields." In Proceedings of the International Congress of Mathematicians (Helsinki, 1978), vol. 2, pp. 565-574. 1980.

\bibitem[1To1]{1To1} Tong, Xin. "Generalized Motives through Witt Vectors." arXiv preprint arXiv:2407.01417 (2024).
\bibitem[1To2]{1To2} Tong, Xin. "Topologization and Functional Analytification I: Intrinsic Morphisms of Commutative Algebras." arXiv preprint arXiv:2102.10766 (2021).
\bibitem[1To3]{1To3} Tong, Xin. "Topologization and Functional Analytification II: $\infty $-Categorical Motivic Constructions for Homotopical Contexts." arXiv preprint arXiv:2112.12679 (2021).
\bibitem[1To4]{1To4} Tong, Xin. "Topologization and Functional Analytification III." arXiv preprint arXiv:2405.14180 (2024).
\bibitem[1To5]{1To5} Xin Tong. "$\infty$-Categorical Generalized Langlands Program I: Mixed-Parity Modules and Sheaves." arXiv:2311.10019.
\bibitem[1S4]{1S4} Peter Scholze. "Some remarks on prismatic cohomology of rigid spaces." 
\bibitem[1ALBRCS]{1ALBRCS} Johannes Ansch\"utz, Arthur-C\'esar Le Bras, Juan Esteban Rodriguez Camargo and Peter Scholze. "Analytic Prismatization."


\bibitem[1G]{1G} Alexandre Grothendieck. "Letter to Serre." 1964.


\bibitem[1T1]{1T1} Tate, John T. "p-Divisible groups." In Proceedings of a Conference on Local Fields: NUFFIC Summer School held at Driebergen (The Netherlands) in 1966, pp. 158-183. Berlin, Heidelberg: Springer Berlin Heidelberg, 1967.
\bibitem[1F1]{1F1} Fontaine, Jean-Marc. "Sur certains types de repr\'esentations p-adiques du groupe de Galois d'un corps local; construction d'un anneau de Barsotti-Tate." Annals of Mathematics 115, no. 3 (1982): 529-577.
\bibitem[1F2]{1F2} Fontaine, Jean-Marc. "Arithm\'etique des repr\'esentations galoisiennes p-adiques." Ast\'erisque 295 (2004): 1-115.
\bibitem[1S1]{1S1} Scholze, Peter. "\'Etale cohomology of diamonds." arXiv preprint arXiv:1709.07343 (2017).
\bibitem[1KL]{1KL} Kiran Kedlaya and Ruochuan Liu. "Relative $p$-adic Hodge theory I: foundations." Ast\'erisque 371, 2015.
\bibitem[1KL1]{1KL1} Kedlaya, Kiran S., and Ruochuan Liu. "Relative p-adic Hodge theory, II: Imperfect period rings." arXiv preprint arXiv:1602.06899 (2016).
\bibitem[1S2]{1S2} Scholze, Peter. "P-adic Hodge theory for rigid-analytic varieties." In Forum of Mathematics, Pi, vol. 1, p. e1. Cambridge University Press, 2013. 
\bibitem[1S3]{1S3} Scholze, Peter. "Perfectoid spaces." Publications math\'ematiques de l'IH\'ES 116, no. 1 (2012): 245-313.
\bibitem[1BS1]{1BS1} Breuil, Christophe and Schneider, Peter. "First steps towards p-adic Langlands functoriality" Journal f\"ur die reine und angewandte Mathematik 2007, no. 610 (2007): 149-180. https://doi.org/10.1515/CRELLE.2007.070.

\bibitem[1CSA]{1CSA} Dustin Clausen and Peter Scholze. Lectures on Condensed Mathematics. Https://www.math.uni-bonn.de/people/scholze/Condensed.pdf.
\bibitem[1CSB]{1CSB} Dustin Clausen and Peter Scholze. Lectures on Analytic Geometry. Https://www.math.uni-bonn.de/people/scholze/Analytic.pdf.
\bibitem[1CS]{1CS} Dustin Clausen and Peter Scholze. Analytic Stacks. Https://people.mpim-bonn.mpg.de/scholze/AnalyticStacks.html.

\bibitem[1BS]{1BS} Bhatt, Bhargav, and Peter Scholze. "Prisms and prismatic cohomology." Annals of Mathematics 196, no. 3 (2022): 1135-1275. 
\bibitem[1D]{1D} Drinfeld, Vladimir. "Prismatization." Selecta Mathematica 30, no. 3 (2024): 49.
\bibitem[1BL]{1BL} Bhatt, Bhargav, and Jacob Lurie. "Absolute prismatic cohomology." arXiv preprint arXiv:2201.06120 (2022).
\bibitem[1BL1]{1BL1} Bhatt, Bhargav and Jacob Lurie. "A $p$-adic Riemann-Hilbert functor I: torsion coefficients."
\bibitem[1BL2]{1BL2} Bhatt, Bhargav and Jacob Lurie. "A $p$-adic Riemann-Hilbert functor II: $\mathbb{Q}_p$-coefficients."

\bibitem[1K]{1K} Kedlaya, Kiran S. "A p-adic local monodromy theorem." Annals of mathematics (2004): 93-184.
\bibitem[1A]{1A} Andr\'e, Yves. "Filtrations de type Hasse-Arf et monodromie p-adique." Inventiones Mathematicae 148 (2002): 285-317.
\bibitem[1M]{1M} Mebkhout, Zoghman. "Analogue p-adique du th\'eoreme de Turrittin et le th\'eoreme de la monodromie p-adique." Inventiones mathematicae 148 (2002): 319-351.

\bibitem[1BA]{1BA} Berger, Laurent. "Repr\'esentations p-adiques et \'equations diff\'erentielles." Inventiones mathematicae 148 (2002): 219-284.

\bibitem[1AB]{1AB} Andreatta, Fabrizio, and Olivier Brinon. "$\mathrm {B} _ {\mathrm {dR}} $-repr\'esentations dans le cas relatif." In Annales scientifiques de l'Ecole normale sup\'erieure, vol. 43, no. 2, pp. 279-339. 2010.

\bibitem[1S5]{1S5} Peter Scholze. "Berkovich Motives." arXiv: 2412.03882.

\bibitem[1S6]{1S6} Peter Scholze. "Geometrization of local Langlands, motivically." In preparation.

\bibitem[1FS]{1FS} Fargues, Laurent, and Peter Scholze. "Geometrization of the local Langlands correspondence." arXiv preprint arXiv:2102.13459 (2021).

\bibitem[1GL]{1GL} Genestier, Alain, and Vincent Lafforgue. "Chtoucas restreints pour les groupes r\'eductifs et param\'etrisation de Langlands locale." arXiv preprint arXiv:1709.00978 (2017).

\bibitem[1VL]{1VL} Lafforgue, Vincent. "Chtoucas pour les groupes r\'eductifs et param\'etrisation de Langlands globale." Journal of the American Mathematical Society 31, no. 3 (2018): 719-891.

\bibitem[1Z]{1Z} Z\'abr\'adi, Gergely. "Multivariable $(\varphi,\Gamma) $-modules and products of Galois groups." Mathematical Research Letters 25, no. 2 (2018): 687-721.

\bibitem[1LL]{1LL} Lafforgue, Laurent. "Chtoucas de Drinfeld et correspondance de Langlands." Inventiones mathematicae 147 (2002): 1-241.

\bibitem[1T]{1T}  Tong, Xin. "Generalized Motives through Witt Vectors." arXiv preprint arXiv:2407.01417 (2024).

\bibitem[1RS]{1RS} Richarz, Timo, and Jakob Scholbach. "The motivic Satake equivalence." Mathematische Annalen 380, no. 3 (2021): 1595-1653.

\bibitem[EGH1]{EGH1} Emerton, Matthew, Toby Gee, and Eugen Hellmann. "An introduction to the categorical p-adic Langlands program." arXiv preprint arXiv:2210.01404 (2022).

\bibitem[2G]{2G} Alexander Grothendieck. "Letter to Serre." 1964.
\bibitem[2LH]{2LH} Li-Huerta, Siyan Daniel. "On close fields and the local Langlands correspondence." arXiv preprint arXiv:2407.07063 (2024).
\bibitem[2BS]{2BS} Bhatt, Bhargav, and Peter Scholze. "Prisms and prismatic cohomology." Annals of Mathematics 196, no. 3 (2022): 1135-1275.
\bibitem[2BL]{2BL} Bhatt, Bhargav, and Jacob Lurie. "Absolute prismatic cohomology." arXiv preprint arXiv:2201.06120 (2022).
\bibitem[2D]{2D} Drinfeld, Vladimir. "Prismatization." Selecta Mathematica 30, no. 3 (2024): 49.
\bibitem[2T1]{2T1} Tong, Xin. "Generalized Motives through Witt Vectors." arXiv preprint arXiv:2407.01417 (2024).
\bibitem[2BL2]{2BL2} Bhatt, Bhargav, and Jacob Lurie. "The prismatization of $ p $-adic formal schemes." arXiv preprint arXiv:2201.06124 (2022).
\bibitem[2A]{2A} Ayoub, Joseph. "Weil cohomology theories and their motivic Hopf algebroids." Indagationes Mathematicae (2024).
\bibitem[2F]{2F} Fontaine, Jean-Marc. "Sur certains types de repr\'esentations p-adiques du groupe de Galois d'un corps local; construction d'un anneau de Barsotti-Tate." Annals of Mathematics 115, no. 3 (1982): 529-577.

\bibitem[HJ]{HJ} Hansen, David, and Christian Johansson. "A note on the cohomology of moduli spaces of local shtukas." arXiv preprint arXiv:2404.04083 (2024).

\bibitem[3G]{3G} A. Grothendieck. "Letter to Serre." 1964. On motives.
\bibitem[3A]{3A} Ayoub, Joseph. "Weil cohomology theories and their motivic Hopf algebroids." Indagationes Mathematicae (2024).
\bibitem[31A]{31A} Ayoub, Joseph. "Motifs des vari\'et\'es analytiques rigides." Soci\'et\'e Math\'ematique de France. Memoires 140 (2015): 1-386.
\bibitem[3S]{3S} Scholze, Peter. "Berkovich motives." arXiv preprint arXiv:2412.03382 (2024).
\bibitem[3KL1]{3KL1} Kedlaya, K. S., and R. Liu. "Relative p-adic hodge theory: Foundations." Ast\'erisque 2015, no. 371 (2015): 1-245.
\bibitem[3KL2]{3KL2} Kedlaya, Kiran S., and Ruochuan Liu. "Relative p-adic Hodge theory, II: Imperfect period rings." arXiv preprint arXiv:1602.06899 (2016).
\bibitem[3BS]{3BS} Bhatt, Bhargav, and Peter Scholze. "Prisms and prismatic cohomology." Annals of Mathematics 196, no. 3 (2022): 1135-1275.
\bibitem[3BL]{3BL} Bhatt, Bhargav, and Jacob Lurie. "Absolute prismatic cohomology." arXiv preprint arXiv:2201.06120 (2022).
\bibitem[3D]{3D} Drinfeld, Vladimir. "Prismatization." Selecta Mathematica 30, no. 3 (2024): 49.
\bibitem[3LH]{3LH} Li-Huerta, Siyan Daniel. "On close fields and the local Langlands correspondence." arXiv preprint arXiv:2407.07063 (2024).
\bibitem[3CS]{3CS} Dustin Clausen and Peter Scholze. "Lectures on Condensed Mathematics." https://www.math.uni-bonn.de/people/scholze/Condensed.pdf.
\bibitem[3CS1]{3CS1} Dustin Clausen and Peter Scholze. "Analytic Stacks." https://people.mpim-bonn.mpg.de/scholze/AnalyticStacks.html.
\bibitem[3CS2]{3CS2} Dustin Clausen and Peter Scholze. "Lecture on Analytic Geometry." https://www.math.uni-bonn.de/people/scholze/Analytic.pdf.
\bibitem[3A1]{3A1} Abe, Tomoyuki. "Langlands correspondence for isocrystals and the existence of crystalline companions for curves." Journal of the American Mathematical Society 31, no. 4 (2018): 921-1057.

\bibitem[3ALBRCS]{3ALBRCS} Johannes Ansch\"utz, Arthur-C\'esar Le Bras, Juan Esteban Rodriguez Camargo and Peter Scholze. "Analytic prismatization." 
\bibitem[3F]{3F} Fontaine, Jean-Marc. "Sur certains types de repr\'esentations p-adiques du groupe de Galois d'un corps local; construction d'un anneau de Barsotti-Tate." Annals of Mathematics 115, no. 3 (1982): 529-577.
\bibitem[3S2]{3S2} Scholze, Peter. "\'Etale cohomology of diamonds." arXiv preprint arXiv:1709.07343 (2017).
\bibitem[3S1]{3S1} Peter Scholze. "Some remarks on prismatic cohomology of rigid spaces."
\bibitem[3Ta]{3Ta} Tate, John T. "p-Divisible groups." In Proceedings of a Conference on Local Fields: NUFFIC Summer School held at Driebergen (The Netherlands) in 1966, pp. 158-183. Berlin, Heidelberg: Springer Berlin Heidelberg, 1967.
\bibitem[3S3]{3S3} Scholze, Peter. "p-adic Hodge theory for rigid-analytic varieties." In Forum of Mathematics, Pi, vol. 1, p. e1. Cambridge University Press, 2013.
\bibitem[3SI]{3SI} Sagave, Steffen. "DG-algebras and derived $A_\infty$-algebras." Journal f\"ur die Reine und Angewandte Mathematik 2010, no. 639 (2010).

\bibitem[3V]{3V} Voevodsky, Vladimir. "A1-homotopy theory." In Proceedings of the international congress of mathematicians, vol. 1, pp. 579-604. 1998.

\bibitem[GRI]{GRI} Gaitsgory, Dennis, and Nick Rozenblyum. A study in derived algebraic geometry: Volume I: correspondences and duality. Vol. 221. American Mathematical Society, 2019.
\bibitem[GRII]{GRII} Gaitsgory, Dennis, and Nick Rozenblyum. "A study in derived algebraic geometry: Volume ii: Deformations, lie theory and formal geometry." American Mathematical Society (2017).
\bibitem[3BSI]{3BSI} Borger, James, and Arnab Saha. "Differential characters of Drinfeld modules and de Rham cohomology." Algebra \& Number Theory 13, no. 4 (2019): 797-837.



\end{thebibliography}







\end{document}